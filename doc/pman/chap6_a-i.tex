

\chapter{Detailed Function and Keyword Reference.}
\label{sec:six}
\setlength{\parindent}{0pt}

\newcommand{\rl}{\rule{\textwidth}{0.3mm}}

\newenvironment{desc}[1]
 {\begin{list}{}%
  {\renewcommand\makelabel[1]{{##1:}\hfil}%
   \settowidth\labelwidth{\makelabel{#1}}%
   \setlength\leftmargin{\labelwidth+\labelsep}}}%
 {\end{list}}




\rl
\section{Symbols}

\rl

\begin{desc}{Name/Symbol}
\item[Name/Symbol] \verb!+!

\item[Description] Adds two expressions together.  Also,
concatenates strings together.

\item[Usage]
\begin{verbatim}
<num1> + <num2>
<string1> + <string2>
<string1> + <num1>
\end{verbatim}
 Using other types of variables will cause errors.

\item[Example]
\begin{verbatim}
33 + 322                   --> 355
"Hello" + " " + "World"    --> "Hello World"
"Hello" + 33 + 322.5       --> "Hello355.5"
33 + 322.5 + "Hello"       --> "33322.5Hello"
\end{verbatim}

\item[See Also]     \verb!-!, \texttt{ToString()}

\end{desc}

\rl


\begin{desc}{Name/Symbol}

\item[Name/Symbol] \verb+-+

\item[Description]  Subtracts one expression from another

\item[Usage]        \verb!<num1> - <num2>!

\item[Example]     

\item[See Also]

\end{desc}

\rl     


\begin{desc}{Name/Symbol}

\item[Name/Symbol] \verb+/+ 

\item[Description]  Divides one expression by another

\item[Usage]        \verb+<expression> / <expression>+ 

\item[Example]
\begin{verbatim}
333 / 10    # == 33.3
\end{verbatim}

\item[See Also]

\end{desc} 

\rl

\begin{desc}{Name/Symbol}
   

\item[Name/Symbol] \verb+*+

\item[Description]        Multiplies two expressions together

\item[Usage]       \verb+<expression> * <expression>+

\item[Example]
\begin{verbatim}
32 * 2 # == 64
\end{verbatim}

\item[See Also]     

\end{desc}

\rl


\begin{desc}{Name/Symbol}

\item[Name/Symbol] \verb!^!

\item[Description]  Raises one expression to the power of  another expression

\item[Usage]       \verb!<expression> ^ <expression>!

\item[Example]
\begin{verbatim}
25 ^ 2  # == 625
\end{verbatim}

\item[See Also]    \texttt{Exp}, \texttt{NthRoot}

\end{desc}

\rl

\begin{desc}{Name/Symbol}

\item[Name/Symbol] \verb+;+ 

\item[Description]        Finishes a statement, can start new statement
                      on the same line (not needed at end of line)

\item[Usage]       

\item[Example]     

\item[See Also]

\end{desc}

\rl

\begin{desc}{Name/Symbol}     

\item[Name/Symbol] \verb!#!

\item[Description]   Comment indicator; anything until the next CR
	       following this character is ignored

\item[Usage]       

\item[Example]     

\item[See Also]

\end{desc} 


\rl
     
\begin{desc}{Name/Symbol}

\item[Name/Symbol] \verb!<-!                  

\item[Description]  The assignment operator.  Assigns a value to a variable\\
              N.B.: This two-character sequence takes the place of the
	      `\verb!=!' operator found in many programming languages.

\item[Usage]       

\item[Example]     

\item[See Also]  

\end{desc}   

\rl

\begin{desc}{Name/Symbol}

\item[Name/Symbol] \verb+( )+                  

\item[Description] Groups mathematical operations

\item[Usage]      \verb+(expression)+

\item[Example]
\begin{verbatim}
(3 + 22) * 4  # == 100
\end{verbatim}

\item[See Also]     

\end{desc}

\rl

\begin{desc}{Name/Symbol}

\item[Name/Symbol] \verb!{ }!                  

\item[Description] Groups a series of statements

\item[Usage]
\begin{verbatim}
{ statement1
  statement2
  statement3
}
\end{verbatim}
	     

\item[Example]     

\item[See Also]     
\end{desc}

\rl

\begin{desc}{Name/Symbol}

\item[Name/Symbol] \verb+[ ]+                 

\item[Description]  Creates a list. Closing \verb+]+ must be on
 	      same line as last element of list, even
	      for nested lists.

\item[Usage]       \verb+[<item1>, <item2>, ....]+
            

\item[Example]
\begin{verbatim}
[]                    #Creates an empty list
[1,2,3]               #Simple list
[[3,3,3],[2,2],0]     #creates a nested list structure
\end{verbatim}


\item[See Also]     \texttt{List()}
\end{desc}

\rl

\begin{desc}{Name/Symbol}

\item[Name/Symbol] 	\verb+<+ 

\item[Description] 	Less than.  Used to compare two numeric quantities.

\item[Usage]
\begin{verbatim}
3 < 5
3 < value
\end{verbatim}
             
\item[Example]
\begin{verbatim}
if(j < 33)
{
  Print ("j is less than 33.")
}
\end{verbatim}

See Also:     	\verb+>+, \verb+>=+, \verb+<=+, \verb+==+, \verb+~=+, \verb+!=+, \verb+<>+

\end{desc}

\rl



\begin{desc}{Name/Symbol}

\item[Name/Symbol] 	\verb!>!                    

\item[Description] 	Greater than. Used to compare two numeric quantities.

\item[Usage]
\begin{verbatim}
5 > 3
5 > value
\end{verbatim}

\item[Example]
\begin{verbatim}
if(j > 55)
{
 Print ("j is greater than 55.")
}
\end{verbatim}

\item[See Also]     	\verb+<+, \verb+>=+, \verb+<=+, \verb+==+, \verb+~=+, \verb+!=+, \verb+<>+
\end{desc}

\rl


\begin{desc}{Name/Symbol}

\item[Name/Symbol] 	\verb+<=+                   

\item[Description] 	Less than or equal to.

\item[Usage]
\begin{verbatim}
3<=5  
3<=value
\end{verbatim}

\item[Example]
\begin{verbatim}
if(j <= 33)
{
 Print ("j is less than or equal to 33.")
}
\end{verbatim}
	
\item[See Also]     	\verb+<, >, >=, ==, ~=, !=, <>+

\end{desc}

\rl

\begin{desc}{Name/Symbol}

\item[Name/Symbol] 	\verb+>=+                   

\item[Description] 	Greater than or equal to.

\item[Usage]
\begin{verbatim}
5>=3  
5>=value
\end{verbatim}

\item[Example]
\begin{verbatim}
if(j >= 55)
{
 Print ("j is greater than or equal to 55.")
}
\end{verbatim}

\item[See Also]     	\verb+<,+ \verb+>+, \verb+<=+, \verb+==+, \verb+~=+, \verb+!=+, \verb+<>+
\end{desc}

\rl

\begin{desc}{Name/Symbol}

\item[Name/Symbol] 	\verb+==+                   

\item[Description] 	Equal to.

\item[Usage]       	\verb+4 == 4+
		

\item[Example]
\begin{verbatim}
2 + 2 == 4
\end{verbatim}

\item[See Also]     	\verb+<,+ \verb+>+, \verb+>=+, \verb+<=+, \verb+~=+, \verb+!=+, \verb+<>+
\end{desc}

\rl

\begin{desc}{Name/Symbol}
\item[Name/Symbol]  	\verb+<>+, \verb+!=+, \verb+~=+

\item[Description]  	Not equal to.

\item[Usage]		

\item[Example]	

\item[See Also]     	\verb+<+, \verb+>+, \verb+>=+, \verb+<=+, \verb+==+

\end{desc}

\rl 

\section{A} 
\rl

\begin{desc}{Name/Symbol}

\item[Name/Symbol] 	\verb+Abs()+

\item[Description]   	Returns the absolute value of the number.

\item[Usage]
\begin{verbatim}
Abs(<num>)
\end{verbatim}        

\item[Example]
\begin{verbatim}
Abs(-300)  	# ==300
Abs(23)    	# ==23
\end{verbatim}

\item[See Also]     	\verb+Round()+, \verb+Floor()+, \verb+AbsFloor()+, \verb+Sign()+, \verb+Ceiling()+
\end{desc}

\rl


\begin{desc}{Name/Symbol}

\item[Name/Symbol]  	\verb+AbsFloor()+

\item[Description]  	Rounds \verb+<num>+ toward 0 to an integer.

\item[Usage]       	
\begin{verbatim}
AbsFloor(<num>)
\end{verbatim}

\item[Example]
\begin{verbatim}
AbsFloor(-332.7)   	# == -332
AbsFloor(32.88)    	# == 32
\end{verbatim}

\item[See Also]     	\verb+Round()+, \verb+Floor()+, \verb+Abs()+, \verb+Sign()+, \verb+Ceiling()+
\end{desc}

\rl


\begin{desc}{Name/Symbol}

\item[Name/Symbol] 	\verb+ACos()+ 

\item[Description]  	Inverse cosine of \verb+<num>+, in degrees.

\item[Usage]
\begin{verbatim}
ACos(<num>)
\end{verbatim}

\item[Example]	

\item[See Also]    	\verb+Cos()+, \verb+Sin()+, \verb+Tan()+, \verb+ATan()+, \verb+ATan()+ 

\end{desc}

\rl
 


\begin{desc}{Name/Symbol}

\item[Name/Symbol]  	\verb+AddObject()+

\item[Description] 	Adds a widget to a parent window.

\item[Usage]		

\item[Example]	

\item[See Also]    	\verb+RemoveObject()+
\end{desc}

\rl



\begin{desc}{Name/Symbol}
\item[Name/Symbol]  	\verb+and+
  
\item[Description]  	Logical and operator.

\item[Usage]       	
\begin{verbatim}
<expression> and <expression>
\end{verbatim}

\item[Example]	

\item[See Also]     	\verb+or+, \verb+not+

\end{desc}

\rl


\begin{desc}{Name/Symbol}

\item[Name/Symbol]  	\verb+Append+
  
\item[Description]  	Appends an item to a list.  Useful for constructing lists in conjunction with the loop statement.

\item[Usage] 
\begin{verbatim}
Append(<list>, <item>)
\end{verbatim}

\item[Example]
\begin{verbatim}
list <- Sequence(1,5,1)
double  <- []
loop(i, list)
{
 double <- Append(double, [i,i])
}
Print(double)
# Produces [[1,1],[2,2],[3,3],[4,4],[5,5]]
\end{verbatim}

\item[See Also]     	\verb+List()+, \verb+[ ]+, \verb+Merge()+
\end{desc}

\rl



\begin{desc}{Name/Symbol}

\item[Name/Symbol]  	\verb+ASin()+ 

\item[Description]  	Inverse Sine of \verb+<num>+, in degrees.

\item[Usage]
\begin{verbatim}
ASin(<num>)
\end{verbatim}

\item[Example]	

\item[See Also]    	 \verb+Cos()+, \verb+Sin()+, \verb+Tan()+, \verb+ATan()+, \verb+ACos()+, \verb+ATan()+ 
\end{desc}

\rl



\begin{desc}{Name/Symbol}

\item[Name/Symbol]  	\verb+ATan+ 

\item[Description]  	Inverse Tan of \verb+<num>+, in degrees.

\item[Usage]		

\item[Example]	

\item[See Also]    	\verb+Cos()+, \verb+Sin()+, \verb+Tan()+, \verb+ATan()+, \verb+ACos()+, \verb+ATan()+ 
\end{desc}

\rl

\section{B}
\rl


\begin{desc}{Name/Symbol}
\item[Name/Symbol]  	\verb+break+

\item[Description]  	Breaks out of a loop immediately.

\item[Usage]        	break

\item[Example]
\begin{verbatim}
loop(i ,[1,3,5,9,2,7])
{
 Print(i)
 if(i == 3) 
        {
         break
        }
}
\end{verbatim}

\item[See Also]   	\verb+loop+, \verb+return+
\end{desc}

\rl


\section{C}
\rl


\begin{desc}{Name/Symbol}
\item[Name/Symbol]  	\verb+Ceiling()+

\item[Description] 	Rounds \verb+<num>+ up to the next integer.

\item[Usage]
\begin{verbatim}
Ceiling(<num>)
\end{verbatim}

\item[Example] 
\begin{verbatim}
Ceiling(33.23)  	# == 34
Ceiling(-33.02) 	# == -33
\end{verbatim}

\item[See Also]     	\verb+Round()+, \verb+Floor()+, \verb+AbsFloor()+, \verb+Ceiling()+
\end{desc}

\rl

\begin{desc}{Name/Symbol}
\item[Name/Symbol]  	\verb+ChooseN()+

\item[Description] Samples \verb+<number>+ items from list, returning
  a list in the original order. Items are sampled without replacement, so
  once an item is chosen it will not be chosen again. If
  \verb+<number>+ is larger than the length of the list, the entire
  list is returned in order.  It differs from \verb+SampleN+ in that
  \verb+ChooseN+ returns items in the order they appeared in the
  originial list, but \verb+SampleN+ is shuffled. 

\item[Usage]       	
\begin{verbatim}
ChooseN(<list>, <n>)
\end{verbatim}

\item[Example]   	
\begin{verbatim}
ChooseN([1,1,1,2,2], 5)     # Returns 5 numbers
ChooseN([1,2,3,4,5,6,7], 3) # Returns 3 numbers from 1 and 7
\end{verbatim}

\item[See Also]    	\verb+SampleN()+, \verb+SampleNWithReplacement()+, \verb+Subset()+
\end{desc}

\rl

\begin{desc}{Name/Symbol}
\item[Name/Symbol]	\verb+Circle()+

\item[Description] Creates a circle for graphing at x,y with radius r.
  Circles must be added to a parent widget before it can be drawn; it
  may be added to widgets other than a base window. The properties of
  circles may be changed by accessing their properties directly,
  including the FILLED property which makes the object an outline
  versus a filled shape.


\item[Usage]
\begin{verbatim}
Circle(<x>, <y>, <r>,<color>)
\end{verbatim}

\item[Example]	
\begin{verbatim}
  
  c <- Circle(30,30,20, MakeColor(green))
  AddObject(c, win)
  Draw()

\end{verbatim}
\item[See Also]	\verb+Square()+, \verb+Ellipse()+, \verb+Rectangle()+, \verb+Line()+
\end{desc}

\rl



\begin{desc}{Name/Symbol}
\item[Name/Symbol]  	\verb+ClearEventLoop()+

\item[Description]  	NOT IMPLEMENTED. Advanced Event loop management.

\item[Usage]		

\item[Example]	

\item[See Also]	
\end{desc}

\rl



\begin{desc}{Name/Symbol}
\item[Name/Symbol]	\verb+CloseNetworkConnection()+

\item[Description]	Closes network connection

\item[Usage]
\begin{verbatim}
   CloseNetwork(<network>)
\end{verbatim}

\item[Example]	

\begin{verbatim}
  net <- WaitForNetworkConnection("localhost",1234)
  SendData(net,"Watson, come here. I need you.")
  CloseNetworkConnection(net)
\end{verbatim}
Also see nim.pbl for example of two-way network connection.
\item[See Also]
  \verb+ConnectToIP+,\verb+ConnectToHost+,\verb+GetData+,\verb+WaitForNetworkConnection+,
   \verb+SendData+,\verb+ConvertIPString+
\end{desc}

\rl



\begin{desc}{Name/Symbol}
\item[Name/Symbol]	\verb+ConnectToHost()+

\item[Description]	Connects to a host computer waiting for a
  connection on <port>, returning a network object that can be used to
  communicate.  Host is a text hostname, like \verb+"myname.indiana.edu"+, or
  use \verb+"localhost"+ to specify your current computer.

\item[Usage]
\begin{verbatim}
   ConnectToHost(<hostname>,<port>)
\end{verbatim}

\item[Example]	

  See nim.pbl for example of two-way network connection.
\begin{verbatim}
  net <- ConnectToHost("localhost",1234)
  dat <- GetData(net,20)
  Print(dat)
  CloseNetworkConnection(net)
\end{verbatim}

\item[See Also]
  \verb+ConnectToIP+,\verb+GetData+,\verb+WaitForNetworkConnection+,
   \verb+SendData+,\verb+ConvertIPString+,\verb+CloseNetworkConnection+
\end{desc}

\rl

\begin{desc}{Name/Symbol}
\item[Name/Symbol]	\verb+ConnectToIP()+

\item[Description]	Connects to a host computer waiting for a
  connection on <port>, returning a network object that can be used to
  communicate.  \verb+<ip>+ is a numeric ip address, which must be
  created with the \verb+ConvertIPString(ip)+ function. 

\item[Usage]
\begin{verbatim}
   ConnectToIP(<ip>,<port>)
\end{verbatim}

\item[Example]	

  See nim.pbl for example of two-way network connection.
\begin{verbatim}
  ip <- ConvertIPString("192.168.0.1")
  net <- ConnectToHost(ip,1234)
  dat <- GetData(net,20)
  Print(dat)
  CloseNetworkConnection(net)
\end{verbatim}

\item[See Also]
  \verb+ConnectToHost+,\verb+GetData+,\verb+WaitForNetworkConnection+,
   \verb+SendData+,\verb+ConvertIPString+,\verb+CloseNetworkConnection+
\end{desc}

\rl



\begin{desc}{Name/Symbol}
\item[Name/Symbol]	\verb+ConvertIPString()+

\item[Description]	Converts an IP address specified as a string into
  an integer that can be used by ConnectToIP.

\item[Usage]
\begin{verbatim}
   ConvertIPString(<ip-as-string>)
\end{verbatim}

\item[Example]	

  See nim.pbl for example of two-way network connection.
\begin{verbatim}
  ip <- ConvertIPString("192.168.0.1")
  net <- ConnectToHost(ip,1234)
  dat <- GetData(net,20)
  Print(dat)
  CloseNetworkConnection(net)
\end{verbatim}

\item[See Also]
  \verb+ConnectToHost+, \verb+ConnectToIP+,\verb+GetData+,\verb+WaitForNetworkConnection+,
   \verb+SendData+,\verb+ConvertIPString+,\verb+CloseNetworkConnection+
\end{desc}

\rl


\begin{desc}{Name/Symbol}
\item[Name/Symbol]  	\verb+Cos()+
			 
\item[Description] 	Cosine of \verb+<deg>+ degrees.

\item[Usage]		
\item[Example]	
\begin{verbatim}
  Cos(33.5)
  Cos(-32)
\end{verbatim}

\item[See Also]     	\verb+Sin()+, \verb+Tan()+, \verb+ATan()+, \verb+ACos()+, \verb+ATan()+
\end{desc}

\rl     


\begin{desc}{Name/Symbol}

\item[Name/Symbol] \verb+CR()+

\item[Description]  Produces a <number> linefeeds which can be added to a
  string and printed or saved to a file.

\item[Usage]        \verb!CR(<number>)!

\item[Example]     
\begin{verbatim}
         Print("Number: "  Tab(1) + number  + CR(2))
         Print("We needed space before this line.")
\end{verbatim}
\item[See Also]
\verb+Format()+, \verb+Tab()+
\end{desc}



\rl


\begin{desc}{Name/Symbol}
\item[Name/Symbol]  	\verb+CrossFactorWithoutDuplicates()+

\item[Description] 	This function takes a single list, and returns a list of all 
			pairs, excluding the pairs that have two of the same item. 
			To achieve the same effect but include the duplicates, use 
			\verb+DesignFullCounterBalance(x,x)+.

\item[Usage]
\begin{verbatim}
CrossFactorWithoutDuplicates(<list>)
\end{verbatim}

\item[Example]
\begin{verbatim}
CrossFactorWithoutDuplicates([a,b,c]) 
# == [[a,b],[a,c],[b,a],[b,c],[c,a],[c,b]]
\end{verbatim}

\item[See Also] \verb+DesignFullCounterBalance()+,
  \verb+DesignBalancedSampling()+, \verb+DesignGrecoLatinSquare()+,
  \verb+DesignLatinSquare()+, \verb+Repeat()+, \verb+RepeatList()+,\verb+LatinSquare()+
  \verb+Shuffle()+
\end{desc}

\rl
\section{D}
\rl


\begin{desc}{Name/Symbol}
\item[Name/Symbol]  	\verb+define+

\item[Description]  	Defines a user-specified function.

\item[Usage]
\begin{verbatim}
define functionname (parameters)
{
 statement1
 statement2
 statement3
       #Return statement is optional:
 return <value>
}
\end{verbatim}

\item[Example]    	See above.

\item[See Also]
\end{desc}   	



\begin{desc}{Name/Symbol}
\item[Name/Symbol]  	\verb+DegToRad()+

\item[Description]  	Converts degrees to radians.

\item[Usage]
\begin{verbatim}
DegToRad(<deg>)
\end{verbatim}

\item[Example]     	
\begin{verbatim}
DegToRad(180) # == 3.14159...
\end{verbatim}

\item[See Also]    	\verb+Cos()+, \verb+Sin()+, \verb+Tan()+, \verb+ATan()+, \verb+ACos()+, \verb+ATan()+ 
\end{desc}

\rl



\begin{desc}{Name/Symbol}
\item[Name/Symbol]  	\verb+DesignBalancedSampling()+

\item[Description] 	Samples elements "roughly" equally.
  		This function returns a list of repeated samples from
 		\verb+<treatment_list>+, such that each element in \verb+<treatment_list>+ 
		appears approximately equally.  Each element from 
		\verb+<treatment_list>+ is sampled once without replacement before 
		all elements are returned to the mix and sampling is repeated.  
		If there are no repeated items in \verb+<list>+, there will be no
 		consecutive repeats in the output.  The last repeat-sampling 
		will be truncated so that a \verb+<length>+-size list is returned.  
		If you don't want the repeated epochs this function provides, 
		Shuffle() the results.

\item[Usage]
\begin{verbatim}
DesignBalancedSampling(<list>, <length>)
\end{verbatim}

\item[Example]
\begin{verbatim}
DesignBalancedSampling([1,2,3,4,5],12)
# e.g., produces something like [5,3,1,4,2, 3,1,5,2,4, 3,1 ]
\end{verbatim}

\item[See Also]	\verb+CrossFactorWithoutDuplicates()+, \verb+DesignFullCounterBalance()+,
  		\verb+DesignGrecoLatinSquare()+, \verb+DesignLatinSquare()+, \verb+Repeat()+, 
		\verb+RepeatList()+, \verb+Shuffle()+,\verb+LatinSquare()+

\end{desc}

\rl




\begin{desc}{Name/Symbol}

\item[Name/Symbol]	\verb+DesignFullCounterbalance()+

\item[Description]	This takes two lists as parameters, and returns a nested list 
		of lists that includes the full counterbalancing of both 
		parameter lists.  Use cautiously; this gets very large.

\item[Usage]
\begin{verbatim}
DesignFullCounterbalance(<lista>, <listb>)
\end{verbatim}

\item[Example]
\begin{verbatim}
a <- [1,2,3]
b <- [9,8,7]
DesignFullCounterbalance(a,b)	# == [[1,9],[1,8],[1,7],
				#     [2,9],[2,8],[2,7],
				#     [3,9],[3,8],[3,7]]
\end{verbatim}

\item[See Also] \verb+CrossFactorWithoutDuplicates()+,\verb+LatinSquare()+
  \verb+DesignBalancedSampling()+, \verb+DesignGrecoLatinSquare()+,
  \verb+DesignLatinSquare()+, \verb+Repeat()+, \verb+RepeatList()+,
  \verb+Shuffle()+
\end{desc}

\rl




\begin{desc}{Name/Symbol}
\item[Name/Symbol]	\verb+DesignGrecoLatinSquare()+

\item[Description] This will return a list of lists formed by rotating
  through each element of the \verb+<treatment_list>+s, making a list
  containing all element of the list, according to a greco-latin
  square.  All lists must be of the same length.

\item[Usage]
\begin{verbatim}
DesignGrecoLatinSquare(<factor_list>, <treatment_list>, 
<treatment_list>)
\end{verbatim}

\item[Example]
\begin{verbatim}
x <- ["a","b","c"]
y <- ["p","q","r"]
z <- ["x","y","z"]
Print(DesignGrecoLatinSquare(x,y,z))
# produces:   	[[[a, p, x], [b, q, y], [c, r, z]], 
#               [[a, q, z], [b, r, x], [c, p, y]], 
#               [[a, r, y], [b, p, z], [c, q, x]]]
\end{verbatim}

\item[See Also] \verb+CrossFactorWithoutDuplicates()+,\verb+LatinSquare()+
  \verb+DesignFullCounterBalance()+, \verb+DesignBalancedSampling()+,
  \verb+DesignLatinSquare()+, \verb+Repeat()+, \verb+RepeatList()+,
  \verb+Shuffle()+
\end{desc}

\rl


\begin{desc}{Name/Symbol}
\item[Name/Symbol]	\verb+DesignLatinSquare()+

\item[Description] This returns return a list of lists formed by
  rotating through each element of \verb+<treatment_list>+, making a
  list containing all element of the list. Has no side effect on input
  lists.  This is implemented as a PEBL function in
  \texttt{pebl-lib/Design.pbl}

\item[Usage]
\begin{verbatim}
DesignLatinSquare(<treatment1_list>, <treatment2_list>)
\end{verbatim}

\item[Example]
\begin{verbatim}
order <- [1,2,3]
treatment <- ["A","B","C"]
design <- DesignLatinSquare(order,treatment)
# produces: [[[1, A], [2, B], [3, C]],
#            [[1, B], [2, C], [3, A]],
#            [[1, C], [2, A], [3, B]]]
\end{verbatim}

\item[See Also] \verb+CrossFactorWithoutDuplicates()+,
  \verb+DesignFullCounterBalance()+, \verb+DesignBalancedSampling()+,
  \verb+DesignGrecoLatinSquare()+, \verb+Repeat()+, \verb+LatinSquare()+
  \verb+RepeatList()+, \verb+Shuffle()+, \verb+Rotate()+
\end{desc}

\rl


\begin{desc}{Name/Symbol}
\item[Name/Symbol]	\verb+Div()+

\item[Description]  	NOT IMPLEMENTED.  Returns round(\verb+<num>/<mod>+)

\item[Usage]
\begin{verbatim}
Div(<num>, <mod>)
\end{verbatim}

\item[Example]	

\item[See Also]	\verb+Mod()+
\end{desc}

\rl




\begin{desc}{Name/Symbol}
\item[Name/Symbol]	\verb+Draw()+

\item[Description]	Redraws the screen or a specific widget.

\item[Usage]
\begin{verbatim}
Draw()
Draw(<object>)
\end{verbatim}

\item[Example]	

\item[See Also]	\verb+DrawFor()+, \verb+Show()+, \verb+Hide()+
\end{desc}

\rl



\begin{desc}{Name/Symbol}
\item[Name/Symbol]	\verb+DrawFor()+

\item[Description] Draws a screen or widget, returning after
  \verb+<cycles>+ refreshes. This function currently does not work as
  intended in the SDL implementation, because of a lack of control
  over the refresh blank.  It may work in the future.

\item[Usage]
\begin{verbatim}
DrawFor( <object>, <cycles>)
\end{verbatim}

\item[Example]	

\item[See Also]	\verb+Draw()+, \verb+Show()+, \verb+Hide()+
\end{desc}

\rl
\section{E}
\rl




\begin{desc}{Name/Symbol}
\item[Name/Symbol]	\verb+Ellipse()+
  
\item[Description]	Creates a ellipse for graphing at x,y with radii
  rx and ry. Ellipses are only currently definable oriented in
  horizontal/vertical directions.  Ellipses  must be added
  to a parent widget before it can be drawn; it may be added to
  widgets other than a base window.  The properties of ellipses may be
  changed by accessing their properties directly, including the FILLED
  property which makes the object an outline versus a filled shape.

\item[Usage]
\begin{verbatim}
Ellipse(<x>, <y>, <rx>, <ry>,<color>)
\end{verbatim}

\item[Example]	
\begin{verbatim}
  
  e <- Ellipse(30,30,20,10, MakeColor(green))
  AddObject(e, win)
  Draw()

\end{verbatim}
\item[See Also]	\verb+Square()+, \verb+Circle()+, \verb+Rectangle()+, \verb+Line()+
\end{desc}

\rl



\begin{desc}{Name/Symbol}
\item[Name/Symbol]	\verb+EndOfFile()+

\item[Description]	Returns true if at the end of a file.

\item[Usage]
\begin{verbatim}
EndOfFile(<filestream>)
\end{verbatim}

\item[Example]
\begin{verbatim}
while(not EndOfFile(fstream))
{
 Print(FileReadLine(fstream))
}
\end{verbatim}

\item[See Also]	
\end{desc}

\rl



\begin{desc}{Name/Symbol}
\item[Name/Symbol]	\verb+EndOfLine()+

\item[Description]	Returns true if at end of line.

\item[Usage]
\begin{verbatim}
EndOfLine(<filestream>)
\end{verbatim}

\item[Example]	

\item[See Also]	
\end{desc}

\rl




\begin{desc}{Name/Symbol}
\item[Name/Symbol]  	\verb+Exp()+

\item[Description]	$e$ to the power of \verb+<pow>+.

\item[Usage]
\begin{verbatim}
Exp(<pow>)
\end{verbatim}

\item[Example]
\begin{verbatim}
Exp(0) 		# == 1
Exp(3)		# == 20.0855
\end{verbatim}

\item[See Also]	\verb+Log()+
\end{desc}

\rl




\begin{desc}{Name/Symbol}
\item[Name/Symbol]	\verb+ExtractListItems()+

\item[Description]	Extracts items from a list, forming a new list. 
		The list \verb+<items>+ are the integers representing the
		indices that should be extracted.  
	     
\item[Usage]
\begin{verbatim}
ExtractListItems(<list>,<items>)
\end{verbatim}

\item[Example]
\begin{verbatim}
myList <- Sequence(101, 110, 1)
ExtractListItems(myList, [2,4,5,1,4])
# produces [102, 104, 105, 101, 104]
\end{verbatim}

\item[See Also]	\verb+Subset()+, \verb+SubList()+, \verb+SampleN()+
\end{desc}

\rl

\section{F}
\rl



\begin{desc}{Name/Symbol}
\item[Name/Symbol]	\verb+FileClose()+

\item[Description]	Closes a filestream  variable.  Be sure to 
		pass the variable name, not the filename.  

\item[Usage]
\begin{verbatim}
FileClose(<filestream>)
\end{verbatim}

\item[Example]
\begin{verbatim}
x <- FileOpenRead("file.txt")
# Do relevant stuff here.
FileClose(x)
\end{verbatim}

\item[See Also]	\verb+FileOpenAppend()+, \verb+FileOpenRead()+, \verb+FileOpenWrite()+

\end{desc}

\rl




\begin{desc}{Name/Symbol}
\item[Name/Symbol]	\verb+FileOpenAppend()+

\item[Description] Opens a filename, returning a stream that can be
  used for writing information.  Appends if the file already exists.

\item[Usage]
\begin{verbatim}
FileOpenAppend(<filename>)
\end{verbatim}

\item[Example]	

\item[See Also]	\verb+FileClose()+, \verb+FileOpenRead()+, \verb+FileOpenWrite()+
\end{desc}

\rl



\begin{desc}{Name/Symbol}
\item[Name/Symbol]	\verb+FileOpenRead()+

\item[Description]  	Opens a filename, returning  a stream to be used 
		for reading information.

\item[Usage]
\begin{verbatim}
FileOpenRead(<filename>)
\end{verbatim}

\item[Example]	

\item[See Also]	\verb+FileClose()+, \verb+FileOpenAppend()+, \verb+FileOpenWrite()+
\end{desc}

\rl


\begin{desc}{Name/Symbol}
\item[Name/Symbol]	\verb+FileOpenWrite()+

\item[Description] Opens a filename, returning a stream that can be
  used for writing information.  Overwrites if file already exists.

\item[Usage]
\begin{verbatim}
FileOpenWrite(<filename>)
\end{verbatim}

\item[Example]	

\item[See Also]	\verb+FileClose()+, \verb+FileOpenAppend()+, \verb+FileOpenRead()+
\end{desc}

\rl




\begin{desc}{Name/Symbol}
\item[Name/Symbol]	\verb+FilePrint()+

\item[Description]	Like \verb+Print+, but to a file.  Prints a string to a file, 
		with a carriage return at the end.
	
\item[Usage]
\begin{verbatim}
FilePrint(<filestream>, <value>)
\end{verbatim}

\item[Example]
\begin{verbatim}
FilePrint(fstream, "Another Line.")
\end{verbatim}

\item[See Also]	\verb+Print()+, \verb+FilePrint_()+
\end{desc}

\rl




\begin{desc}{Name/Symbol}
\item[Name/Symbol]	\verb+FilePrint_()+

\item[Description]	Like \verb+Print_+, but to a file.  Prints a string to a file,	without appending a newline character.
	
\item[Usage]
\begin{verbatim}
FilePrint_(<filestream>, <value>)
\end{verbatim}

\item[Example]
\begin{verbatim}
FilePrint_(fstream, "This line doesn't end.")
\end{verbatim}

\item[See Also]	\verb+Print_()+, \verb+FilePrint()+
\end{desc}

\rl




\begin{desc}{Name/Symbol}
\item[Name/Symbol]	\verb+FileReadCharacter()+

\item[Description]	Reads and returns a single character from a filestream.

\item[Usage]
\begin{verbatim}
FileReadCharacter(<filestream>)
\end{verbatim}

\item[Example]	

\item[See Also]	
\end{desc}

\rl




\begin{desc}{Name/Symbol}
\item[Name/Symbol]	\verb+FileReadLine()+

\item[Description]	Reads and returns a line from a file; all characters up
		until the next newline or the end of the file.

\item[Usage]
\begin{verbatim}
FileReadLine(<filestream>)
\end{verbatim}

\item[Example]	

\item[See Also]	
\end{desc}

\rl




\begin{desc}{Name/Symbol}
\item[Name/Symbol]  	\verb+FileReadList()+
 
\item[Description] Given a filename, will open it, read in all the
  items into a list (one item per line), and close the file afterward.
  Ignores blank lines or lines starting with \verb+#+.  Useful with a
  number of pre-defined data files stored in \verb+media/text/+.  See
  Section~\ref{sec:media}: Provided Media Files.

\item[Usage]
\begin{verbatim}
FileReadList(<filename>)
\end{verbatim}

\item[Example]
\begin{verbatim}
FileReadList("data.txt")
\end{verbatim}

\item[See Also]
\end{desc}

\rl


\begin{desc}{Name/Symbol}
\item[Name/Symbol]	\verb+FileReadTable()+

\item[Description]	Reads a table directly from a file. Data in file should
		separated by spaces.  Reads each line onto a sublist,
		with space-separated tokens as items in sublist.  Ignores
		blank lines or lines beginning with \verb+#+. Optionally,
		specify a token separator other than space.

\item[Usage]
\begin{verbatim}
FileReadTable(<filename>, <optional-separator>)
\end{verbatim}

\item[Example]
\begin{verbatim}
a <- FileReadTable("data.txt")
\end{verbatim}

\item[See Also]	\verb+FileReadList()+
\end{desc}

\rl




\begin{desc}{Name/Symbol}
\item[Name/Symbol]	\verb+FileReadText()+

\item[Description]	Returns all of the text from a file, ignoring any lines
		beginning with \verb+#+. Opens and closes the file transparently.

\item[Usage]
\begin{verbatim}
FileReadText(<filename>)
\end{verbatim}

\item[Example]
\begin{verbatim}
instructions <- FileReadText("instructions.txt")
\end{verbatim}

\item[See Also]	\verb+FileReadList()+, \verb+FileReadTable()+
\end{desc}

\rl



\begin{desc}{Name/Symbol}
\item[Name/Symbol]	\verb+FileReadWord()+

\item[Description]	Reads and returns  a `word' from a file; the next
		connected stream of characters not including a \verb+' '+
		or a newline. Will not read newline characters.

\item[Usage]
\begin{verbatim}
FileReadWord(<filestream>)
\end{verbatim}

\item[Example]	

\item[See Also]	\verb+FileReadLine()+, \verb+FileReadTable()+, \verb+FileReadList()+
\end{desc}

\rl




\begin{desc}{Name/Symbol}
\item[Name/Symbol]	\verb+FindInString()+

\item[Description]	Finds a token in a string, returning the position.

\item[Usage]
\begin{verbatim}
FindInString(<string>,<string>)
\end{verbatim}

\item[Example]
\begin{verbatim}
FindInString("about","bo") 	# == 2
\end{verbatim}

\item[See Also]	\verb+SplitString()+
\end{desc}

\rl




\begin{desc}{Name/Symbol}
\item[Name/Symbol]	\verb+First()+

\item[Description]	Returns the first item of a list.

\item[Usage]
\begin{verbatim}
First(<list>)
\end{verbatim}

\item[Example]
\begin{verbatim}
First([3,33,132])		# == 3
\end{verbatim}

\item[See Also]	\verb+Nth()+, \verb+Last()+
\end{desc}

\rl



\begin{desc}{Name/Symbol}
\item[Name/Symbol]	\verb+Flatten()+

\item[Description]
	Flattens nested list \verb+<list>+ to a single flat list.

\item[Usage]
\begin{verbatim}
Flatten(<list>)
\end{verbatim}

\item[Example]
\begin{verbatim}
Flatten([1,2,[3,4],[5,[6,7],8],[9]])	# == [1,2,3,4,5,6,7,8,9]
Flatten([1,2,[3,4],[5,[6,7],8],[9]])	# == [1,2,3,4,5,6,7,8,9]
\end{verbatim}

\item[See Also]	\verb+FlattenN()+, \verb+FoldList()+
\end{desc}

\rl




\begin{desc}{Name/Symbol}
\item[Name/Symbol]	\verb+FlattenN()+

\item[Description]	Flattens \verb+<n>+ levels of nested list \verb+<list>+. 

\item[Usage]
\begin{verbatim}
Flatten(<list>, <n>)
\end{verbatim}

\item[Example]
\begin{verbatim}
Flatten([1,2,[3,4],[5,[6,7],8],[9]],1) 
# == [1,2,3,4,5,[6,7],8,9]
\end{verbatim}

\item[See Also]	\verb+Flatten()+, \verb+FoldList()+
\end{desc}

\rl




\begin{desc}{Name/Symbol}
\item[Name/Symbol]	\verb+Floor()+

\item[Description]	Rounds \verb+<num>+ down to the next integer.

\item[Usage]
\begin{verbatim}
Floor(<num>)
\end{verbatim}

\item[Example]
\begin{verbatim}
Floor(33.23)	# == 33
Floor(3.999)  	# ==3
Floor(-32.23) 	# == -33
\end{verbatim}
 
\item[See Also]	\verb+AbsFloor()+, \verb+Round()+, \verb+Ceiling()+
\end{desc}

\rl


\begin{desc}{Name/Symbol}
\item[Name/Symbol]	\verb+FoldList()+

\item[Description]	Folds a list into equal-length sublists.

\item[Usage]
\begin{verbatim}
FoldList(<list>, <size>)
\end{verbatim}

\item[Example]
\begin{verbatim}
FoldList([1,2,3,4,5,6,7,8],2)	# == [[1,2],[3,4],[5,6],[7,8]]
\end{verbatim}
 
\item[See Also]	\verb+FlattenN()+, \verb+Flatten()+
\end{desc}

\rl



\begin{desc}{Name/Symbol}
\item[Name/Symbol]	\verb+Format()+            

\item[Description]	Formats the printing of values to ensure the
  proper spacing. It will either truncate or pad <value> with spaces
  so that it ends up exactly <length> characters long.  Character
  padding is at the end
 

\item[Usage]
\begin{verbatim}
Format(<value>, <length>)
\end{verbatim}

\item[Example]	
\begin{verbatim}

  x <- 33.23425225
  y <- 23.3
  Print("["+Format(x,5)+"]")
  Print("["+Format(y,5)+"]")
  ## Output: 
  ## [33.23 ]
  ## [23.3  ]
\end{verbatim}
         

\item[See Also]	
\verb+CR()+ \verb+Tab()+
\end{desc}

\rl
\section{G}
\rl


\begin{desc}{Name/Symbol}
\item[Name/Symbol]	\verb+GetCursorPosition()+

\item[Description]	Returns an integer specifying where in a textbox the edit cursor is.  The value indicates which character it is on.

\item[Usage]
\begin{verbatim}
GetCursorPosition(<textbox>)
\end{verbatim}

\item[Example]	

\item[See Also]	\verb+SetCursorPosition()+, \verb+MakeTextBox()+, \verb+SetText()+
\end{desc}

\rl


\begin{desc}{Name/Symbol}
\item[Name/Symbol]	\verb+GetData()+

\item[Description]	Gets Data from network connection.  Example of
  usage in demo/nim.pbl.

\item[Usage]
\begin{verbatim}
val <- GetData(<network>,<size>)
\end{verbatim}

\item[Example]	

On 'server':
\begin{verbatim}
  net <- WaitForNetworkConnection("localhost",1234)
  SendData(net,"Watson, come here. I need you.")
  value <-  GetData(net,10)
  Print(value)

\end{verbatim}
On Client:
\begin{verbatim}
  net <- ConnectToHost("localhost",1234)
  value <-  GetData(net,20)
  Print(value)
##should print out "Watson, come here. I need you."
\end{verbatim}
\item[See Also]
  \verb+ConnectToIP+,\verb+ConnectToHost+,\verb+WaitForNetworkConnection+,
   \verb+SendData+,\verb+ConvertIPString+, \verb+CloseNetworkConnection+
\end{desc}

\rl


\begin{desc}{Name/Symbol}
\item[Name/Symbol]	\verb+GetInput()+

\item[Description]	Allows user to type input into a textbox.

\item[Usage]
\begin{verbatim}
GetInput(<textbox>,<escape-key>)
\end{verbatim}

\item[Example]	

\item[See Also]	\verb+SetEditable()+, \verb+GetCursorPosition()+, \verb+MakeTextBox()+, \verb+SetText()+
\end{desc}

\rl



\begin{desc}{Name/Symbol}
\item[Name/Symbol]	\verb+GetNIMHDemographics()+

\item[Description]	Gets demographic information that are normally required for NIMH-related research.  Currently are gender (M/F/prefer not to say), ethnicity (Hispanic or not), and race (A.I./Alaskan, Asian/A.A., Hawaiian, black/A.A., white/Caucasian, other).  
		It then prints their responses in a single line in the demographics file, along with any special code you supply and a time/date stamp. This code might include a subject number, experiment number, or something else, but many informed consent forms assure the subject that this information cannot be tied back to them or their data, so be careful about what you record. The file output will look something like: 
\begin{verbatim}
---- 
x0413 Thu Apr 22 17:58:15 2004 1 Y 4 
x0413 Thu Apr 23 17:58:20 2004 3 Y 5 
x0413 Thu Apr 24 12:41:30 2004 2 Y 5 
x0413 Thu Apr 24 14:11:54 2004 2 N 5 
---- 
\end{verbatim}


	The first column is the user-specified code (in this 
	case, indicating the experiment number).  The middle columns 
	indicate date/time, and the last three columns indicate 
	gender (M, F, other), Hispanic (Y/N), and race.

\item[Usage]
\begin{verbatim}
GetNIMHDemographics(<code-to-print-out>, <window>, <filename>)
\end{verbatim} 

\item[Example]
\begin{verbatim}
GetNIMHDemographics("x0413", gwindow, "x0413-demographics.txt")
\end{verbatim}

\item[See Also]	
\end{desc}

\rl




\begin{desc}{Name/Symbol}
\item[Name/Symbol]	\verb+GetPEBLVersion()+

\item[Description]	Returns a string describing which version of PEBL you are running.

\item[Usage]
\begin{verbatim}
GetPEBLVersion() 
\end{verbatim}

\item[Example]
\begin{verbatim}
Print(GetPEBLVersion())
\end{verbatim}

\item[See Also]	\verb+TimeStamp()+
\end{desc}

\rl



\begin{desc}{Name/Symbol}
\item[Name/Symbol]	\verb+GetSize()+

\item[Description] Returns a list of \verb+[height, width]+,
  specifying the size of the widget.

\item[Usage]
\begin{verbatim}
GetSize(<widget>)
\end{verbatim}

\item[Example]
\begin{verbatim}
image <- MakeImage("stim1.bmp")
xy <- GetSize(image)
x <- Nth(xy, 1)
y <- Nth(xy, 2)
\end{verbatim}

\item[See Also]	
\end{desc}

\rl



\begin{desc}{Name/Symbol}
\item[Name/Symbol]	\verb+GetText()+

\item[Description]	Returns the text stored in a text object 
		(either a textbox or a label).

\item[Usage]
\begin{verbatim}
GetText(<widget>)
\end{verbatim}

\item[Example]	

\item[See Also]	\verb+SetCursorPosition()+, \verb+GetCursorPosition()+, \verb+SetEditable()+, \verb+MakeTextBox()+
\end{desc}

\rl



\begin{desc}{Name/Symbol}
\item[Name/Symbol]	\verb+GetTime()+

\item[Description] Gets time, in milliseconds, from when PEBL was
  initialized.  Do not use as a seed for the RNG, because it will tend
  to be about the same on each run. Instead, use \verb+RandomizeTimer()+.

\item[Usage]
\begin{verbatim}
GetTime()
\end{verbatim}

\item[Example]
\begin{verbatim}
a <- GetTime()
WaitForKeyDown("A")
b <- GetTime()
Print("Response time is: " + (b - a))
\end{verbatim}

\item[See Also]	\verb+TimeStamp()+
\end{desc}

\rl

\section{H}
\rl



\begin{desc}{Name/Symbol}
\item[Name/Symbol]	\verb+Hide()+ 

\item[Description]	Makes an object invisible, so it will not be drawn.

\item[Usage]
\begin{verbatim}
Hide(<object>)
\end{verbatim}

\item[Example]
\begin{verbatim}
window <- MakeWindow()
image1  <- MakeImage("pebl.bmp")
image2  <- MakeImage("pebl.bmp")
AddObject(image1, window)
AddObject(image2, window)
Hide(image1)
Hide(image2)
Draw()		# empty screen will be drawn.
	
Wait(3000)
Show(image2)
Draw()		# image2 will appear.

Hide(image2)
Draw()		# image2 will disappear.

Wait(1000)
Show(image1)
Draw()		# image1 will appear.
\end{verbatim}
 
\item[See Also]	\verb+Show()+
\end{desc}

\rl


\section{I}
\rl


\begin{desc}{Name/Symbol}
\item[Name/Symbol]	\verb+if+ 

\item[Description]	Simple conditional test.

\item[Usage]
\begin{verbatim}
if(test)
{
 statements
 to
 be 
 executed
}
\end{verbatim}

\item[Example]	

\item[See Also]	
\end{desc}

\rl




\begin{desc}{Name/Symbol}
\item[Name/Symbol]	\verb+if...elseif...else+            

\item[Description] Complex conditional test.  Be careful of spacing
  the else---if you put carriage returns on either side of it, you
  will get a syntax error. The \verb+elseif+ is optional, but
  multiple \verb+elseif+ statements can be strung together.  The
  \verb+else+ is also optional, although only one can appear.

\item[Usage]
\begin{verbatim}
if(test)
{
 statements if true
} elseif (newtest) {
 statements if newtest true; test false
} else {
 other statements
} 
\end{verbatim}

\item[Example]	
\begin{verbatim}
 if(3 == 1) {
             Print("ONE")
  }elseif(3==4){
             Print("TWO")
  }elseif(4==4){
             Print("THREE")
  }elseif(4==4){
             Print("FOUR")
  }else{Print("FIVE")}
\end{verbatim}
\item[See Also]	
if
\end{desc}

\rl


\begin{desc}{Name/Symbol}
\item[Name/Symbol]	\verb+IsAnyKeyDown()+

\item[Description]	

\item[Usage]		

\item[Example]	

\item[See Also]	
\end{desc}

\rl


\begin{desc}{Name/Symbol}
\item[Name/Symbol]	\verb+IsAudioOut()+

\item[Description]	Tests whether \verb+<variant>+ is a AudioOut stream.

\item[Usage]
\begin{verbatim}
IsAudioOut(<variant>)
\end{verbatim}

\item[Example]
\begin{verbatim}
if(IsAudioOut(x))
{
 Play(x)
}
\end{verbatim}

\item[See Also] \verb+IsColor()+, \verb+IsImage()+,
  \verb+IsInteger()+, \verb+IsFileStream()+, \verb+IsFloat()+,
  \verb+IsFont()+, \verb+IsLabel()+, \verb+IsList()+,
  \verb+IsNumber()+, \verb+IsString()+, \verb+IsTextBox()+,
  \verb+IsWidget()+
\end{desc}

\rl


\begin{desc}{Name/Symbol}
\item[Name/Symbol]	\verb+IsColor()+

\item[Description]	Tests whether \verb+<variant>+ is a Color.

\item[Usage]
\begin{verbatim}
IsColor(<variant>)
\end{verbatim}

\item[Example]
\begin{verbatim}
if(IsColor(x)
{
 gWin <- MakeWindow(x)
}
\end{verbatim}

\item[See Also] \verb+IsAudioOut()+, \verb+IsImage()+,
  \verb+IsInteger()+, \verb+IsFileStream()+, \verb+IsFloat()+,
  \verb+IsFont()+, \verb+IsLabel()+, \verb+IsList()+,
  \verb+IsNumber()+, \verb+IsString()+, \verb+IsTextBox()+,
  \verb+IsWidget()+
\end{desc}

\rl




\begin{desc}{Name/Symbol}
\item[Name/Symbol]	\verb+IsImage()+

\item[Description]	Tests whether \verb+<variant>+ is an Image.

\item[Usage]
\begin{verbatim}
IsImage(<variant>)
\end{verbatim}

\item[Example]	
\begin{verbatim}
if(IsImage(x))
{
 AddObject(gWin, x)
}
\end{verbatim}

\item[See Also] \verb+IsAudioOut()+, \verb+IsColor()+,
  \verb+IsInteger()+, \verb+IsFileStream()+, \verb+IsFloat()+,
  \verb+IsFont()+, \verb+IsLabel()+, \verb+IsList()+,
  \verb+IsNumber()+, \verb+IsString()+, \verb+IsTextBox()+,
  \verb+IsWidget()+
\end{desc}

\rl




\begin{desc}{Name/Symbol}
\item[Name/Symbol]	\verb+IsInteger()+

\item[Description] Tests whether \verb+<variant>+ is an integer type.
  Note: a number represented internally as a floating-point type whose
  is an integer will return false.  Floating-point numbers can be
  converted to internally- represented integers with the
  \verb+ToInteger()+ or \verb+Round()+ commands.
 
\item[Usage]		
\begin{verbatim}
IsInteger(<variant>)
\end{verbatim}

\item[Example]
\begin{verbatim}
x <- 44
y <- 23.5
z <- 6.5
test <- x + y + z 
	
IsInteger(x)		# true
IsInteger(y)		# false
IsInteger(z)		# false
IsInteger(test)		# false
\end{verbatim}

\item[See Also] \verb+IsAudioOut()+, \verb+IsColor()+,
  \verb+IsImage()+, \verb+IsFileStream()+, \verb+IsFloat()+,
  \verb+IsFont()+, \verb+IsLabel()+, \verb+IsList()+,
  \verb+IsNumber()+, \verb+IsString()+, \verb+IsTextBox()+,
  \verb+IsWidget()+
\end{desc}

\rl




\begin{desc}{Name/Symbol}
\item[Name/Symbol]	\verb+IsFileStream()+

\item[Description]	Tests whether \verb+<variant>+ is a FileStream object.

\item[Usage]		
\begin{verbatim}
IsFileStream(<variant>)
\end{verbatim}

\item[Example]
\begin{verbatim}
if(IsFileStream(x))
{
 Print(FileReadWord(x)
}
\end{verbatim}

\item[See Also] \verb+IsAudioOut()+, \verb+IsColor()+,
  \verb+IsImage()+, \verb+IsInteger()+, \verb+IsFloat()+,
  \verb+IsFont()+, \verb+IsLabel()+, \verb+IsList()+,
  \verb+IsNumber()+, \verb+IsString()+, \verb+IsTextBox()+,
  \verb+IsWidget()+
\end{desc}

\rl




\begin{desc}{Name/Symbol}
\item[Name/Symbol]	\verb+IsFloat()+

\item[Description] Tests whether \verb+<variant>+ is a floating-point
  value. Note that floating-point can represent integers with great
  precision, so that a number appearing as an integer can still be a
  float.

\item[Usage]
\begin{verbatim}
IsFloat(<variant>)
\end{verbatim}

\item[Example]
\begin{verbatim}
x <- 44
y <- 23.5
z <- 6.5
test <- x + y + z 

IsFloat(x)     	# false
IsFloat(y)     	# true
IsFloat(z)     	# true
IsFloat(test)  	# true
\end{verbatim}

\item[See Also] \verb+IsAudioOut()+, \verb+IsColor()+,
  \verb+IsImage()+, \verb+IsInteger()+, \verb+IsFileStream()+,
  \verb+IsFont()+, \verb+IsLabel()+, \verb+IsList()+,
  \verb+IsNumber()+, \verb+IsString()+, \verb+IsTextBox()+,
  \verb+IsWidget()+
\end{desc}

\rl




\begin{desc}{Name/Symbol}
\item[Name/Symbol]	\verb+IsFont()+

\item[Description]	Tests whether \verb+<variant>+ is a Font object.

\item[Usage]
\begin{verbatim}
IsFont(<variant>)
\end{verbatim}

\item[Example]	
\begin{verbatim}
if(IsFont(x))
{
 y <- MakeLabel("stimulus", x)
}
\end{verbatim}

\item[See Also] \verb+IsAudioOut()+, \verb+IsColor()+,
  \verb+IsImage()+, \verb+IsInteger()+, \verb+IsFileStream()+,
  \verb+IsFloat()+, \verb+IsLabel()+, \verb+IsList()+,
  \verb+IsNumber()+, \verb+IsString()+, \verb+IsTextBox()+,
  \verb+IsWidget()+
\end{desc}

\rl




\begin{desc}{Name/Symbol}
\item[Name/Symbol]	\verb+IsKeyDown()+

\item[Description]	

\item[Usage]		

\item[Example]	

\item[See Also]	\verb+IsKeyUp()+
\end{desc}

\rl


\begin{desc}{Name/Symbol}
\item[Name/Symbol]	\verb+IsKeyUp()+

\item[Description]	

\item[Usage]		

\item[Example]	

\item[See Also]	\verb+IsKeyDown()+
\end{desc}

\rl


\begin{desc}{Name/Symbol}
\item[Name/Symbol]	\verb+IsLabel()+

\item[Description]	Tests whether \verb+<variant>+ is a text Label object.

\item[Usage]		
\begin{verbatim}
IsLabel(<variant>)
\end{verbatim}

\item[Example]	
\begin{verbatim}
if(IsLabel(x)
{
 text <- GetText(x)
}
\end{verbatim}

\item[See Also] \verb+IsAudioOut()+, \verb+IsColor()+,
  \verb+IsImage()+, \verb+IsInteger()+, \verb+IsFileStream()+,
  \verb+IsFloat()+, \verb+IsFont()+, \verb+IsList()+,
  \verb+IsNumber()+, \verb+IsString()+, \verb+IsTextBox()+,
  \verb+IsWidget()+
\end{desc}

\rl


\begin{desc}{Name/Symbol}
\item[Name/Symbol]	\verb+IsList()+

\item[Description]	Tests whether \verb+<variant>+ is a PEBL list.

\item[Usage]
\begin{verbatim}
IsList(<variant>)
\end{verbatim}

\item[Example]	
\begin{verbatim}
if(IsList(x))
{
 loop(item, x)
 {
  Print(item)
 }
}
\end{verbatim}

\item[See Also] \verb+IsAudioOut()+, \verb+IsColor()+,
  I\verb+sImage()+, \verb+IsInteger()+, \verb+IsFileStream()+,
  \verb+IsFloat()+, \verb+IsFont()+, \verb+IsLabel()+,
  \verb+IsNumber()+, \verb+IsString()+, \verb+IsTextBox()+,
  \verb+IsWidget()+
\end{desc}

\rl


\begin{desc}{Name/Symbol}
\item[Name/Symbol]	\verb+IsMember()+

\item[Description]	Returns true if \verb+<element>+ is a member of \verb+<list>+.

\item[Usage]		
\begin{verbatim}
IsMember(<element>,<list>)
\end{verbatim}

\item[Example]	
\begin{verbatim}
IsMember(2,[1,4,6,7,7,7,7])		# false
IsMember(2,[1,4,6,7,2,7,7,7]) 		# true
\end{verbatim}

\item[See Also]	
\end{desc}

\rl


\begin{desc}{Name/Symbol}
\item[Name/Symbol]	\verb+IsNumber()+

\item[Description]	Tests whether \verb+<variant>+ is a number, either a
		floating-point or an integer.

\item[Usage]		
\begin{verbatim}
IsNumber(<variant>)
\end{verbatim}

\item[Example]	
\begin{verbatim}
if(IsNumber(x))
{
 Print(Sequence(x, x+10, 1))
}
\end{verbatim}

\item[See Also] \verb+IsAudioOut()+, \verb+IsColor()+,
  \verb+IsImage()+, \verb+IsInteger()+, \verb+IsFileStream()+,
  \verb+IsFloat()+, \verb+IsFont()+, \verb+IsLabel()+,
  \verb+IsList()+, \verb+IsString()+, \verb+IsTextBox()+,
  \verb+IsWidget()+
\end{desc}

\rl




\begin{desc}{Name/Symbol}
\item[Name/Symbol]	\verb+IsString()+

\item[Description]	Tests whether \verb+<variant>+ is a text string.

\item[Usage]		
\begin{verbatim}
IsString(<variant>)
\end{verbatim}

\item[Example]	
\begin{verbatim}
if(IsString(x))
{
 tb <- MakeTextBox(x, 100, 100)
}
\end{verbatim}

\item[See Also]	\verb+IsAudioOut()+, \verb+IsColor()+, \verb+IsImage()+, \verb+IsInteger()+, 
		\verb+IsFileStream()+, \verb+IsFloat()+, \verb+IsFont()+, \verb+IsLabel()+,
		\verb+IsList()+, \verb+IsNumber()+, \verb+IsTextBox()+, \verb+IsWidget()+
\end{desc}

\rl




\begin{desc}{Name/Symbol}
\item[Name/Symbol]	\verb+IsTextBox()+

\item[Description]	Tests whether \verb+<variant>+ is a TextBox Object

\item[Usage]
\begin{verbatim}
IsTextBox(<variant>)
\end{verbatim}

\item[Example]	
\begin{verbatim}
if(IsTextBox(x))
{
 Print(GetText(x))
}
\end{verbatim}

\item[See Also] \verb+IsAudioOut()+, \verb+IsColor()+,
  \verb+IsImage()+, \verb+IsInteger()+, \verb+IsFileStream()+,
  \verb+IsFloat()+, \verb+IsFont()+,\verb+ IsLabel()+,
  \verb+IsList()+, \verb+IsNumber()+, \verb+IsString()+,
  \verb+IsWidget()+
\end{desc}

\rl


\begin{desc}{Name/Symbol}
\item[Name/Symbol]	\verb+IsWidget+

\item[Description]	Tests whether \verb+<variant>+ is any kind of a widget object
		(image, label, or textbox).

\item[Usage]		
\begin{verbatim}
IsWidget(<variant>)
\end{verbatim}

\item[Example]	
\begin{verbatim}
if(IsWidget(x))
{
 Move(x, 200,300)
}
\end{verbatim}

\item[See Also] \verb+IsAudioOut()+, \verb+IsColor()+,
  \verb+IsImage()+, \verb+IsInteger()+, \verb+IsFileStream()+,
  \verb+IsFloat()+, \verb+IsFont()+, \verb+IsLabel()+,
  \verb+IsList()+, \verb+IsNumber()+, \verb+IsString()+,
  \verb+IsTextBox()+
\end{desc}

\rl




%%% Local Variables: 
%%% mode: latex
%%% TeX-master: "main"
%%% End: 
