\chapter{About}

PEBL (Psychology Experiment Building Language) is a cross-platform,
open-source programming language and execution environment for
constructing programs to conduct a wide range of archetypal psychology
experiments. It is entirely free of charge, and may be modified to
suit your needs as long as you follow the terms of the GPL, under
which the source code is licensed. PEBL is written primarily in
\texttt{C++}, but requires a few other tools (\texttt{flex},
\texttt{yacc}) and libraries (\texttt{SDL}, \texttt{SDL\_image},
\texttt{SDL\_gfx}, and \texttt{SDL\_ttf}) to use.  In addition, a set
of audio recording functions are available using the (now old and
basically unmaintained) sdl\_audioin library.  Finally, the
\texttt{waave} library optionally supports movie playback on linux and
windows.


It currently compiles and runs on Linux (using \texttt{g++}), Mac OSX
(using xcode), and Microsoft Windows (using \texttt{code:blocks} and
\texttt{mingw}) platforms using free tools. It has been developed
primarily by Shane T.~Mueller,
Ph.D. (\href{mailto:smueller@obereed.net}{smueller@obereed.net}). This
document was prepared with editorial and formatting help from Gulab
Parab and Samuele Carcagno. In addition, much of the material in the chapter on the PEBL Test battery was contributed by Bryan Rowley. Contributions are welcome and encouraged.

\chapter{Usage}
Most users will be able to download a precompiled version of PEBL and
run experiments directly. Some advanced users may wish to compile
their own version, however.  The next section describes how to do this.

\sect{How to Compile}

Currently, there is no automated compile procedure.  PEBL requires the
\texttt{SDL}, \texttt{SDL-image}, \texttt{SDL-gfx} \texttt{SDL\_net},
 \texttt{SDL\_audioin} and
\texttt{SDL-ttf} libraries and development headers.  It also uses
\texttt{flex} and \texttt{bison}, but you can compile without these
tools. PEBL compiles on both Linux and Windows using the free
\texttt{gcc} compiler; on windows this is most easily supported through the code:blocks IDE. Note that \texttt{SDL-image} may require
\texttt{jpeg}, \texttt{png}, and a \texttt{zlib} compression library,
while \texttt{SDL-ttf} uses \texttt{truetype 2.0}.

\subsection{Linux}

PEBL should compile by typing `\texttt{make}' in its base directory
once all requisite tools are installed and the source distribution is
uncompressed. Currently, PEBL does not use autotools, so its make
system is rather brittle. Assistance is welcome.

On Linux, compiling will fail if you don't have an \texttt{/obj}
directory and all the appropriate subdirectories (that mirror the main
tree.)  These will not exist if you check out from CVS.

\subsection{Microsoft Windows}

On Microsoft Windows, PEBL is designed to be compiled using the Free
IDE \texttt{code:blocks}.  A code:blocks project file is included in the source code directory. Email the PEBL list for more details.

\subsection{Mac OSX}

Originally, PEBL compiled to a command-line function.  Since 0.12, PEBL will compile to a .app package using xcode.  An xcode package is available in the source package.


\sect{Installation}
\subsection{Linux}

On Linux, there are .deb packages available for debian.  However, it is fairly easy to 
compile and install from source. To begin, be sure that all the sdl packages are installed.  Then, go to the main pebl directory and type:
\begin{verbatim}
>make
>sudo make install
\end{verbatim}
Once installed, you can install the test battery into \texttt{Documents/pebl-exp.X} using the command pebl --install.

\subsection{Microsoft Windows}

In Microsoft Windows, we provide an installer package that contains
all necessary executable binary files and \texttt{.dlls}. This
installer places PEBL in \texttt{c:\char92 Program Files\char92PEBL},
and creates a directory \texttt{pebl-exp.X} in \texttt{My Documents}
with a shortcut that allows PEBL to be launched and programs that
reside there to be run.

\subsection{Macintosh OSX}

For OSX, we provide a \texttt{.app} package that can be dragged into your Applications folder.  The first time any user  runs pebl, it gives the option to install the battery and other files into Documentspebl-exp.X.  Afterward, it will run the launcher from that directory.

\sect{How to Run a PEBL Program}


The simplest way to run any PEBL script is via the launcher, which is available on all platforms. The launcher is covered in detail in Chapter 6. But, you can also launch experiments individually on each platform.

\subsection{Linux}

If you have installed PEBL into \texttt{/usr/local/bin}, you will be able to
invoke PEBL by typing `\texttt{pebl}' at a command line.  PEBL requires you to
specify one or more source files that it will compile and run, e.g., the
command:
\begin{verbatim}
  > pebl stroop.pbl library.pbl
\end{verbatim}
will load the experiment described in \texttt{stroop.pbl}, and will load the
supplementary library functions in \texttt{library.pbl}.

Additionally, PEBL can take the \texttt{-v} or \texttt{-V} command-line parameter, which allows you to pass values into the script.  This is useful for
entering subject numbers and condition types using an outside program
like a bash script (possibly one that invokes dialog or zenity). A sample zenity script that asks for subject number and then runs a sample experiment which uses that input resides in the \texttt{bin} directory. The script can be edited to use fullscreen mode or change the display dimensions, for example. See Section~\ref{sec:2.5}: Command-Line Arguments.

You can also specify directories without a filename on the command-line (as long as they end with `\texttt{/}').  Doing so will add that directory to the search path when files are opened.

\subsection{ Microsoft Windows}

PEBL can be launched from the command line in Windows by going to the
\texttt{pebl\char92bin} directory and typing `\texttt{pebl.exe}'.
PEBL requires you to specify one or more source files that it will
compile and run.  For example, the command
\begin{verbatim}
  > pebl.exe stroop.pbl library.pbl
\end{verbatim}
loads the experiment described in \texttt{stroop.pbl}, and loads supplementary library functions in \texttt{library.pbl}.

Additionally, PEBL can take the \texttt{-v} or \texttt{-V}
command-line parameter, which allows you to pass values in to the
script.  This is useful for entering condition types using an outside
program like a batch file. the \texttt{-s} and \texttt{-S} allow one
to specify a subject code, which gets bound to the gSubNum variable.
If no value is specified, gSubNum is initialized to 0.  You can also
specify directories without a file (as long as they end with
`\texttt{\char92}').  Doing so will add that directory to the search
path when files are opened. See Section \ref{sec:2.5}: Command-Line
Arguments.

Launching programs from the command-line on Windows is cumbersome.
One easy way to launch PEBL on Windows is to create a shortcut to the
executable file and then edit the properties so that the shortcut
launches PEBL with the proper script and command-line parameters.
Another way is to write and launch a batch file, which is especially
useful if you wish to enter configuration data before loading the
script.


\subsection{ Macintosh OSX}

The latest version of PEBL packaged for OSX is 0.12. It is compiled as an application bundle with both 32-bit and 64-bit architectures available. We do not support PPC architecture.


The simplest way to run PEBL is through the launcher, but you can also use Applescript
to create your own sequences of experiments.

On OSX, PEBL can be run as a command-line tool, just as in linux.  Once installed, the application is located at /Applications/pebl.app/Contents/MacOS/pebl. 

\sect{How to stop running a program}

In order to improve performance, PEBL runs at the highest
priority possible on your computer.  This means that if it
gets stuck somewhere, you may have difficulty terminating the
process.  We have added an `abort program' shortcut key
combination that will immediately terminate the program and
report the location at which it became stuck in your code: \newline press \verb+<CTRL><SHIFT><ALT><\>+ simultaneously.
  

\sect{Command-line arguments}
\label{sec:2.5}
Some aspects of PEBL's display can be controlled via
command-line arguments.  Some of these are platform
specific, or their use depends on your exact hardware and
software. The following guide to command-line
arguments is adapted from the output produced by
invoking PEBL  with no arguments:\\
\\
Usage:  Invoke PEBL with the experiment script files (\texttt{.pbl})
and  command-line arguments.\\
\\
Examples:
\begin{verbatim}
pebl experiment.pbl -s sub1 --fullscreen --display 800x600
      --driver dga
pebl experiment.pbl --driver xf86  --language es
pebl experiment.pbl -v 33 -v 2 --fullscreen --display 640x480 
\end{verbatim}
 

\subsubsection{Command-Line Options}

\begin{description}

\item
\begin{verbatim}
-v VALUE1 -v VALUE2
\end{verbatim}
  Invokes script and passes \texttt{VALUE1} and \texttt{VALUE2} (or any text
  immediately following a \texttt{-v}) to a list in the argument of
  the \texttt{Start()}  function. \newline This is useful for passing in
  conditions, subject numbers, randomization cues, and other
  entities that are easier to control from outside  the
  script.  Variables appear as strings, so numeric values
  need to be converted to be used as numbers.

\item
\begin{verbatim}
 -s VALUE
 -S VALUE
\end{verbatim}
  Binds \texttt{VALUE} to the global variable gSubNum, which is set by
  default to 0.

\item 
\begin{verbatim}
--driver <drivername>
\end{verbatim}
  Sets the video driver, when there is more than one.  In Linux SDL,
  options \texttt{xf86}, \texttt{dga}, \texttt{svgalib} (from
  console), itcan also be controlled via environment variables.  In
  fact, for SDL versions of PEBL simply set the
  \texttt{SDL\_VIDEO\_DRIVER} environment variable to the passed-in
  argument, without doing any checking, and without checking or
  returning it to its original state.

\item
\begin{verbatim}
--display  <widthxheight>
\end{verbatim}
  Controls the screen width and height (in pixels). Defaults
  to the current resolution of the screen.  Unlike older versions of PEBL, 
after 0.12 any legal combination of width and height should work.

The screensize a PEBL script runs at depends on a number of things. If no --display size is given
 (e.g., when 'default' is chosen in the launcher), PEBL will try to determine the current screen size 
and use that, for both fullscreen and windowed mode.  Otherwise, it will try to use the specified value.

However, these values are only a request.  When the script starts, it sets the values
of the global variables \texttt{gVideoWidth} and \texttt{gVideoHeight} based on either the 
specified values or the current screen size.  These values can be changed in the script before the \texttt{MakeWindow} function is called,
so that a script can require a particular screen size. 
Then, the window will be created with those dimensions, 
overriding any command-line parameters.   For greatest flexibility, it is recommended that you do not
hard-code screen size but rather make your test adapt to a large number of screen sizes.

Finally, if a screen size is selected that the video card cannot
support (i.e., in fullscreen mode), \texttt{gVideWidth} and \texttt{gVideoHeight} will
be set to the legal screen size closest to the one you requested.  PEBL should never crash because 
you have specified the wrong screen size, but it should rather use one it can support.  The values of 
\texttt{gVideoWidth} and \texttt{gVideoHeight} will be changed by MakeWindow to whatever screen size it actually uses.

\item
\begin{verbatim}
--depth
\end{verbatim}
  Controls the pixel depth, which also  depends on your video card.
  Currently, depths of 2,8,15,16,24, and 32 are allowed on the
  command-line.  There is no guarantee that you will get the
  specified bit depth, and bit depths such as 2 and 8 are
  likely never useful.  Changing depths can, for
  some drivers and video cards, enable better performance or
  possibly better video sychrony.  Defaults to 32.

\item
\begin{verbatim}
--language
\end{verbatim}
  Allows user to specify a language code that can get tested for
  within a script to select proper translation.  It sets a global
  variable gLanguage, and is ``en'' by default.

\item
\verb+--windowed+ or \verb+--fullscreen+
Controls whether the script will run in a window or fullscreen.  
The screen resolution a PEBL script runs at depends on a number of things.  See the \verb+--display+ option above for more details.


\end{description}



%%% Local Variables: 
%%% mode: latex
%%% TeX-master: "main"
%%% End: 
