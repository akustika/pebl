\chapter{About}

PEBL (Psychology Experiment Building Language) is a cross-platform,
open-source programming language and execution environment for
constructing programs to conduct a wide range of archetypal psychology
experiments. It is entirely free of charge, and may be modified to
suit your needs as long as you follow the terms of the GPL, under
which the source code is licensed. PEBL is written primarily in
\texttt{C++}, but requires a few other tools (\texttt{flex},
\texttt{yacc}) and libraries (\texttt{SDL}, \texttt{SDL\_image},
\texttt{SDL\_gfx}, and \texttt{SDL\_ttf}) to use. It currently
compiles and runs on Linux (using \texttt{g++}), Mac OSX (also using
\texttt{g++}), and Microsoft Windows (using \texttt{Dev-cpp} and
\texttt{mingw}) platforms using free tools. It has been developed
primarily by Shane T.~Mueller,
Ph.D. (\href{mailto:smueller@obereed.net}{smueller@obereed.net}). This
document was prepared with editorial and formatting help from Gulab
Parab and Samuele Carcagno. Contributions are welcome and encouraged.

\chapter{Usage}

\section{How to Compile}

Currently, there is no automated compile procedure.  PEBL requires the
\texttt{SDL}, \texttt{SDL-image}, \texttt{SDL-gfx} and
\texttt{SDL-ttf} libraries and development headers.  It also uses
\texttt{flex} and \texttt{bison}, but you can compile without these
tools. PEBL compiles on both Linux and Windows using the free
\texttt{gcc} compiler. Note that \texttt{SDL-image} may require
\texttt{jpeg}, \texttt{png}, and a \texttt{zlib} compression library,
while \texttt{SDL-ttf} requires \texttt{truetype 2.0}.

\subsection{Linux}

PEBL should compile by typing `\texttt{make}' in its base directory
once all requisite tools are installed and the source distribution is
uncompressed. Currently, PEBL does not use autotools, so its make
system is rather brittle. Assistance is welcome.

On Linux, compiling will fail if you don't have an \texttt{/obj}
directory and all the appropriate subdirectories (that mirror the main
tree.)  These will not exist if you check out from CVS.

\subsection{Microsoft Windows}

On Microsoft Windows, PEBL is designed to be compiled using the Free
ide \texttt{dev-c++} available at
\href{http://www.bloodshed.net/dev/devcpp.html}{http://www.bloodshed.net/dev/devcpp.html}
and instructions for installing \texttt{dev-c++}, \texttt{sdl}, and
the \texttt{minGW} system can be found at:\\

\href{http://www.libsdl.org/pipermail/sdl/2002-June/046382.html}{http://www.libsdl.org/pipermail/sdl/2002-June/046382.html}\\

and elsewhere on the net.  Email the PEBL list for more details.

\subsection{Mac OSX}

Currently, only Mac OSX 10.4 (TIGER) has been tested. PEBL is compiled
and runs from the command-line, and resides in
\texttt{/usr/local/share/pebl} and \texttt{/usr/local/bin}.  To
compile, download the source and compile \texttt{libSDL},
\texttt{libSDL\_image}, and \texttt{libSDL\_ttf}.
\texttt{libSDL\_ttf} requires freetype, and \texttt{libSDL\_image}
requires \texttt{libpng}, \texttt{libjpeg}, and \texttt{zlib}.

\section{How to Install}

Currently, binary distributions of PEBL are available with automatic
package installers for Windows and OSX.  No binary distribution is
available on Linux.

\subsection{Linux}

On Linux, you will probably have to install from source.
There is an `\texttt{install}' option in the Makefile that, if invoked with
`\texttt{make install}', will copy \texttt{bin/pebl} to \texttt{/usr/local/bin/pebl}, and the
\texttt{media/} directories to \texttt{/usr/local/share/pebl/media}.  You must be root
to do this, and you can just as easily do it by hand.

\subsection{Microsoft Windows}

In Microsoft Windows, we provide an installer package that contains
all necessary executable binary files and \texttt{.dlls}. This
installer places PEBL in \texttt{c:\char92 Program Files\char92PEBL},
and creates a directory \texttt{pebl-exp} in \texttt{My Documents}
with a shortcut that allows PEBL to be launched and programs that
reside there to be run.

\subsection{Macintosh OSX}

For the MAC, we provide a \texttt{.pkg} installer that installs all the
necessary files and libraries in \texttt{/usr/local/share/pebl}.  PEBL can be
run from the command line by invoking \texttt{/usr/local/bin/pebl}.


\section{How to Run a PEBL Program}


\subsection{Linux}

If you have installed PEBL into \texttt{/usr/local/bin}, you will be able to
invoke PEBL by typing `\texttt{pebl}' at a command line.  PEBL requires you to
specify one or more source files that it will compile and run, e.g., the
command:
\begin{verbatim}
  > pebl stroop.pbl library.pbl
\end{verbatim}
will load the experiment described in \texttt{stroop.pbl}, and will load the
supplementary library functions in \texttt{library.pbl}.

Additionally, PEBL can take the \texttt{-v} or \texttt{-V} command-line parameter, which allows you to pass values into the script.  This is useful for
entering subject numbers and condition types using an outside program
like a bash script (possibly one that invokes dialog or zenity). A sample zenity script that asks for subject number and then runs a sample experiment which uses that input resides in the \texttt{bin} directory. The script can be edited to use fullscreen mode or change the display dimensions, for example. See Section~\ref{sec:2.5}: Command-Line Arguments.

You can also specify directories without a filename on the command-line (as long as they end with `\texttt{/}').  Doing so will add that directory to the search path when files are opened.

\subsection{ Microsoft Windows}

PEBL can be launched from the command line in Windows by going to the
\texttt{pebl\char92bin} directory and typing `\texttt{pebl.exe}'.
PEBL requires you to specify one or more source files that it will
compile and run.  For example, the command
\begin{verbatim}
  > pebl stroop.pbl library.pbl
\end{verbatim}
loads the experiment described in \texttt{stroop.pbl}, and loads supplementary library functions in \texttt{library.pbl}.

Additionally, PEBL can take the \texttt{-v} or \texttt{-V}
command-line parameter, which allows you to pass values in to the
script.  This is useful for entering condition types using an outside
program like a batch file. the \texttt{-s} and \texttt{-S} allow one
to specify a subject code, which gets bound to the gSubNum variable.
If no value is specified, gSubNum is initialized to 0.  You can also
specify directories without a file (as long as they end with
`\texttt{\char92}').  Doing so will add that directory to the search
path when files are opened. See Section \ref{sec:2.5}: Command-Line
Arguments.


Launching programs from the command-line on Windows is cumbersome.
One easy way to launch PEBL on Windows is to create a shortcut to the
executable file and then edit the properties so that the shortcut
launches PEBL with the proper script and command-line parameters.
Another way is to write and launch a batch file, which is especially
useful if you wish to enter configuration data before loading the
script.

\subsubsection{ Win32 Launcher}
PEBL comes with a launcher program that launches PEBL scripts in
Microsoft Windows (tm).  It will allows you to specify variables to
pass into PEBL on execution, select multiple source files to load, and
configure with a text file.

The launcher is written in Visual Basic, and so you might need some
\texttt{.dll} files in order for it to run.

When the launcher is run, it first looks for a file called
\texttt{pebl-init.txt}. This file should have the following format:
\begin{verbatim}
---------------Beginning of file----------------------
"Quoted_Path_To_PEBL_Executable"
"Quoted_Path_To_Directory_To_Load_Files_From"
"File1.pbl" "File2.pbl"
1st Variable Label|Initial_Value
2nd Variable Label|Initial_Value
3rd Variable Label|Initial_Value
.
.
.
------------------End of file-------------------------
\end{verbatim}
My actual pebl-init looks like this:
\begin{verbatim}
-------------
"c:\Documents and Settings\smueller\My Documents\pebl\bin\pebl.exe"
"c:\Documents and Settings\smueller\My Documents\pebl\demo\"
"hello.pbl"
Subject Number|1
Condition|fast
-------------
\end{verbatim}


The Launcher will select the specified files in the listed directory
(this can be changed by selecting other files).  After the third line,
every pair separated with a `\texttt{|}' will appear as a text-entry
box with the pre-specified default value. This can be used to specify
subject numbers, conditions, and such, which are then fed into the
PEBL script.

As of version 0.06, the launcher is improved so that it will open the
\verb+stdout.txt+ and \verb+stderr.txt+ files after a script has been run and
display them in a tabbed interface at the bottom of the launcher
window.

\subsection{ Macintosh OSX}

The latest version of PEBL packaged for OSX is 0.07.  Until hardware/developers are available, newer versions for OSX will need to be compiled on your own.

Installing \texttt{pebl.pkg} places PEBL in
\texttt{/usr/local/share/pebl} and \texttt{/usr/local/bin}.
Currently, PEBL must be run from the command-line (there is no
graphical front-end).  Open a terminal (in the applications folder)
and type at the \texttt{\$} prompt:
\begin{verbatim}
$ /usr/local/bin/pebl
\end{verbatim}
To execute, type:
\begin{verbatim}
$  /usr/local/bin/pebl Documents/test.pbl
\end{verbatim}
To truncate, add \texttt{/usr/local/bin to the path}:
\begin{verbatim}
export PATH=$PATH:/usr/local/bin
\end{verbatim}
Then you can run:
\begin{verbatim}
$  pebl Documents/test.pbl
\end{verbatim}

On OSX, there is no such thing as double-buffering.  However, under
fullscreen mode, drawing can be synced to the vertical refresh. But as
a caveat, this has not yet been implemented.

\section{How to stop running a program}

In order to improve performance, PEBL runs at the highest
priority possible on your computer.  This means that if it
gets stuck somewhere, you may have difficulty terminating the
process.  We have added an `abort program' shortcut key
combination that will immediately terminate the program and
report the location at which it became stuck in your code: \newline press \verb+<CTRL><SHIFT><ALT><\>+ simultaneously.
  

\section{Command-line arguments}
\label{sec:2.5}
Some aspects of PEBL's display can be controlled via
command-line arguments.  Some of these are platform
specific, or their use depends on your exact hardware and
software. The following guide to command-line
arguments is adapted from the output produced by
invoking PEBL  with no arguments:\\
\\
Usage:  Invoke PEBL with the experiment script files (\texttt{.pbl})
and  command-line arguments.\\
\\
Examples:
\begin{verbatim}
pebl experiment.pbl -s sub1 --fullscreen --display 800x600 --driver dga
pebl experiment.pbl --driver xf86
pebl experiment.pbl -v 33 -v 2 --fullscreen --display 640x480 
\end{verbatim}
 

\subsubsection{Command-Line Options}

\begin{description}

\item
\begin{verbatim}
-v VALUE1 -v VALUE2
\end{verbatim}
  Invokes script and passes \texttt{VALUE1} and \texttt{VALUE2} (or any text
  immediately following a \texttt{-v}) to a list in the argument of
  the \texttt{Start()}  function. \newline This is useful for passing in
  conditions, subject numbers, randomization cues, and other
  entities that are easier to control from outside  the
  script.  Variables appear as strings, so numeric values
  need to be converted to be used as numbers.

\item
\begin{verbatim}
 -s VALUE
 -S VALUE
\end{verbatim}
  Binds \texttt{VALUE} to the global variable gSubNum, which is set by
  default to 0.

\item 
\begin{verbatim}
--driver <drivername>
\end{verbatim}
  Sets the video driver, when there is more than one.  In Linux SDL,
  options \texttt{xf86}, \texttt{dga}, \texttt{svgalib} (from
  console), itcan also be controlled via environment variables.  In
  fact, for SDL versions of PEBL simply set the
  \texttt{SDL\_VIDEO\_DRIVER} environment variable to the passed-in
  argument, without doing any checking, and without checking or
  returning it to its original state.

\item
\begin{verbatim}
--display  <widthxheight>
\end{verbatim}
  Controls the screen width and height (in pixels). Defaults
  to \texttt{640x480}.
  Currently, only the following screens are supported:
\begin{description}
\item
\begin{verbatim}
512x384
\end{verbatim}
\item
\begin{verbatim}
640x480
\end{verbatim}
\item
\begin{verbatim}
800x600
\end{verbatim}
\item
\begin{verbatim}
960x720
\end{verbatim}
\item
\begin{verbatim}
1024x768
\end{verbatim}
\item
\begin{verbatim}
1152x864
\end{verbatim}
\item
\begin{verbatim}
1280x1024
\end{verbatim}
\end{description}

Note: the way this is invoked may be changed in the
future.  Your video display may not support the
command-line argument.  If it does not, PEBL should exit
and disply a useful error message; of course, it could
possibly damage your hardware. 


For the sake of convenience, the width, height, and bit depth can be
accessed from within a PEBL script using the global variables
\texttt{gVideoWidth}, \texttt{gVideoHeight}, and \texttt{gVideoDepth}.
If these values are set within a script before the function
\verb+MakeWindow()+ is called,  the window will be created with these values,
overriding any command-line parameters.
  
\item
\begin{verbatim}
--depth
\end{verbatim}
  Controls the pixel depth, which also  depends on your video card.
  Currently, depths of 2,8,15,16,24, and 32 are allowed on the
  command-line.  There is no guarantee that you will get the
  specified bit depth, and bit depths such as 2 and 8 are
  likely never useful.  Changing depths can, for
  some drivers and video cards, enable better performance or
  possibly better video sychrony.

\item
\verb+--windowed+ or \verb+--fullscreen+

Controls whether the script will run in a window or fullscreen.  
\end{description}



%%% Local Variables: 
%%% mode: latex
%%% TeX-master: "main"
%%% End: 
