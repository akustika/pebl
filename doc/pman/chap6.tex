\chapter{Detailed Function and Keyword Reference.}
\label{sec:six}

\setlength{\parindent}{0pt}

\newcommand{\rl}{\rule{\textwidth}{0.3mm}}

\newenvironment{desc}[1]
 {\begin{list}{}%
  {\renewcommand\makelabel[1]{{##1:}\hfil}%
   \settowidth\labelwidth{\makelabel{#1}}%
   \setlength\leftmargin{\labelwidth+\labelsep}}}%
 {\end{list}}




\rl
\section{Symbols}

\rl

\begin{desc}{Name/Symbol}
\item[Name/Symbol] \verb!+!

\item[Description] Adds two expressions together.  Also,
concatenates strings together.

\item[Usage]
\begin{verbatim}
<num1> + <num2>
<string1> + <string2>
<string1> + <num1>
\end{verbatim}
 Using other types of variables will cause errors.

\item[Example]
\begin{verbatim}
33 + 322                   --> 355
"Hello" + " " + "World"    --> "Hello World"
"Hello" + 33 + 322.5       --> "Hello355.5"
33 + 322.5 + "Hello"       --> "33322.5Hello"
\end{verbatim}

\item[See Also]     \verb!-!, \texttt{ToString()}

\end{desc}

\rl


\begin{desc}{Name/Symbol}

\item[Name/Symbol] \verb+-+

\item[Description]  Subtracts one expression from another

\item[Usage]        \verb!<num1> - <num2>!

\item[Example]     

\item[See Also]

\end{desc}

\rl     


\begin{desc}{Name/Symbol}

\item[Name/Symbol] \verb+/+ 

\item[Description]  Divides one expression by another

\item[Usage]        \verb+<expression> / <expression>+ 

\item[Example]
\begin{verbatim}
333 / 10    # == 33.3
\end{verbatim}

\item[See Also]

\end{desc} 

\rl

\begin{desc}{Name/Symbol}
   

\item[Name/Symbol] \verb+*+

\item[Description]        Multiplies two expressions together

\item[Usage]       \verb+<expression> * <expression>+

\item[Example]
\begin{verbatim}
32 * 2 # == 64
\end{verbatim}

\item[See Also]     

\end{desc}

\rl


\begin{desc}{Name/Symbol}

\item[Name/Symbol] \verb!^!

\item[Description]  Raises one expression to the power of  another expression

\item[Usage]       \verb!<expression> ^ <expression>!

\item[Example]
\begin{verbatim}
25 ^ 2  # == 625
\end{verbatim}

\item[See Also]    \texttt{Exp}, \texttt{NthRoot}

\end{desc}

\rl

\begin{desc}{Name/Symbol}

\item[Name/Symbol] \verb+;+ 

\item[Description]        Finishes a statement, can start new statement
                      on the same line (not needed at end of line)

\item[Usage]       

\item[Example]     

\item[See Also]

\end{desc}

\rl

\begin{desc}{Name/Symbol}     

\item[Name/Symbol] \verb!#!

\item[Description]   Comment indicator; anything until the next CR
	       following this character is ignored

\item[Usage]       

\item[Example]     

\item[See Also]

\end{desc} 


\rl
     
\begin{desc}{Name/Symbol}

\item[Name/Symbol] \verb!<-!                  

\item[Description]  The assignment operator.  Assigns a value to a variable\\
              N.B.: This two-character sequence takes the place of the
	      `\verb!=!' operator found in many programming languages.

\item[Usage]       

\item[Example]     

\item[See Also]  

\end{desc}   

\rl

\begin{desc}{Name/Symbol}

\item[Name/Symbol] \verb+( )+                  

\item[Description] Groups mathematical operations

\item[Usage]      \verb+(expression)+

\item[Example]
\begin{verbatim}
(3 + 22) * 4  # == 100
\end{verbatim}

\item[See Also]     

\end{desc}

\rl

\begin{desc}{Name/Symbol}

\item[Name/Symbol] \verb!{ }!                  

\item[Description] Groups a series of statements

\item[Usage]
\begin{verbatim}
{ statement1
  statement2
  statement3
}
\end{verbatim}
	     

\item[Example]     

\item[See Also]     
\end{desc}

\rl

\begin{desc}{Name/Symbol}

\item[Name/Symbol] \verb+[ ]+                 

\item[Description]  Creates a list. Closing \verb+]+ must be on
 	      same line as last element of list, even
	      for nested lists.

\item[Usage]       \verb+[<item1>, <item2>, ....]+
            

\item[Example]
\begin{verbatim}
[]                    #Creates an empty list
[1,2,3]               #Simple list
[[3,3,3],[2,2],0]     #creates a nested list structure
\end{verbatim}


\item[See Also]     \texttt{List()}
\end{desc}

\rl

\begin{desc}{Name/Symbol}

\item[Name/Symbol] 	\verb+<+ 

\item[Description] 	Less than.  Used to compare two numeric quantities.

\item[Usage]
\begin{verbatim}
3 < 5
3 < value
\end{verbatim}
             
\item[Example]
\begin{verbatim}
if(j < 33)
{
  Print ("j is less than 33.")
}
\end{verbatim}

See Also:     	\verb+>+, \verb+>=+, \verb+<=+, \verb+==+, \verb+~=+, \verb+!=+, \verb+<>+

\end{desc}

\rl



\begin{desc}{Name/Symbol}

\item[Name/Symbol] 	\verb!>!                    

\item[Description] 	Greater than. Used to compare two numeric quantities.

\item[Usage]
\begin{verbatim}
5 > 3
5 > value
\end{verbatim}

\item[Example]
\begin{verbatim}
if(j > 55)
{
 Print ("j is greater than 55.")
}
\end{verbatim}

\item[See Also]     	\verb+<+, \verb+>=+, \verb+<=+, \verb+==+, \verb+~=+, \verb+!=+, \verb+<>+
\end{desc}

\rl


\begin{desc}{Name/Symbol}

\item[Name/Symbol] 	\verb+<=+                   

\item[Description] 	Less than or equal to.

\item[Usage]
\begin{verbatim}
3<=5  
3<=value
\end{verbatim}

\item[Example]
\begin{verbatim}
if(j <= 33)
{
 Print ("j is less than or equal to 33.")
}
\end{verbatim}
	
\item[See Also]     	\verb+<, >, >=, ==, ~=, !=, <>+

\end{desc}

\rl

\begin{desc}{Name/Symbol}

\item[Name/Symbol] 	\verb+>=+                   

\item[Description] 	Greater than or equal to.

\item[Usage]
\begin{verbatim}
5>=3  
5>=value
\end{verbatim}

\item[Example]
\begin{verbatim}
if(j >= 55)
{
 Print ("j is greater than or equal to 55.")
}
\end{verbatim}

\item[See Also]     	\verb+<,+ \verb+>+, \verb+<=+, \verb+==+, \verb+~=+, \verb+!=+, \verb+<>+
\end{desc}

\rl

\begin{desc}{Name/Symbol}

\item[Name/Symbol] 	\verb+==+                   

\item[Description] 	Equal to.

\item[Usage]       	\verb+4 == 4+
		

\item[Example]
\begin{verbatim}
2 + 2 == 4
\end{verbatim}

\item[See Also]     	\verb+<,+ \verb+>+, \verb+>=+, \verb+<=+, \verb+~=+, \verb+!=+, \verb+<>+
\end{desc}

\rl

\begin{desc}{Name/Symbol}
\item[Name/Symbol]  	\verb+<>+, \verb+!=+, \verb+~=+

\item[Description]  	Not equal to.

\item[Usage]		

\item[Example]	

\item[See Also]     	\verb+<+, \verb+>+, \verb+>=+, \verb+<=+, \verb+==+

\end{desc}

\rl 

\section{A} 
\rl

\begin{desc}{Name/Symbol}

\item[Name/Symbol] 	\verb+Abs()+

\item[Description]   	Returns the absolute value of the number.

\item[Usage]
\begin{verbatim}
Abs(<num>)
\end{verbatim}        

\item[Example]
\begin{verbatim}
Abs(-300)  	# ==300
Abs(23)    	# ==23
\end{verbatim}

\item[See Also]     	\verb+Round()+, \verb+Floor()+, \verb+AbsFloor()+, \verb+Sign()+, \verb+Ceiling()+
\end{desc}

\rl


\begin{desc}{Name/Symbol}

\item[Name/Symbol]  	\verb+AbsFloor()+

\item[Description]  	Rounds \verb+<num>+ toward 0 to an integer.

\item[Usage]       	
\begin{verbatim}
AbsFloor(<num>)
\end{verbatim}

\item[Example]
\begin{verbatim}
AbsFloor(-332.7)   	# == -332
AbsFloor(32.88)    	# == 32
\end{verbatim}

\item[See Also]     	\verb+Round()+, \verb+Floor()+, \verb+Abs()+, \verb+Sign()+, \verb+Ceiling()+
\end{desc}

\rl


\begin{desc}{Name/Symbol}

\item[Name/Symbol] 	\verb+ACos()+ 

\item[Description]  	Inverse cosine of \verb+<num>+, in degrees.

\item[Usage]
\begin{verbatim}
ACos(<num>)
\end{verbatim}

\item[Example]	

\item[See Also]    	\verb+Cos()+, \verb+Sin()+, \verb+Tan()+, \verb+ATan()+, \verb+ATan()+ 

\end{desc}

\rl
 


\begin{desc}{Name/Symbol}

\item[Name/Symbol]  	\verb+AddObject()+

\item[Description] 	Adds a widget to a parent window.

\item[Usage]		

\item[Example]	

\item[See Also]    	\verb+RemoveObject()+
\end{desc}

\rl



\begin{desc}{Name/Symbol}
\item[Name/Symbol]  	\verb+and+
  
\item[Description]  	Logical and operator.

\item[Usage]       	
\begin{verbatim}
<expression> and <expression>
\end{verbatim}

\item[Example]	

\item[See Also]     	\verb+or+, \verb+not+

\end{desc}

\rl


\begin{desc}{Name/Symbol}

\item[Name/Symbol]  	\verb+Append+
  
\item[Description]  	Appends an item to a list.  Useful for constructing lists in conjunction with the loop statement.

\item[Usage] 
\begin{verbatim}
Append(<list>, <item>)
\end{verbatim}

\item[Example]
\begin{verbatim}
list <- Sequence(1,5,1)
double  <- []
loop(i, list)
{
 double <- Append(double, [i,i])
}
Print(double)
# Produces [[1,1],[2,2],[3,3],[4,4],[5,5]]
\end{verbatim}

\item[See Also]     	\verb+List()+, \verb+[ ]+, \verb+Merge()+
\end{desc}

\rl



\begin{desc}{Name/Symbol}

\item[Name/Symbol]  	\verb+ASin()+ 

\item[Description]  	Inverse Sine of \verb+<num>+, in degrees.

\item[Usage]
\begin{verbatim}
ASin(<num>)
\end{verbatim}

\item[Example]	

\item[See Also]    	 \verb+Cos()+, \verb+Sin()+, \verb+Tan()+, \verb+ATan()+, \verb+ACos()+, \verb+ATan()+ 
\end{desc}

\rl



\begin{desc}{Name/Symbol}

\item[Name/Symbol]  	\verb+ATan+ 

\item[Description]  	Inverse Tan of \verb+<num>+, in degrees.

\item[Usage]		

\item[Example]	

\item[See Also]    	\verb+Cos()+, \verb+Sin()+, \verb+Tan()+, \verb+ATan()+, \verb+ACos()+, \verb+ATan()+ 
\end{desc}

\rl

\section{B}
\rl

\begin{desc}{Name/Symbol}
\item[Name/Symbol]  	\verb+Bezier+ 

\item[Description] Creates a smoothed line through the  points
specified by \verb+<xpoints>+, \verb+<ypoints>+. The lists \verb+<xpoints>+ and
\verb+<ypoints>+ are adjusted by  \verb+<x>+ and \verb+<y>+, so they
should be relative to 0, not the location you want the points to be at.

Like other drawn objects, the bezier must then be added to the window
to appear. <steps> denotes how smooth the approximation will be.

\item[Usage]		
\begin{verbatim}
  Bezier(<x>,<y>,<xpoints>,<ypoints>,<steps>,<color>)
\end{verbatim}

\item[Example]	
\begin{verbatim}
  win <- MakeWindow()
   #This makes a T
   xpoints <- [-10,10,10,20,20,-20,-20,-10]
   ypoints <- [-20,-20,40,40,50,50,40,40]
  p1 <-    Bezier(100,100,xpoints, ypoints,5, MakeColor("black"))
  AddObject(p1,win)
  Draw()
\end{verbatim}

\item[See Also]   
\verb+BlockE()+, \verb+Polygon()+, \verb+MakeStarPoints()+,
\verb+MakeNGonPoints()+
\end{desc}

\rl


\begin{desc}{Name/Symbol}

\item[Name/Symbol]  	\verb+BlockE+ 

\item[Description] Creates a polygon in the shape of a
 block E, pointing in one of four directions.
Arguments include position in window.
\begin{itemize}
\item \verb+<x>+ and \verb+<y>+ is the position of the center
\item \verb+<h>+ and \verb+<w>+ or the size of the E in pixels
\item \verb+<thickness>+ thickness of the E
\item \verb+<direction>+ specifies which way the E points:  1=right,
  2=down, 3=left, 4=up.
\item \verb+<color>+ is a color object (not just the name)
\end{itemize}

Like other drawn objects, the Block E must then be added to the window
to appear.

\item[Usage]		
\begin{verbatim}
 BlockE(x,y,h,w,thickness,direction,color)
\end{verbatim}

\item[Example]	
\begin{verbatim}
  win <- MakeWindow()
  e1 <- BlockE(100,100,40,80,10,1,MakeColor("black"))
  AddObject(e1,win)
  Draw()
\end{verbatim}

\item[See Also]   
\verb+Plus()+, \verb+Polygon()+, \verb+MakeStarPoints()+,
\verb+MakeNGonPoints()+
\end{desc}

\rl

\begin{desc}{Name/Symbol}
\item[Name/Symbol]  	\verb+break+

\item[Description]  	Breaks out of a loop immediately.

\item[Usage]        	break

\item[Example]
\begin{verbatim}
loop(i ,[1,3,5,9,2,7])
{
 Print(i)
 if(i == 3) 
        {
         break
        }
}
\end{verbatim}

\item[See Also]   	\verb+loop+, \verb+return+
\end{desc}

\rl


\section{C}
\rl


\begin{desc}{Name/Symbol}
\item[Name/Symbol]  	\verb+Ceiling()+

\item[Description] 	Rounds \verb+<num>+ up to the next integer.

\item[Usage]
\begin{verbatim}
Ceiling(<num>)
\end{verbatim}

\item[Example] 
\begin{verbatim}
Ceiling(33.23)  	# == 34
Ceiling(-33.02) 	# == -33
\end{verbatim}

\item[See Also]     	\verb+Round()+, \verb+Floor()+, \verb+AbsFloor()+, \verb+Ceiling()+
\end{desc}

\rl

\begin{desc}{Name/Symbol}
\item[Name/Symbol]  	\verb+ChooseN()+

\item[Description] Samples \verb+<number>+ items from list, returning
  a list in the original order. Items are sampled without replacement, so
  once an item is chosen it will not be chosen again. If
  \verb+<number>+ is larger than the length of the list, the entire
  list is returned in order.  It differs from \verb+SampleN+ in that
  \verb+ChooseN+ returns items in the order they appeared in the
  originial list, but \verb+SampleN+ is shuffled. 

\item[Usage]       	
\begin{verbatim}
ChooseN(<list>, <n>)
\end{verbatim}

\item[Example]   	
\begin{verbatim}
ChooseN([1,1,1,2,2], 5)     # Returns 5 numbers
ChooseN([1,2,3,4,5,6,7], 3) # Returns 3 numbers from 1 and 7
\end{verbatim}

\item[See Also]    	\verb+SampleN()+, \verb+SampleNWithReplacement()+, \verb+Subset()+
\end{desc}

\rl

\begin{desc}{Name/Symbol}
\item[Name/Symbol]	\verb+Circle()+

\item[Description] Creates a circle for graphing at x,y with radius r.
  Circles must be added to a parent widget before it can be drawn; it
  may be added to widgets other than a base window. The properties of
  circles may be changed by accessing their properties directly,
  including the FILLED property which makes the object an outline
  versus a filled shape.


\item[Usage]
\begin{verbatim}
Circle(<x>, <y>, <r>,<color>)
\end{verbatim}

\item[Example]	
\begin{verbatim}
  
  c <- Circle(30,30,20, MakeColor(green))
  AddObject(c, win)
  Draw()

\end{verbatim}
\item[See Also]	\verb+Square()+, \verb+Ellipse()+, \verb+Rectangle()+, \verb+Line()+
\end{desc}

\rl




\begin{desc}{Name/Symbol}
\item[Name/Symbol]	\verb+CloseNetworkConnection()+

\item[Description]	Closes network connection

\item[Usage]
\begin{verbatim}
   CloseNetwork(<network>)
\end{verbatim}

\item[Example]	

\begin{verbatim}
  net <- WaitForNetworkConnection("localhost",1234)
  SendData(net,"Watson, come here. I need you.")
  CloseNetworkConnection(net)
\end{verbatim}
Also see nim.pbl for example of two-way network connection.
\item[See Also]
  \verb+ConnectToIP+, \verb+ConnectToHost+, \verb+GetData+, \verb+WaitForNetworkConnection+,
   \verb+SendData+, \verb+ConvertIPString+
\end{desc}

\rl



\begin{desc}{Name/Symbol}
\item[Name/Symbol]	\verb+ConnectToHost()+

\item[Description]	Connects to a host computer waiting for a
  connection on <port>, returning a network object that can be used to
  communicate.  Host is a text hostname, like \verb+"myname.indiana.edu"+, or
  use \verb+"localhost"+ to specify your current computer.

\item[Usage]
\begin{verbatim}
   ConnectToHost(<hostname>,<port>)
\end{verbatim}

\item[Example]	

  See nim.pbl for example of two-way network connection.
\begin{verbatim}
  net <- ConnectToHost("localhost",1234)
  dat <- GetData(net,20)
  Print(dat)
  CloseNetworkConnection(net)
\end{verbatim}

\item[See Also]
  \verb+ConnectToIP+, \verb+GetData+, 
\verb+WaitForNetworkConnection+, \verb+SendData+, \verb+ConvertIPString+, \verb+CloseNetworkConnection+
\end{desc}

\rl

\begin{desc}{Name/Symbol}
\item[Name/Symbol]	\verb+ConnectToIP()+

\item[Description]	Connects to a host computer waiting for a
  connection on \verb+<port>+, returning a network object that can be used to
  communicate.  \verb+<ip>+ is a numeric ip address, which must be
  created with the \verb+ConvertIPString(ip)+ function. 

\item[Usage]
\begin{verbatim}
   ConnectToIP(<ip>,<port>)
\end{verbatim}

\item[Example]	

  See nim.pbl for example of two-way network connection.
\begin{verbatim}
  ip <- ConvertIPString("192.168.0.1")
  net <- ConnectToHost(ip,1234)
  dat <- GetData(net,20)
  Print(dat)
  CloseNetworkConnection(net)
\end{verbatim}

\item[See Also]
  \verb+ConnectToHost+, \verb+GetData+, \verb+WaitForNetworkConnection+,
   \verb+SendData+, \verb+ConvertIPString+, \verb+CloseNetworkConnection+
\end{desc}

\rl



\begin{desc}{Name/Symbol}
\item[Name/Symbol] \verb+ConvertIPString()+

\item[Description]	Converts an IP address specified as a string into
  an integer that can be used by ConnectToIP.

\item[Usage]
\begin{verbatim}
   ConvertIPString(<ip-as-string>)
\end{verbatim}

\item[Example]	

  See nim.pbl for example of two-way network connection.
\begin{verbatim}
  ip <- ConvertIPString("192.168.0.1")
  net <- ConnectToHost(ip,1234)
  dat <- GetData(net,20)
  Print(dat)
  CloseNetworkConnection(net)
\end{verbatim}

\item[See Also]
  \verb+ConnectToHost+, \verb+ConnectToIP+, \verb+GetData+, \verb+WaitForNetworkConnection+,
   \verb+SendData+, \verb+ConvertIPString+, \verb+CloseNetworkConnection+
\end{desc}

\rl


\begin{desc}{Name/Symbol}
\item[Name/Symbol]  	\verb+Cos()+
			 
\item[Description] 	Cosine of \verb+<deg>+ degrees.

\item[Usage]		
\item[Example]	
\begin{verbatim}
  Cos(33.5)
  Cos(-32)
\end{verbatim}

\item[See Also]     	\verb+Sin()+, \verb+Tan()+, \verb+ATan()+, \verb+ACos()+, \verb+ATan()+
\end{desc}

\rl     


\begin{desc}{Name/Symbol}

\item[Name/Symbol] \verb+CR()+

\item[Description]  Produces \verb+<number>+ linefeeds which can be added to a
  string and printed or saved to a file.

\item[Usage]        \verb!CR(<number>)!

\item[Example]     
\begin{verbatim}
         Print("Number: "  Tab(1) + number  + CR(2))
         Print("We needed space before this line.")
\end{verbatim}
\item[See Also]
\verb+Format()+, \verb+Tab()+
\end{desc}


\rl



\begin{desc}{Name/Symbol}
\item[Name/Symbol]  	\verb+CrossFactorWithoutDuplicates()+

\item[Description] 	This function takes a single list, and returns a list of all 
			pairs, excluding the pairs that have two of the same item. 
			To achieve the same effect but include the duplicates, use: \\
			\verb+DesignFullCounterBalance(x,x)+.

\item[Usage]
\begin{verbatim}
CrossFactorWithoutDuplicates(<list>)
\end{verbatim}

\item[Example]
\begin{verbatim}
CrossFactorWithoutDuplicates([a,b,c]) 
# == [[a,b],[a,c],[b,a],[b,c],[c,a],[c,b]]
\end{verbatim}

\item[See Also] \verb+DesignFullCounterBalance()+, \verb+Repeat()+, \verb+DesignBalancedSampling()+,
 \verb+DesignGrecoLatinSquare()+,
  \verb+DesignLatinSquare()+,  \verb+RepeatList()+, 
  \verb+LatinSquare()+ \verb+Shuffle()+
\end{desc}

\rl
\begin{desc}{Name/Symbol}
\item[Name/Symbol]  	\verb+CumNormInv()+

\item[Description] 	This function takes a probability and returns the 
  corresponding z-score for the cumulative standard normal distribution.
  It uses an accurate numerical approximation from:\\
  \texttt{http://home.online.no/~pjacklam/notes/invnorm}
\item[Usage]
\begin{verbatim}
CumNormInv(<p>)
\end{verbatim}

\item[Example]
\begin{verbatim}

 Print(CumNormInv(0))    #= NA
  Print(CumNormInv(.01)) #= -2.32634
  Print(CumNormInv(.5))  #= 0
  Print(CumNormInv(.9))  #= 1.28
  Print(CumNormInv(1))   #= NA
\end{verbatim}

\item[See Also] \verb+NormalDensity()+, \verb+RandomNormal()+
\end{desc}


\rl
\section{D}
\rl


\begin{desc}{Name/Symbol}
\item[Name/Symbol]  	\verb+define+

\item[Description]  	Defines a user-specified function.

\item[Usage]
\begin{verbatim}
define functionname (parameters)
{
 statement1
 statement2
 statement3
       #Return statement is optional:
 return <value>
}
\end{verbatim}

\item[Example]    	See above.

\item[See Also]
\end{desc}   	

\rl

\begin{desc}{Name/Symbol}
\item[Name/Symbol]  	\verb+DegToRad()+

\item[Description]  	Converts degrees to radians.

\item[Usage]
\begin{verbatim}
DegToRad(<deg>)
\end{verbatim}

\item[Example]     	
\begin{verbatim}
DegToRad(180) # == 3.14159...
\end{verbatim}

\item[See Also]    	\verb+Cos()+, \verb+Sin()+, \verb+Tan()+, \verb+ATan()+, \verb+ACos()+, \verb+ATan()+ 
\end{desc}

\rl



\begin{desc}{Name/Symbol}
\item[Name/Symbol]  	\verb+DesignBalancedSampling()+

\item[Description] 	Samples elements "roughly" equally.
  		This function returns a list of repeated samples from
 		\verb+<treatment_list>+, such that each element in \verb+<treatment_list>+ 
		appears approximately equally.  Each element from 
		\verb+<treatment_list>+ is sampled once without replacement before 
		all elements are returned to the mix and sampling is repeated.  
		If there are no repeated items in \verb+<list>+, there will be no
 		consecutive repeats in the output.  The last repeat-sampling 
		will be truncated so that a \verb+<length>+-size list is returned.  
		If you don't want the repeated epochs this function provides, 
		Shuffle() the results.

\item[Usage]
\begin{verbatim}
DesignBalancedSampling(<list>, <length>)
\end{verbatim}

\item[Example]
\begin{verbatim}
DesignBalancedSampling([1,2,3,4,5],12)
## e.g., produces something like:
##    [5,3,1,4,2, 3,1,5,2,4, 3,1 ]
\end{verbatim}

\item[See Also]	\verb+CrossFactorWithoutDuplicates()+,
  \verb+Shuffle()+, \verb+DesignFullCounterBalance()+,
  		\verb+DesignGrecoLatinSquare()+, \verb+DesignLatinSquare()+, \verb+Repeat()+, 
		\verb+RepeatList()+,  \verb+LatinSquare()+

\end{desc}

\rl




\begin{desc}{Name/Symbol}

\item[Name/Symbol]	\verb+DesignFullCounterbalance()+

\item[Description]	This takes two lists as parameters, and returns a nested list 
		of lists that includes the full counterbalancing of both 
		parameter lists.  Use cautiously; this gets very large.

\item[Usage]
\begin{verbatim}
DesignFullCounterbalance(<lista>, <listb>)
\end{verbatim}

\item[Example]
\begin{verbatim}
a <- [1,2,3]
b <- [9,8,7]
DesignFullCounterbalance(a,b)	# == [[1,9],[1,8],[1,7],
				#     [2,9],[2,8],[2,7],
				#     [3,9],[3,8],[3,7]]
\end{verbatim}

\item[See Also] \verb+CrossFactorWithoutDuplicates()+,
  \verb+LatinSquare()+, \verb+Shuffle()+,
  \verb+DesignBalancedSampling()+, \verb+DesignGrecoLatinSquare()+, 
  \verb+DesignLatinSquare()+, \verb+Repeat()+, \verb+RepeatList()+,

\end{desc}

\rl




\begin{desc}{Name/Symbol}
\item[Name/Symbol]	\verb+DesignGrecoLatinSquare()+

\item[Description] This will return a list of lists formed by rotating
  through each element of the \verb+<treatment_list>+s, making a list
  containing all element of the list, according to a greco-latin
  square.  All lists must be of the same length.

\item[Usage]
\begin{verbatim}
DesignGrecoLatinSquare(<factor_list>, <treatment_list>, 
<treatment_list>)
\end{verbatim}

\item[Example]
\begin{verbatim}
x <- ["a","b","c"]
y <- ["p","q","r"]
z <- ["x","y","z"]
Print(DesignGrecoLatinSquare(x,y,z))
# produces:   	[[[a, p, x], [b, q, y], [c, r, z]], 
#               [[a, q, z], [b, r, x], [c, p, y]], 
#               [[a, r, y], [b, p, z], [c, q, x]]]
\end{verbatim}

\item[See Also] \verb+CrossFactorWithoutDuplicates()+, \verb+LatinSquare()+
  \verb+DesignFullCounterBalance()+, \verb+DesignBalancedSampling()+,
  \verb+DesignLatinSquare()+, \verb+Repeat()+, \verb+RepeatList()+,
  \verb+Shuffle()+
\end{desc}

\rl


\begin{desc}{Name/Symbol}
\item[Name/Symbol]	\verb+DesignLatinSquare()+

\item[Description] This returns return a list of lists formed by
  rotating through each element of \verb+<treatment_list>+, making a
  list containing all element of the list. Has no side effect on input
  lists.  

\item[Usage]
\begin{verbatim}
DesignLatinSquare(<treatment1_list>, <treatment2_list>)
\end{verbatim}

\item[Example]
\begin{verbatim}
order <- [1,2,3]
treatment <- ["A","B","C"]
design <- DesignLatinSquare(order,treatment)
# produces: [[[1, A], [2, B], [3, C]],
#            [[1, B], [2, C], [3, A]],
#            [[1, C], [2, A], [3, B]]]
\end{verbatim}

\item[See Also] \verb+CrossFactorWithoutDuplicates()+,
  \verb+DesignFullCounterBalance()+, \verb+DesignBalancedSampling()+,
  \verb+DesignGrecoLatinSquare()+, \verb+Repeat()+, \verb+LatinSquare()+
  \verb+RepeatList()+, \verb+Shuffle()+, \verb+Rotate()+
\end{desc}

\rl


\begin{desc}{Name/Symbol}
\item[Name/Symbol]	\verb+Div()+

\item[Description]  	Returns round(\verb+<num>/<mod>+)

\item[Usage]
\begin{verbatim}
Div(<num>, <mod>)
\end{verbatim}

\item[Example]	

\item[See Also]	\verb+Mod()+
\end{desc}

\rl




\begin{desc}{Name/Symbol}
\item[Name/Symbol]	\verb+Draw()+

\item[Description]	Redraws the screen or a specific widget.

\item[Usage]
\begin{verbatim}
Draw()
Draw(<object>)
\end{verbatim}

\item[Example]	

\item[See Also]	\verb+DrawFor()+, \verb+Show()+, \verb+Hide()+
\end{desc}

\rl



\begin{desc}{Name/Symbol}
\item[Name/Symbol]	\verb+DrawFor()+

\item[Description] Draws a screen or widget, returning after
  \verb+<cycles>+ refreshes. This function currently does not work as
  intended in the SDL implementation, because of a lack of control
  over the refresh blank.  It may work in the future.

\item[Usage]
\begin{verbatim}
DrawFor( <object>, <cycles>)
\end{verbatim}

\item[Example]	

\item[See Also]	\verb+Draw()+, \verb+Show()+, \verb+Hide()+
\end{desc}

\rl
\section{E}
\rl


\begin{desc}{Name/Symbol}
\item[Name/Symbol]	\verb+EasyLabel()+

\item[Description] Creates and adds to the window location a label
  at specified location. Uses standard vera font with grey background.
   (May in the future get background color from window).
  Easy-to-use replacement for the \verb+MakeFont+, \verb+MakeLabel+,
  \verb+AddObject+, \verb+Move+, steps you typically have to go through.

\item[Usage]
\begin{verbatim}
EasyLabel( <text>,<x>, <y>, <win>, <fontsize>)
\end{verbatim}

\item[Example]	
\begin{verbatim}
  win <- MakeWindow()
  lab <- EasyLabel("What?",200,100,win,12)
  Draw()
\end{verbatim}
\item[See Also]	
\verb+EasyTextBox()+, \verb+MakeLabel()+
\end{desc}


\rl

\begin{desc}{Name/Symbol}
\item[Name/Symbol]	\verb+EasyTextBox()+

\item[Description] Creates and adds to the window location a textbox
  at specified location. Uses standard vera font with white background.
  Easy-to-use replacement for the MakeFont,MakeTextBox,
  AddObject, Move, steps.

\item[Usage]
\begin{verbatim}
EasyTextBox( <text>,<x>, <y>, <win>, <fontsize>,<width>,<height>)
\end{verbatim}

\item[Example]	
\begin{verbatim}
  win <- MakeWindow()
  entry <- EasyTextBox("1 2 3 4 5",200,100,win,12,200,50)
  Draw()
\end{verbatim}
\item[See Also]	
\verb+EasyLabel()+, \verb+MakeTextBox()+
\end{desc}
\rl


\begin{desc}{Name/Symbol}
\item[Name/Symbol]	\verb+Ellipse()+
  
\item[Description]	Creates a ellipse for graphing at x,y with radii
  rx and ry. Ellipses are only currently definable oriented in
  horizontal/vertical directions.  Ellipses  must be added
  to a parent widget before it can be drawn; it may be added to
  widgets other than a base window.  The properties of ellipses may be
  changed by accessing their properties directly, including the FILLED
  property which makes the object an outline versus a filled shape.

\item[Usage]
\begin{verbatim}
Ellipse(<x>, <y>, <rx>, <ry>,<color>)
\end{verbatim}

\item[Example]	
\begin{verbatim}
  
  e <- Ellipse(30,30,20,10, MakeColor(green))
  AddObject(e, win)
  Draw()

\end{verbatim}
\item[See Also]	\verb+Square()+, \verb+Circle()+, \verb+Rectangle()+, \verb+Line()+
\end{desc}

\rl



\begin{desc}{Name/Symbol}
\item[Name/Symbol]	\verb+EndOfFile()+

\item[Description]	Returns true if at the end of a file.

\item[Usage]
\begin{verbatim}
EndOfFile(<filestream>)
\end{verbatim}

\item[Example]
\begin{verbatim}
while(not EndOfFile(fstream))
{
 Print(FileReadLine(fstream))
}
\end{verbatim}

\item[See Also]	
\end{desc}

\rl



\begin{desc}{Name/Symbol}
\item[Name/Symbol]	\verb+EndOfLine()+

\item[Description]	Returns true if at end of line.

\item[Usage]
\begin{verbatim}
EndOfLine(<filestream>)
\end{verbatim}

\item[Example]	

\item[See Also]	
\end{desc}

\rl




\begin{desc}{Name/Symbol}
\item[Name/Symbol]  	\verb+Exp()+

\item[Description]	$e$ to the power of \verb+<pow>+.

\item[Usage]
\begin{verbatim}
Exp(<pow>)
\end{verbatim}

\item[Example]
\begin{verbatim}
Exp(0) 		# == 1
Exp(3)		# == 20.0855
\end{verbatim}

\item[See Also]	\verb+Log()+
\end{desc}

\rl




\begin{desc}{Name/Symbol}
\item[Name/Symbol]	\verb+ExtractListItems()+

\item[Description]	Extracts items from a list, forming a new list. 
		The list \verb+<items>+ are the integers representing the
		indices that should be extracted.  
	     
\item[Usage]
\begin{verbatim}
ExtractListItems(<list>,<items>)
\end{verbatim}

\item[Example]
\begin{verbatim}
myList <- Sequence(101, 110, 1)
ExtractListItems(myList, [2,4,5,1,4])
# produces [102, 104, 105, 101, 104]
\end{verbatim}

\item[See Also]	\verb+Subset()+, \verb+SubList()+, \verb+SampleN()+
\end{desc}

\rl

\section{F}
\rl



\begin{desc}{Name/Symbol}
\item[Name/Symbol]	\verb+FileClose()+

\item[Description]	Closes a filestream  variable.  Be sure to 
		pass the variable name, not the filename.  

\item[Usage]
\begin{verbatim}
FileClose(<filestream>)
\end{verbatim}

\item[Example]
\begin{verbatim}
x <- FileOpenRead("file.txt")
# Do relevant stuff here.
FileClose(x)
\end{verbatim}

\item[See Also]	\verb+FileOpenAppend()+, \verb+FileOpenRead()+, \verb+FileOpenWrite()+

\end{desc}

\rl




\begin{desc}{Name/Symbol}
\item[Name/Symbol]	\verb+FileOpenAppend()+

\item[Description] Opens a filename, returning a stream that can be
  used for writing information.  Appends if the file already exists.

\item[Usage]
\begin{verbatim}
FileOpenAppend(<filename>)
\end{verbatim}

\item[Example]	

\item[See Also]	\verb+FileClose()+, \verb+FileOpenRead()+, \verb+FileOpenWrite()+
\end{desc}

\rl



\begin{desc}{Name/Symbol}
\item[Name/Symbol]	\verb+FileOpenRead()+

\item[Description]  	Opens a filename, returning  a stream to be used 
		for reading information.

\item[Usage]
\begin{verbatim}
FileOpenRead(<filename>)
\end{verbatim}

\item[Example]	

\item[See Also]	\verb+FileClose()+, \verb+FileOpenAppend()+, \verb+FileOpenWrite()+
\end{desc}

\rl


\begin{desc}{Name/Symbol}
\item[Name/Symbol]	\verb+FileOpenWrite()+

\item[Description] Opens a filename, returning a stream that can be
  used for writing information.  Overwrites if file already exists.

\item[Usage]
\begin{verbatim}
FileOpenWrite(<filename>)
\end{verbatim}

\item[Example]	

\item[See Also]	\verb+FileClose()+, \verb+FileOpenAppend()+, \verb+FileOpenRead()+
\end{desc}

\rl




\begin{desc}{Name/Symbol}
\item[Name/Symbol]	\verb+FilePrint()+

\item[Description]	Like \verb+Print+, but to a file.  Prints a string to a file, 
		with a carriage return at the end.  Returns a
  copy of the string it prints.
	
\item[Usage]
\begin{verbatim}
FilePrint(<filestream>, <value>)
\end{verbatim}

\item[Example]
\begin{verbatim}
FilePrint(fstream, "Another Line.")
\end{verbatim}

\item[See Also]	\verb+Print()+, \verb+FilePrint_()+
\end{desc}

\rl




\begin{desc}{Name/Symbol}
\item[Name/Symbol]	\verb+FilePrint_()+

\item[Description]	Like \verb+Print_+, but to a file.  Prints a
  string to a file,	without appending a newline character.  Returns a
  copy of the string it prints.
	
\item[Usage]
\begin{verbatim}
FilePrint_(<filestream>, <value>)
\end{verbatim}

\item[Example]
\begin{verbatim}
FilePrint_(fstream, "This line doesn't end.")
\end{verbatim}

\item[See Also]	\verb+Print_()+, \verb+FilePrint()+
\end{desc}

\rl



\begin{desc}{Name/Symbol}
\item[Name/Symbol]	\verb+FilePrintList()+

\item[Description]	Prints a list to a file, without the ','s or []
  characters. Puts a carriage return at the end.  Returns a string
  that was printed.  If a list contains other lists, the printing will
  wrap multiple lines and the internal lists will be printed as
  normal.  To avoid this, try FilePrintList(file,Flatten(list)).

\item[Usage]
\begin{verbatim}
FilePrintList(<filestream>, <list>)
\end{verbatim}

\item[Example]
\begin{verbatim}

FilePrintList(fstream, [1,2,3,4,5,5,5])
##
##  Produces:
##1 2 3 4 5 5 5
FilePrintList(fstream,[[1,2],[3,4],[5,6]])
#Produces:
# [1,2]
#,[3,4]
#,[5,6]

FilePrintList(fstream,Flatten([[1,2],[3,4],[5,6]]))
#Produces:
# 1 2 3 4 5 6


\end{verbatim}

\item[See Also]	\verb+Print()+, \verb+Print_()+, \verb+FilePrint()+, \verb+FilePrint_()+, \verb+PrintList()+,
\end{desc}

\rl




\begin{desc}{Name/Symbol}
\item[Name/Symbol]	\verb+FileReadCharacter()+

\item[Description]	Reads and returns a single character from a filestream.

\item[Usage]
\begin{verbatim}
FileReadCharacter(<filestream>)
\end{verbatim}

\item[Example]	

\item[See Also]	
\end{desc}

\rl




\begin{desc}{Name/Symbol}
\item[Name/Symbol]	\verb+FileReadLine()+

\item[Description]	Reads and returns a line from a file; all characters up
		until the next newline or the end of the file.

\item[Usage]
\begin{verbatim}
FileReadLine(<filestream>)
\end{verbatim}

\item[Example]	

\item[See Also]	
\end{desc}

\rl




\begin{desc}{Name/Symbol}
\item[Name/Symbol]  	\verb+FileReadList()+
 
\item[Description] Given a filename, will open it, read in all the
  items into a list (one item per line), and close the file afterward.
  Ignores blank lines or lines starting with \verb+#+.  Useful with a
  number of pre-defined data files stored in \verb+media/text/+.  See
  Section~\ref{sec:media}: Provided Media Files.

\item[Usage]
\begin{verbatim}
FileReadList(<filename>)
\end{verbatim}

\item[Example]
\begin{verbatim}
FileReadList("data.txt")
\end{verbatim}

\item[See Also]
\end{desc}

\rl


\begin{desc}{Name/Symbol}
\item[Name/Symbol]	\verb+FileReadTable()+

\item[Description]	Reads a table directly from a file. Data in file should
		separated by spaces.  Reads each line onto a sublist,
		with space-separated tokens as items in sublist.  Ignores
		blank lines or lines beginning with \verb+#+. Optionally,
		specify a token separator other than space.

\item[Usage]
\begin{verbatim}
FileReadTable(<filename>, <optional-separator>)
\end{verbatim}

\item[Example]
\begin{verbatim}
a <- FileReadTable("data.txt")
\end{verbatim}

\item[See Also]	\verb+FileReadList()+
\end{desc}

\rl




\begin{desc}{Name/Symbol}
\item[Name/Symbol]	\verb+FileReadText()+

\item[Description]	Returns all of the text from a file, ignoring any lines
		beginning with \verb+#+. Opens and closes the file transparently.

\item[Usage]
\begin{verbatim}
FileReadText(<filename>)
\end{verbatim}

\item[Example]
\begin{verbatim}
instructions <- FileReadText("instructions.txt")
\end{verbatim}

\item[See Also]	\verb+FileReadList()+, \verb+FileReadTable()+
\end{desc}

\rl



\begin{desc}{Name/Symbol}
\item[Name/Symbol]	\verb+FileReadWord()+

\item[Description]	Reads and returns  a `word' from a file; the next
		connected stream of characters not including a \verb+' '+
		or a newline. Will not read newline characters.

\item[Usage]
\begin{verbatim}
FileReadWord(<filestream>)
\end{verbatim}

\item[Example]	

\item[See Also]	\verb+FileReadLine()+, \verb+FileReadTable()+, \verb+FileReadList()+
\end{desc}

\rl




\begin{desc}{Name/Symbol}
\item[Name/Symbol]	\verb+FindInString()+

\item[Description]	Finds a token in a string, returning the position.

\item[Usage]
\begin{verbatim}
FindInString(<string>,<string>)
\end{verbatim}

\item[Example]
\begin{verbatim}
FindInString("about","bo") 	# == 2
\end{verbatim}

\item[See Also]	\verb+SplitString()+
\end{desc}

\rl




\begin{desc}{Name/Symbol}
\item[Name/Symbol]	\verb+First()+

\item[Description]	Returns the first item of a list.

\item[Usage]
\begin{verbatim}
First(<list>)
\end{verbatim}

\item[Example]
\begin{verbatim}
First([3,33,132])		# == 3
\end{verbatim}

\item[See Also]	\verb+Nth()+, \verb+Last()+
\end{desc}

\rl



\begin{desc}{Name/Symbol}
\item[Name/Symbol]	\verb+Flatten()+

\item[Description]
	Flattens nested list \verb+<list>+ to a single flat list.

\item[Usage]
\begin{verbatim}
Flatten(<list>)
\end{verbatim}

\item[Example]
\begin{verbatim}
Flatten([1,2,[3,4],[5,[6,7],8],[9]])	# == [1,2,3,4,5,6,7,8,9]
Flatten([1,2,[3,4],[5,[6,7],8],[9]])	# == [1,2,3,4,5,6,7,8,9]
\end{verbatim}

\item[See Also]	\verb+FlattenN()+, \verb+FoldList()+
\end{desc}

\rl




\begin{desc}{Name/Symbol}
\item[Name/Symbol]	\verb+FlattenN()+

\item[Description]	Flattens \verb+<n>+ levels of nested list \verb+<list>+. 

\item[Usage]
\begin{verbatim}
Flatten(<list>, <n>)
\end{verbatim}

\item[Example]
\begin{verbatim}
Flatten([1,2,[3,4],[5,[6,7],8],[9]],1) 
# == [1,2,3,4,5,[6,7],8,9]
\end{verbatim}

\item[See Also]	\verb+Flatten()+, \verb+FoldList()+
\end{desc}

\rl




\begin{desc}{Name/Symbol}
\item[Name/Symbol]	\verb+Floor()+

\item[Description]	Rounds \verb+<num>+ down to the next integer.

\item[Usage]
\begin{verbatim}
Floor(<num>)
\end{verbatim}

\item[Example]
\begin{verbatim}
Floor(33.23)	# == 33
Floor(3.999)  	# ==3
Floor(-32.23) 	# == -33
\end{verbatim}
 
\item[See Also]	\verb+AbsFloor()+, \verb+Round()+, \verb+Ceiling()+
\end{desc}

\rl


\begin{desc}{Name/Symbol}
\item[Name/Symbol]	\verb+FoldList()+

\item[Description]	Folds a list into equal-length sublists.

\item[Usage]
\begin{verbatim}
FoldList(<list>, <size>)
\end{verbatim}

\item[Example]
\begin{verbatim}
FoldList([1,2,3,4,5,6,7,8],2)	# == [[1,2],[3,4],[5,6],[7,8]]
\end{verbatim}
 
\item[See Also]	\verb+FlattenN()+, \verb+Flatten()+
\end{desc}

\rl



\begin{desc}{Name/Symbol}
\item[Name/Symbol]	\verb+Format()+            

\item[Description]	Formats the printing of values to ensure the
  proper spacing. It will either truncate or pad \verb+<value>+ with spaces
  so that it ends up exactly \verb+<length>+ characters long.  Character
  padding is at the end.
 
\item[Usage]
\begin{verbatim}
Format(<value>, <length>)
\end{verbatim}

\item[Example]	
\begin{verbatim}

  x <- 33.23425225
  y <- 23.3
  Print("["+Format(x,5)+"]")
  Print("["+Format(y,5)+"]")
  ## Output: 
  ## [33.23 ]
  ## [23.3  ]
\end{verbatim}
         

\item[See Also]	
\verb+CR()+ \verb+Tab()+
\end{desc}

\rl
\section{G}
\rl


\begin{desc}{Name/Symbol}
\item[Name/Symbol]	\verb+GetCursorPosition()+

\item[Description]	Returns an integer specifying where in a textbox the edit cursor is.  The value indicates which character it is on.

\item[Usage]
\begin{verbatim}
GetCursorPosition(<textbox>)
\end{verbatim}

\item[Example]	

\item[See Also]	\verb+SetCursorPosition()+, \verb+MakeTextBox()+, \verb+SetText()+
\end{desc}

\rl


\begin{desc}{Name/Symbol}
\item[Name/Symbol]	\verb+GetData()+

\item[Description]	Gets Data from network connection.  Example of
  usage in demo/nim.pbl.

\item[Usage]
\begin{verbatim}
val <- GetData(<network>,<size>)
\end{verbatim}

\item[Example]	

On 'server':
\begin{verbatim}
  net <- WaitForNetworkConnection("localhost",1234)
  SendData(net,"Watson, come here. I need you.")
  value <-  GetData(net,10)
  Print(value)

\end{verbatim}
On Client:
\begin{verbatim}
  net <- ConnectToHost("localhost",1234)
  value <-  GetData(net,20)
  Print(value)
##should print out "Watson, come here. I need you."
\end{verbatim}
\item[See Also]
  \verb+ConnectToIP+, \verb+ConnectToHost+, \verb+WaitForNetworkConnection+,
   \verb+SendData+, \verb+ConvertIPString+, \verb+CloseNetworkConnection+
\end{desc}

\rl


\begin{desc}{Name/Symbol}
\item[Name/Symbol]	\verb+GetInput()+

\item[Description]	Allows user to type input into a textbox.

\item[Usage]
\begin{verbatim}
GetInput(<textbox>,<escape-key>)
\end{verbatim}

\item[Example]	

\item[See Also]	\verb+SetEditable()+, \verb+GetCursorPosition()+, \verb+MakeTextBox()+, \verb+SetText()+
\end{desc}

\rl

\begin{desc}{Name/Symbol}
\item[Name/Symbol]	\verb+GetMouseCursorPosition()+

\item[Description] Gets the current x,y coordinates of the mouse
  pointer.

\item[Usage]
\begin{verbatim}
   GetMouseCursorPosition()
\end{verbatim}

\item[Example]	
\begin{verbatim}


 pos <- GetMouseCursorPosition()
\end{verbatim}


\item[See Also]
  \verb+ShowCursor+, \verb+WaitForMouseButton+,
  \verb+SetMouseCursorPosition+, \verb+GetMouseCursorPosition+
\end{desc}

\rl


\begin{desc}{Name/Symbol}
\item[Name/Symbol]	\verb+GetNIMHDemographics()+

\item[Description]	Gets demographic information that are normally required for NIMH-related research.  Currently are gender (M/F/prefer not to say), ethnicity (Hispanic or not), and race (A.I./Alaskan, Asian/A.A., Hawaiian, black/A.A., white/Caucasian, other).  
		It then prints their responses in a single line in the demographics file, along with any special code you supply and a time/date stamp. This code might include a subject number, experiment number, or something else, but many informed consent forms assure the subject that this information cannot be tied back to them or their data, so be careful about what you record. The file output will look something like: 
\begin{verbatim}
---- 
x0413 Thu Apr 22 17:58:15 2004 1 Y 4 
x0413 Thu Apr 23 17:58:20 2004 3 Y 5 
x0413 Thu Apr 24 12:41:30 2004 2 Y 5 
x0413 Thu Apr 24 14:11:54 2004 2 N 5 
---- 
\end{verbatim}


	The first column is the user-specified code (in this 
	case, indicating the experiment number).  The middle columns 
	indicate date/time, and the last three columns indicate 
	gender (M, F, other), Hispanic (Y/N), and race.

\item[Usage]
\begin{verbatim}
GetNIMHDemographics(<code-to-print-out>, <window>, <filename>)
\end{verbatim} 

\item[Example]
\begin{verbatim}
GetNIMHDemographics("x0413", gwindow, "x0413-demographics.txt")
\end{verbatim}

\item[See Also]	
\end{desc}

\rl




\begin{desc}{Name/Symbol}
\item[Name/Symbol]	\verb+GetPEBLVersion()+

\item[Description]	Returns a string describing which version of PEBL you are running.

\item[Usage]
\begin{verbatim}
GetPEBLVersion() 
\end{verbatim}

\item[Example]
\begin{verbatim}
Print(GetPEBLVersion())
\end{verbatim}

\item[See Also]	\verb+TimeStamp()+
\end{desc}

\rl



\begin{desc}{Name/Symbol}
\item[Name/Symbol]	\verb+GetSize()+

\item[Description] Returns a list of \verb+[height, width]+,
  specifying the size of the widget.
  The .width and .height properties can also be used instead of this function

\item[Usage]
\begin{verbatim}
GetSize(<widget>)
\end{verbatim}

\item[Example]
\begin{verbatim}
image <- MakeImage("stim1.bmp")
xy <- GetSize(image)
x <- Nth(xy, 1)
y <- Nth(xy, 2)
\end{verbatim}

\item[See Also]	
\end{desc}

\rl


\begin{desc}{Name/Symbol}
\item[Name/Symbol]	\verb+GetSubNum()+

\item[Description]	Creates dialog to ask user to input a subject code

\item[Usage]
\begin{verbatim}
GetSubNum(<win>)
\end{verbatim}

\item[Example]

\begin{verbatim}
## Put this at the beginning of an experiment, 
## after a window gWin has been defined.
##
 if(gSubNum == 0)
  {
    gSubNum <- GetSubNum(gWin)
  }
\end{verbatim}
Note: gSubNum can also be set from the command line.
\item[See Also]
\end{desc}



\rl


\begin{desc}{Name/Symbol}
\item[Name/Symbol]	\verb+GetText()+

\item[Description]	Returns the text stored in a text object 
		(either a textbox or a label).  The .text properties can also
  be used instead of this function.

\item[Usage]
\begin{verbatim}
GetText(<widget>)
\end{verbatim}

\item[Example]	

\item[See Also]	\verb+SetCursorPosition()+, \verb+GetCursorPosition()+, \verb+SetEditable()+, \verb+MakeTextBox()+
\end{desc}

\rl



\begin{desc}{Name/Symbol}
\item[Name/Symbol]	\verb+GetTime()+

\item[Description] Gets time, in milliseconds, from when PEBL was
  initialized.  Do not use as a seed for the RNG, because it will tend
  to be about the same on each run. Instead, use \verb+RandomizeTimer()+.

\item[Usage]
\begin{verbatim}
GetTime()
\end{verbatim}

\item[Example]
\begin{verbatim}
a <- GetTime()
WaitForKeyDown("A")
b <- GetTime()
Print("Response time is: " + (b - a))
\end{verbatim}

\item[See Also]	\verb+TimeStamp()+
\end{desc}

\rl

\section{H}
\rl



\begin{desc}{Name/Symbol}
\item[Name/Symbol]	\verb+Hide()+ 

\item[Description]	Makes an object invisible, so it will not be drawn.

\item[Usage]
\begin{verbatim}
Hide(<object>)
\end{verbatim}

\item[Example]
\begin{verbatim}
window <- MakeWindow()
image1  <- MakeImage("pebl.bmp")
image2  <- MakeImage("pebl.bmp")
AddObject(image1, window)
AddObject(image2, window)
Hide(image1)
Hide(image2)
Draw()		# empty screen will be drawn.
	
Wait(3000)
Show(image2)
Draw()		# image2 will appear.

Hide(image2)
Draw()		# image2 will disappear.

Wait(1000)
Show(image1)
Draw()		# image1 will appear.
\end{verbatim}
 
\item[See Also]	\verb+Show()+
\end{desc}

\rl


\section{I}
\rl


\begin{desc}{Name/Symbol}
\item[Name/Symbol]	\verb+if+ 

\item[Description]	Simple conditional test.

\item[Usage]
\begin{verbatim}
if(test)
{
 statements
 to
 be 
 executed
}
\end{verbatim}

\item[Example]	

\item[See Also]	
\end{desc}

\rl




\begin{desc}{Name/Symbol}
\item[Name/Symbol]	\verb+if...elseif...else+            

\item[Description] Complex conditional test.  Be careful of spacing
  the else---if you put carriage returns on either side of it, you
  will get a syntax error. The \verb+elseif+ is optional, but
  multiple \verb+elseif+ statements can be strung together.  The
  \verb+else+ is also optional, although only one can appear.

\item[Usage]
\begin{verbatim}
if(test)
{
 statements if true
} elseif (newtest) {
 statements if newtest true; test false
} else {
 other statements
} 
\end{verbatim}

\item[Example]	
\begin{verbatim}
 if(3 == 1) {
             Print("ONE")
  }elseif(3==4){
             Print("TWO")
  }elseif(4==4){
             Print("THREE")
  }elseif(4==4){
             Print("FOUR")
  }else{Print("FIVE")}
\end{verbatim}
\item[See Also]	
if
\end{desc}

\rl


\begin{desc}{Name/Symbol}
\item[Name/Symbol]	\verb+IsAnyKeyDown()+

\item[Description]	

\item[Usage]		

\item[Example]	

\item[See Also]	
\end{desc}

\rl


\begin{desc}{Name/Symbol}
\item[Name/Symbol]	\verb+IsAudioOut()+

\item[Description]	Tests whether \verb+<variant>+ is a AudioOut stream.

\item[Usage]
\begin{verbatim}
IsAudioOut(<variant>)
\end{verbatim}

\item[Example]
\begin{verbatim}
if(IsAudioOut(x))
{
 Play(x)
}
\end{verbatim}

\item[See Also] \verb+IsColor()+, \verb+IsImage()+,
  \verb+IsInteger()+, \verb+IsFileStream()+, \verb+IsFloat()+,
  \verb+IsFont()+, \verb+IsLabel()+, \verb+IsList()+,
  \verb+IsNumber()+, \verb+IsString()+, \verb+IsTextBox()+,
  \verb+IsWidget()+
\end{desc}

\rl


\begin{desc}{Name/Symbol}
\item[Name/Symbol]	\verb+IsColor()+

\item[Description]	Tests whether \verb+<variant>+ is a Color.

\item[Usage]
\begin{verbatim}
IsColor(<variant>)
\end{verbatim}

\item[Example]
\begin{verbatim}
if(IsColor(x)
{
 gWin <- MakeWindow(x)
}
\end{verbatim}

\item[See Also] \verb+IsAudioOut()+, \verb+IsImage()+,
  \verb+IsInteger()+, \verb+IsFileStream()+, \verb+IsFloat()+,
  \verb+IsFont()+, \verb+IsLabel()+, \verb+IsList()+,
  \verb+IsNumber()+, \verb+IsString()+, \verb+IsTextBox()+,
  \verb+IsWidget()+
\end{desc}

\rl




\begin{desc}{Name/Symbol}
\item[Name/Symbol]	\verb+IsImage()+

\item[Description]	Tests whether \verb+<variant>+ is an Image.

\item[Usage]
\begin{verbatim}
IsImage(<variant>)
\end{verbatim}

\item[Example]	
\begin{verbatim}
if(IsImage(x))
{
 AddObject(gWin, x)
}
\end{verbatim}

\item[See Also] \verb+IsAudioOut()+, \verb+IsColor()+,
  \verb+IsInteger()+, \verb+IsFileStream()+, \verb+IsFloat()+,
  \verb+IsFont()+, \verb+IsLabel()+, \verb+IsList()+,
  \verb+IsNumber()+, \verb+IsString()+, \verb+IsTextBox()+,
  \verb+IsWidget()+
\end{desc}

\rl




\begin{desc}{Name/Symbol}
\item[Name/Symbol]	\verb+IsInteger()+

\item[Description] Tests whether \verb+<variant>+ is an integer type.
  Note: a number represented internally as a floating-point type whose
  is an integer will return false.  Floating-point numbers can be
  converted to internally- represented integers with the
  \verb+ToInteger()+ or \verb+Round()+ commands.
 
\item[Usage]		
\begin{verbatim}
IsInteger(<variant>)
\end{verbatim}

\item[Example]
\begin{verbatim}
x <- 44
y <- 23.5
z <- 6.5
test <- x + y + z 
	
IsInteger(x)		# true
IsInteger(y)		# false
IsInteger(z)		# false
IsInteger(test)		# false
\end{verbatim}

\item[See Also] \verb+IsAudioOut()+, \verb+IsColor()+,
  \verb+IsImage()+, \verb+IsFileStream()+, \verb+IsFloat()+,
  \verb+IsFont()+, \verb+IsLabel()+, \verb+IsList()+,
  \verb+IsNumber()+, \verb+IsString()+, \verb+IsTextBox()+,
  \verb+IsWidget()+
\end{desc}

\rl




\begin{desc}{Name/Symbol}
\item[Name/Symbol]	\verb+IsFileStream()+

\item[Description]	Tests whether \verb+<variant>+ is a FileStream object.

\item[Usage]		
\begin{verbatim}
IsFileStream(<variant>)
\end{verbatim}

\item[Example]
\begin{verbatim}
if(IsFileStream(x))
{
 Print(FileReadWord(x)
}
\end{verbatim}

\item[See Also] \verb+IsAudioOut()+, \verb+IsColor()+,
  \verb+IsImage()+, \verb+IsInteger()+, \verb+IsFloat()+,
  \verb+IsFont()+, \verb+IsLabel()+, \verb+IsList()+,
  \verb+IsNumber()+, \verb+IsString()+, \verb+IsTextBox()+,
  \verb+IsWidget()+
\end{desc}

\rl




\begin{desc}{Name/Symbol}
\item[Name/Symbol]	\verb+IsFloat()+

\item[Description] Tests whether \verb+<variant>+ is a floating-point
  value. Note that floating-point can represent integers with great
  precision, so that a number appearing as an integer can still be a
  float.

\item[Usage]
\begin{verbatim}
IsFloat(<variant>)
\end{verbatim}

\item[Example]
\begin{verbatim}
x <- 44
y <- 23.5
z <- 6.5
test <- x + y + z 

IsFloat(x)     	# false
IsFloat(y)     	# true
IsFloat(z)     	# true
IsFloat(test)  	# true
\end{verbatim}

\item[See Also] \verb+IsAudioOut()+, \verb+IsColor()+,
  \verb+IsImage()+, \verb+IsInteger()+, \verb+IsFileStream()+,
  \verb+IsFont()+, \verb+IsLabel()+, \verb+IsList()+,
  \verb+IsNumber()+, \verb+IsString()+, \verb+IsTextBox()+,
  \verb+IsWidget()+
\end{desc}

\rl




\begin{desc}{Name/Symbol}
\item[Name/Symbol]	\verb+IsFont()+

\item[Description]	Tests whether \verb+<variant>+ is a Font object.

\item[Usage]
\begin{verbatim}
IsFont(<variant>)
\end{verbatim}

\item[Example]	
\begin{verbatim}
if(IsFont(x))
{
 y <- MakeLabel("stimulus", x)
}
\end{verbatim}

\item[See Also] \verb+IsAudioOut()+, \verb+IsColor()+,
  \verb+IsImage()+, \verb+IsInteger()+, \verb+IsFileStream()+,
  \verb+IsFloat()+, \verb+IsLabel()+, \verb+IsList()+,
  \verb+IsNumber()+, \verb+IsString()+, \verb+IsTextBox()+,
  \verb+IsWidget()+
\end{desc}

\rl




\begin{desc}{Name/Symbol}
\item[Name/Symbol]	\verb+IsKeyDown()+

\item[Description]	

\item[Usage]		

\item[Example]	

\item[See Also]	\verb+IsKeyUp()+
\end{desc}

\rl


\begin{desc}{Name/Symbol}
\item[Name/Symbol]	\verb+IsKeyUp()+

\item[Description]	

\item[Usage]		

\item[Example]	

\item[See Also]	\verb+IsKeyDown()+
\end{desc}

\rl


\begin{desc}{Name/Symbol}
\item[Name/Symbol]	\verb+IsLabel()+

\item[Description]	Tests whether \verb+<variant>+ is a text Label object.

\item[Usage]		
\begin{verbatim}
IsLabel(<variant>)
\end{verbatim}

\item[Example]	
\begin{verbatim}
if(IsLabel(x)
{
 text <- GetText(x)
}
\end{verbatim}

\item[See Also] \verb+IsAudioOut()+, \verb+IsColor()+,
  \verb+IsImage()+, \verb+IsInteger()+, \verb+IsFileStream()+,
  \verb+IsFloat()+, \verb+IsFont()+, \verb+IsList()+,
  \verb+IsNumber()+, \verb+IsString()+, \verb+IsTextBox()+,
  \verb+IsWidget()+
\end{desc}

\rl


\begin{desc}{Name/Symbol}
\item[Name/Symbol]	\verb+IsList()+

\item[Description]	Tests whether \verb+<variant>+ is a PEBL list.

\item[Usage]
\begin{verbatim}
IsList(<variant>)
\end{verbatim}

\item[Example]	
\begin{verbatim}
if(IsList(x))
{
 loop(item, x)
 {
  Print(item)
 }
}
\end{verbatim}

\item[See Also] \verb+IsAudioOut()+, \verb+IsColor()+,
  I\verb+sImage()+, \verb+IsInteger()+, \verb+IsFileStream()+,
  \verb+IsFloat()+, \verb+IsFont()+, \verb+IsLabel()+,
  \verb+IsNumber()+, \verb+IsString()+, \verb+IsTextBox()+,
  \verb+IsWidget()+
\end{desc}

\rl


\begin{desc}{Name/Symbol}
\item[Name/Symbol]	\verb+IsMember()+

\item[Description]	Returns true if \verb+<element>+ is a member of \verb+<list>+.

\item[Usage]		
\begin{verbatim}
IsMember(<element>,<list>)
\end{verbatim}

\item[Example]	
\begin{verbatim}
IsMember(2,[1,4,6,7,7,7,7])		# false
IsMember(2,[1,4,6,7,2,7,7,7]) 		# true
\end{verbatim}

\item[See Also]	
\end{desc}

\rl


\begin{desc}{Name/Symbol}
\item[Name/Symbol]	\verb+IsNumber()+

\item[Description]	Tests whether \verb+<variant>+ is a number, either a
		floating-point or an integer.

\item[Usage]		
\begin{verbatim}
IsNumber(<variant>)
\end{verbatim}

\item[Example]	
\begin{verbatim}
if(IsNumber(x))
{
 Print(Sequence(x, x+10, 1))
}
\end{verbatim}

\item[See Also] \verb+IsAudioOut()+, \verb+IsColor()+,
  \verb+IsImage()+, \verb+IsInteger()+, \verb+IsFileStream()+,
  \verb+IsFloat()+, \verb+IsFont()+, \verb+IsLabel()+,
  \verb+IsList()+, \verb+IsString()+, \verb+IsTextBox()+,
  \verb+IsWidget()+
\end{desc}

\rl




\begin{desc}{Name/Symbol}
\item[Name/Symbol]	\verb+IsString()+

\item[Description]	Tests whether \verb+<variant>+ is a text string.

\item[Usage]		
\begin{verbatim}
IsString(<variant>)
\end{verbatim}

\item[Example]	
\begin{verbatim}
if(IsString(x))
{
 tb <- MakeTextBox(x, 100, 100)
}
\end{verbatim}

\item[See Also]	\verb+IsAudioOut()+, \verb+IsColor()+, \verb+IsImage()+, \verb+IsInteger()+, 
		\verb+IsFileStream()+, \verb+IsFloat()+, \verb+IsFont()+, \verb+IsLabel()+,
		\verb+IsList()+, \verb+IsNumber()+, \verb+IsTextBox()+, \verb+IsWidget()+
\end{desc}

\rl




\begin{desc}{Name/Symbol}
\item[Name/Symbol]	\verb+IsTextBox()+

\item[Description]	Tests whether \verb+<variant>+ is a TextBox Object

\item[Usage]
\begin{verbatim}
IsTextBox(<variant>)
\end{verbatim}

\item[Example]	
\begin{verbatim}
if(IsTextBox(x))
{
 Print(GetText(x))
}
\end{verbatim}

\item[See Also] \verb+IsAudioOut()+, \verb+IsColor()+,
  \verb+IsImage()+, \verb+IsInteger()+, \verb+IsFileStream()+,
  \verb+IsFloat()+, \verb+IsFont()+, \verb+ IsLabel()+,
  \verb+IsList()+, \verb+IsNumber()+, \verb+IsString()+,
  \verb+IsWidget()+
\end{desc}

\rl

\begin{desc}{Name/Symbol}
\item[Name/Symbol]	\verb+IsWidget+

\item[Description]	Tests whether \verb+<variant>+ is any kind of a widget object
		(image, label, or textbox).

\item[Usage]		
\begin{verbatim}
IsWidget(<variant>)
\end{verbatim}

\item[Example]	
\begin{verbatim}
if(IsWidget(x))
{
 Move(x, 200,300)
}
\end{verbatim}

\item[See Also] \verb+IsAudioOut()+, \verb+IsColor()+,
  \verb+IsImage()+, \verb+IsInteger()+, \verb+IsFileStream()+,
  \verb+IsFloat()+, \verb+IsFont()+, \verb+IsLabel()+,
  \verb+IsList()+, \verb+IsNumber()+, \verb+IsString()+,
  \verb+IsTextBox()+
\end{desc}

\rl




%%% Local Variables: 
%%% mode: latex
%%% TeX-master: "main"
%%% End: 
\section{L}
\rl

\begin{desc}{Name/Symbol}
\item[Name/Symbol]	\verb+Last()+

\item[Description]	Returns the last item in a list. Provides faster 
		access to the last item of a list than does Nth().

\item[Usage]
\begin{verbatim}
Last(<list>)
\end{verbatim}

\item[Example]
\begin{verbatim}
Last([1,2,3,444])	# == 444
\end{verbatim}

\item[See Also]	\verb+Nth()+, \verb+First()+
\end{desc}

\rl





\begin{desc}{Name/Symbol}
\item[Name/Symbol]	\verb+LatinSquare()+

\item[Description]	Quick and dirty latin square, taking on just one
  list argument.

\item[Usage]
\begin{verbatim}
LatinSquare(<list>)
\end{verbatim}

\item[Example]
\begin{verbatim}
Print(LatinSquare([11,12,13,14,15,16]))
# Output:
#[[11, 12, 13, 14, 15, 16]
#, [12, 13, 14, 15, 16, 11]
#, [13, 14, 15, 16, 11, 12]
#, [14, 15, 16, 11, 12, 13]
#, [15, 16, 11, 12, 13, 14]
#, [16, 11, 12, 13, 14, 15]
#]

\end{verbatim}

\item[See Also] \verb+DesignFullCounterBalance()+,
  \verb+DesignBalancedSampling()+, \verb+DesignGrecoLatinSquare()+,
  \verb+DesignLatinSquare()+, \verb+Repeat()+, \verb+RepeatList()+,
  \verb+Shuffle()+

\end{desc}

\rl




\begin{desc}{Name/Symbol}
\item[Name/Symbol]	\verb+Line()+

\item[Description] Creates a line for graphing at x,y ending at x+dx,
  y+dy.  dx and dy describe the size of the line.  Lines must be added
  to a parent widget before it can be drawn; it may be added to
  widgets other than a base window. Properties of lines may be
  accessed and set later.

\item[Usage]
\begin{verbatim}
Line(<x>, <y>, <dx>, <dy>, <color>)
\end{verbatim}

\item[Example]	
\begin{verbatim}
  l <- Line(30,30,20,20, MakeColor("green")
  AddObject(l, win)
  Draw()

\end{verbatim}
\item[See Also]	\verb+Square()+, \verb+Ellipse()+, \verb+Rectangle()+, \verb+Circle()+
\end{desc}

\rl

\begin{desc}{Name/Symbol}
\item[Name/Symbol]	\verb+List()+

\item[Description]	Creates a list of items. Functional version of \verb+[]+.

\item[Usage]
\begin{verbatim}
List(<item1>, <item2>, ....)
\end{verbatim}

\item[Example]
\begin{verbatim}
List(1,2,3,444)		# == [1,2,3,444]
\end{verbatim}

\item[See Also]	\verb+[ ]+, \verb+Merge()+, \verb+Append()+
\end{desc}

\rl

 
\begin{desc}{Name/Symbol}
\item[Name/Symbol]	\verb+ListToString()+

\item[Description]	Converts a list of things to a single string

\item[Usage]
\begin{verbatim}
  ListToString(<list>)
\end{verbatim}

\item[Example]
\begin{verbatim}
ListToString([1,2,3,444])		# == "123444"
ListToString(["a","b","c","d","e"])		# == "abcde"

\end{verbatim}

\item[See Also] \verb+SubString+, \verb+StringLength+
\end{desc}

\rl


\begin{desc}{Name/Symbol}
\item[Name/Symbol]	\verb+Length()+

\item[Description]	Returns the number of items in a list.

\item[Usage]
\begin{verbatim}
Length(<list>)
\end{verbatim}

\item[Example]
\begin{verbatim}
Length([1,3,55,1515])	# == 4
\end{verbatim}

\item[See Also]	\verb+StringLength()+
\end{desc}

\rl



\begin{desc}{Name/Symbol}
\item[Name/Symbol]	\verb+LoadSound()+

\item[Description]	Loads a soundfile from \verb+<filename>+, 
		returning a variable that can be played.

\item[Usage]
\begin{verbatim}
LoadSound(<filename>)
\end{verbatim}

\item[Example]	

\item[See Also]	
\end{desc}

\rl



\begin{desc}{Name/Symbol}
\item[Name/Symbol]	\verb+Log10()+

\item[Description]	Log base 10 of \verb+<num>+.

\item[Usage]
\begin{verbatim}
Log10(<num>)
\end{verbatim}

\item[Example]	

\item[See Also]	\verb+Log2()+, \verb+LogN()+, \verb+Ln()+, \verb+Exp()+
\end{desc}

\rl


\begin{desc}{Name/Symbol}
\item[Name/Symbol]	\verb+Log2()+

\item[Description]	Log base 2 of \verb+<num>+.

\item[Usage]
\begin{verbatim}
Log2(<num>)
\end{verbatim}

\item[Example]	

\item[See Also]	\verb+Log()+, \verb+LogN()+, \verb+Ln()+, \verb+Exp()+
\end{desc}

\rl


\begin{desc}{Name/Symbol}
\item[Name/Symbol]	\verb+LogN()+

\item[Description]	Log base \verb+<base>+ of \verb+<num>+.

\item[Usage]
\begin{verbatim}
LogN(<num>, <base>)
\end{verbatim}

\item[Example]
\begin{verbatim}
LogN(100,10)	# == 2
LogN(256,2)	# == 8
\end{verbatim}

\item[See Also]	\verb+Log()+, \verb+Log2()+, \verb+Ln()+, \verb+Exp()+
\end{desc}

\rl


\begin{desc}{Name/Symbol}
\item[Name/Symbol]	\verb+Lowercase()+

\item[Description]	Changes a string to lowercase.  Useful for testing user
		input against a stored value, to ensure case differences
		are not detected.

\item[Usage]
\begin{verbatim}
Lowercase(<string>)
\end{verbatim}

\item[Example]
\begin{verbatim}
Lowercase("POtaTo")	# == "potato"
\end{verbatim}

\item[See Also]	\verb+Uppercase()+
\end{desc}

\rl


\begin{desc}{Name/Symbol}
\item[Name/Symbol]	\verb+Ln()+

\item[Description]	Natural log of \verb+<num>+.

\item[Usage]		
\begin{verbatim}
Ln(<num>)
\end{verbatim}

\item[Example]	

\item[See Also]	\verb+Log()+, \verb+Log2()+, \verb+LogN()+, \verb+Exp()+     
\end{desc}

\rl






\begin{desc}{Name/Symbol}
\item[Name/Symbol]	\verb+Lookup()+

\item[Description] Returns
element in \verb+<database>+ corresponding to element of
\verb+<keylist>+ that matches \verb+<key>+.

\item[Usage]		
\begin{verbatim}
Lookup(<key>,<keylist>,<database>)
\end{verbatim}

\item[Example]	

\begin{verbatim}
 keys     <- [1,2,3,4,5]
 database <- ["market","home","roast beef","none","wee wee wee"]
 Print(Lookup(3,keys,database))) 

## Or, do something like this:
  
data  <- [["punky","brewster"],
          ["arnold","jackson"],
          ["richie","cunningham"],
          ["alex","keaton"]]

d2 <- Transpose(data)
key <- First(data)

Print(Lookup("alex", key, data))
##Returns ["alex","keaton"]
\end{verbatim}
\item[See Also]	
\end{desc}

\rl


\begin{desc}{Name/Symbol}
\item[Name/Symbol]	\verb+loop()+

\item[Description]	Loops over elements in a list.  During each iteration, \verb+<counter>+ is bound to each consecutive member of \verb+<list>+.

\item[Usage]		
\begin{verbatim}
loop(<counter>, <list>)
{
 statements
 to
 be	   
 executed
}
\end{verbatim}

\item[Example]	

\item[See Also]	\verb+while()+, \verb+{ }+
\end{desc}

\rl

\section{M}
\rl




\begin{desc}{Name/Symbol}
\item[Name/Symbol]	\verb+MakeColor()+

\item[Description] Makes a color from \verb+<colorname>+ such as
  ``red'', ``green'', and nearly 800 others.  Color names and
  corresponding RGB values can be found in \verb+doc/colors.txt+.

\item[Usage]
\begin{verbatim}
MakeColor(<colorname>)
\end{verbatim}

\item[Example]	

\item[See Also]	\verb+MakeColorRGB()+
\end{desc}

\rl


\begin{desc}{Name/Symbol}
\item[Name/Symbol]	\verb+MakeColorRGB()+ 

\item[Description] Makes an RGB color by specifying \verb+<red>+,
  \verb+<green>+, and \verb+<blue>+ values (between 0 and 255).

\item[Usage]		
\begin{verbatim}
MakeColorRGB(<red>, <green>, <blue>)
\end{verbatim}

\item[Example]	

\item[See Also]	\verb+MakeColor()+
\end{desc}

\rl





\begin{desc}{Name/Symbol}
\item[Name/Symbol]	\verb+MakeFont()+

\item[Description]	Makes a font.

\item[Usage]
\begin{verbatim}
MakeFont(<ttf_filename>, <style>, <size>, 
<fgcolor>, <bgcolor>, <anti-aliased>)
\end{verbatim}

\item[Example]	

\item[See Also]	
\end{desc}

\rl



\begin{desc}{Name/Symbol}
\item[Name/Symbol]	\verb+MakeImage()+

\item[Description]	Makes an image widget from an image file.
		\texttt{.bmp} formats should be supported; others may be as well.

\item[Usage]		
\begin{verbatim}

MakeImage(<filename>)
\end{verbatim}

\item[Example]	

\item[See Also]	
\end{desc}

\rl




\begin{desc}{Name/Symbol}
\item[Name/Symbol]	\verb+MakeLabel()+

\item[Description] Makes a text label for display on-screen. Text will
  be on a single line, and the \verb+Move()+ command centers
  \verb+<text>+ on the specified point.

\item[Usage]
\begin{verbatim}
MakeLabel(<text>, <font>)
\end{verbatim}

\item[Example]	

\item[See Also]	
\end{desc}

\rl

\begin{desc}{Name/Symbol}
\item[Name/Symbol]	\verb+MakeNGonPoints()+

\item[Description] 
Creates a set of points that form a regular n-gon.  It can be
transformed with functions like \verb+RotatePoints+, or it can be 
used to create a graphical object with \verb+Polygon+.

Note: MakeNGonPoints returns a list [[x1, x2, x3,...],[y1,y2,y3,...]],
while Polygon() takes the X and Y lists independently.


\item[Usage]
\begin{verbatim}
MakeNGonPoints(<radius>, <num_peaks>)
\end{verbatim}

\item[Example]	
\begin{verbatim}
   window <- MakeWindow()
   ngonp <- MakeNGonPoints(50,10)
   ngon <- PolyGon(200,200,First(ngonp),Nth(ngonp,2),MakeColor("red"),1)
   AddObject(ngon,window)
   Draw()
\end{verbatim}

\item[See Also]	
\verb+MakeStarPoints+, \verb+Polygon+, \verb+RotatePoints+, \verb+ZoomPoints+
\end{desc}

\rl


\begin{desc}{Name/Symbol}
\item[Name/Symbol]	\verb+MakeStarPoints()+

\item[Description] 
Creates a set of points that form a regular star.  It can be
transformed with functions like \verb+RotatePoints+, or it can be 
used to create a graphical object with \verb+Polygon+.

Note: MakeStarPoints returns a list [[x1, x2, x3,...],[y1,y2,y3,...]],
while Polygon() takes the X and Y lists independently.

\item[Usage]
\begin{verbatim}
MakeNGonPoints(<outer_radius>, <inner_radius>, <num_peaks>)
\end{verbatim}

\item[Example]	
\begin{verbatim}
   window <- MakeWindow()
   sp <- MakeStarPoints(50,20,10)
   star <- PolyGon(200,200,First(sp),Nth(sp,2),MakeColor("red"),1)
   AddObject(star,window)
   Draw()
\end{verbatim}

\item[See Also]	
\verb+MakeNGonPoints+, \verb+Polygon+, \verb+RotatePoints+, \verb+ZoomPoints+
\end{desc}

\rl



\begin{desc}{Name/Symbol}
\item[Name/Symbol]	\verb+MakeTextBox()+

\item[Description]	Creates a textbox in which to display text. 
		Textboxes allow multiple lines of text to be rendered;
		automatically breaking the text into lines. 

\item[Usage]
\begin{verbatim}
MakeWindow(<text>,<font>,<width>,<height>)
\end{verbatim}

\item[Example]	
\begin{verbatim}
font <-MakeFont("Vera.ttf", 1, 12, MakeColor("red"), 
MakeColor("green"), 1)
tb <- MakeTextBox("This is the text in the textbox", 
font, 100, 250)
\end{verbatim}

\item[See Also]	\verb+MakeLabel()+, \verb+GetText()+, \verb+SetText()+, \verb+SetCursorPosition()+,
		\verb+GetCursorPosition()+, \verb+SetEditable()+
\end{desc}

\rl


\begin{desc}{Name/Symbol}
\item[Name/Symbol]	\verb+MakeWindow()+ 

\item[Description]	Creates a window to display things in.
		Background is specified by \verb+<color>+.

\item[Usage]		
\begin{verbatim}
MakeWindow(<color>)
\end{verbatim}

\item[Example]	

\item[See Also]	
\end{desc}

\rl


\begin{desc}{Name/Symbol}
\item[Name/Symbol]	\verb+Max()+            

\item[Description] Returns the largest of \verb+<list>+.

\item[Usage]		
\begin{verbatim}
Max(<list>)
\end{verbatim}

\item[Example]	
\begin{verbatim} 
  c <- [3,4,5,6]
  m <- Max(c) # m == 6
\end{verbatim}

\item[See Also]	\verb+Min()+, \verb+Mean()+, \verb+StDev()+
\end{desc}

\rl





\begin{desc}{Name/Symbol}
\item[Name/Symbol]	\verb+Mean()+

\item[Description] 	Returns the mean of the numbers in \verb+<list>+.

\item[Usage]	Mean(\verb+<list-of-numbers>+)	

\item[Example]	
\begin{verbatim} 
  c <- [3,4,5,6]
  m <- Mean(c) # m == 4.5
\end{verbatim}

\item[See Also]	\verb+Median()+, \verb+Quantile()+, \verb+StDev()+, \verb+Min()+, \verb+Max()+
\end{desc}

\rl



\begin{desc}{Name/Symbol}
\item[Name/Symbol]	\verb+Median()+

\item[Description]	Returns the median of the numbers in
  \verb+<list>+.  

\item[Usage]	Median(\verb+<list-of-numbers>+)

\item[Example]	
\begin{verbatim} 
  c <- [3,4,5,6,7]
  m <- Median(c) # m == 5
\end{verbatim}
\item[See Also]	\verb+Mean()+, \verb+Quantile()+, \verb+StDev()+, \verb+Min()+, \verb+Max()+
\end{desc}

\rl


\begin{desc}{Name/Symbol}
\item[Name/Symbol]	\verb+Merge()+

\item[Description]	Combines two lists, \verb+<lista>+ and \verb+<listb>+, into a single list.

\item[Usage]		
\begin{verbatim}
Merge(<lista>,<listb>)
\end{verbatim}

\item[Example]	
\begin{verbatim}
Merge([1,2,3],[8,9]) 	# == [1,2,3,8,9]
\end{verbatim}

\item[See Also]	\verb+[ ]+, \verb+Append()+, \verb+List()+
\end{desc}

\rl


\begin{desc}{Name/Symbol}
\item[Name/Symbol]	\verb+Min()+

\item[Description]	Returns the `smallest' element of a list.

\item[Usage]	
\begin{verbatim}
Min(<list>)
\end{verbatim}

\item[Example]	
\begin{verbatim}
  c <- [3,4,5,6]
  m <-  Min(c) # == 3
\end{verbatim}

\item[See Also]	\verb+Max()+
\end{desc}

\rl


\begin{desc}{Name/Symbol}
\item[Name/Symbol]	\verb+Mod()+

\item[Description]	Returns \verb+<num>+, \verb+<mod>+, or remainder of \verb+<num>/<mod>+

\item[Usage]		
\begin{verbatim}
Mod( <num> <mod>)
\end{verbatim}

\item[Example]	
\begin{verbatim}
Mod(34, 10)	# == 4
Mod(3, 10)	# == 3
\end{verbatim}

\item[See Also]	\verb+Div()+
\end{desc}

\rl


\begin{desc}{Name/Symbol}
\item[Name/Symbol]	\verb+Move()+

\item[Description]	Moves an object to a specified location.  
		Images and Labels are moved according to their center; 
		TextBoxes are moved according to their upper left corner.

\item[Usage]
\begin{verbatim}
Move(<object>, <x>, <y>)
\end{verbatim}

\item[Example]	
\begin{verbatim}
Move(label, 33, 100)
\end{verbatim}

\item[See Also]	\verb+MoveCorner()+, \verb+MoveCenter()+, \verb+.X+ and \verb+.Y+ properties.
\end{desc}

\rl


\begin{desc}{Name/Symbol}
\item[Name/Symbol]	\verb+MoveCorner()+

\item[Description]	Moves a label or image to a specified location
		according to its upper left corner, instead of its center. 

\item[Usage]
\begin{verbatim}
MoveCorner(<object>, <x>, <y>)
\end{verbatim}

\item[Example]	
\begin{verbatim}
MoveCorner(label, 33, 100)
\end{verbatim}

\item[See Also]	\verb+Move()+, \verb+MoveCenter()+, \verb+.X+ and \verb+.Y+ properties
\end{desc}

\rl




\begin{desc}{Name/Symbol}
\item[Name/Symbol]	\verb+MoveCenter()+

\item[Description]	Moves a TextBox to a specified location
		according to its center, instead of its upper left corner.

\item[Usage]
\begin{verbatim}
MoveCenter(<object>, <x>, <y>)
\end{verbatim}

\item[Example]	
\begin{verbatim}
MoveCenter(TextBox, 33, 100)
\end{verbatim}

\item[See Also]	\verb+Move()+, \verb+MoveCorner()+, \verb+.X+ and \verb+.Y+ properties
\end{desc}

\rl


\begin{desc}{Name/Symbol}
\item[Name/Symbol]	\verb+not+

\item[Description]	Logical not

\item[Usage]		

\item[Example]	

\item[See Also]	\verb+and+, \verb+or+
\end{desc}

\rl
\section{N}
\rl

\begin{desc}{Name/Symbol}
\item[Name/Symbol]	\verb+NormalDensity()+

\item[Description]	Computes density of normal standard distribution
\item[Usage]
\begin{verbatim}
NormalDensity(<x>)
\end{verbatim}

\item[Example]	
\begin{verbatim}


  Print(NormalDensity(-100))     # 1.8391e-2171
  Print(NormalDensity(-2.32635)) #5.97
  Print(NormalDensity(0))        #0.398942
  Print(NormalDensity(1.28155))  #.90687
  Print(NormalDensity(1000))     #inf

\end{verbatim}

\item[See Also]	\verb+RandomNormal()+, \verb+CumNormInv()+ 
\end{desc}

\rl




\begin{desc}{Name/Symbol}
\item[Name/Symbol]	\verb+Nth()+

\item[Description]	Extracts the Nth item from a list.  Indexes from 1 upwards.
		\verb+Last()+ provides faster access than \verb+Nth()+ to the end of a list, 
		which must walk along the list to the desired position.

\item[Usage]
\begin{verbatim}
Nth(<list>, <index>)
\end{verbatim}

\item[Example]	
\begin{verbatim}
a <- ["a","b","c","d"]
Print(Nth(a,3)) 		# == 'c'
\end{verbatim}

\item[See Also]	\verb+First()+, \verb+Last()+ 
\end{desc}

\rl


\begin{desc}{Name/Symbol}
\item[Name/Symbol]	\verb+NthRoot()+

\item[Description]	\verb+<num>+ to the power of  1/\verb+<root>+.

\item[Usage]		
\begin{verbatim}
NthRoot(<num>, <root>)
\end{verbatim}

\item[Example]	

\item[See Also]	
\end{desc}

\rl
\section{O}
\rl


\begin{desc}{Name/Symbol}
\item[Name/Symbol]	\verb+or+                   

\item[Description]	Logical or

\item[Usage]		

\item[Example]	

\item[See Also]	\verb+and+, \verb+not+
\end{desc}

\rl






\begin{desc}{Name/Symbol}
\item[Name/Symbol]	\verb+Order()+

\item[Description]	Returns a list of indices describing the order of values by position, from min to max. 

\item[Usage]
\begin{verbatim}
		Order(<list-of-numbers>)
\end{verbatim}

\item[Example]	
\begin{verbatim}
	n <- [33,12,1,5,9]
  	o <- Order(n)
    Print(o) #should print [3,4,5,2,1]
\end{verbatim}

\item[See Also]	\verb+Rank()+
\end{desc}

\rl
\section{P}
\rl


\begin{desc}{Name/Symbol}
\item[Name/Symbol]	\verb+PlayForeground()+  

\item[Description]	Plays the sound `in the foreground'; 
		does not return until the sound is complete.

\item[Usage]		
\begin{verbatim}
PlayForeground(<sound>)
\end{verbatim}

\item[Example]	

\item[See Also]	\verb+PlayBackground()+, \verb+Stop()+
\end{desc}

\rl


\begin{desc}{Name/Symbol}
\item[Name/Symbol]	\verb+PlayBackground()+
 
\item[Description]	Plays the sound `in the background', returning immediately.

\item[Usage]		
\begin{verbatim}
PlayBackground(<sound>)
\end{verbatim}

\item[Example]	

\item[See Also]	\verb+PlayForeground()+, \verb+Stop()+
\end{desc}

\rl



\begin{desc}{Name/Symbol}
\item[Name/Symbol]  	\verb+Plus+ 

\item[Description] Creates a polygon in the shape of a
 plus sign. Arguments include position in window.
\begin{itemize}
\item \verb+<x>+ and \verb+<y>+ is the position of the center
\item \verb+<size>+ or the size of the plus sign in pixels
\item \verb+<width>+ thickness of the plus
\item \verb+<color>+ is a color object (not just the name)
\end{itemize}

Like other drawn objects, the plus must then be added to the window
to appear.

\item[Usage]		
\begin{verbatim}
 Plus(x,y,size,width,color)
\end{verbatim}

\item[Example]	
\begin{verbatim}
  win <- MakeWindow()
  p1 <- Plus(100,100,80,15,MakeColor("red"))
  AddObject(p1,win)
  Draw()
\end{verbatim}

\item[See Also]   
\verb+BlockE()+, \verb+Polygon()+, \verb+MakeStarPoints()+,
\verb+MakeNGonPoints()+
\end{desc}

\rl


\begin{desc}{Name/Symbol}
\item[Name/Symbol]  	\verb+Polygon+ 

\item[Description] Creates a polygon in the shape of the points
specified by \verb+<xpoints>+, \verb+<ypoints>+. The lists \verb+<xpoints>+ and
\verb+<ypoints>+ are adjusted by  \verb+<x>+ and \verb+<y>+, so they
should be relative to 0, not the location you want the points to be at.

Like other drawn objects, the polygon must then be added to the window
to appear.

\item[Usage]		
\begin{verbatim}
  Polygon(<x>,<y>,<xpoints>,<ypoints>,<color>,<filled>)
\end{verbatim}

\item[Example]	
\begin{verbatim}
  win <- MakeWindow()
   #This makes a T
   xpoints <- [-10,10,10,20,20,-20,-20,-10]
   ypoints <- [-20,-20,40,40,50,50,40,40]
  p1 <-    Polygon(100,100,xpoints, ypoints,MakeColor("black"),1)
  AddObject(p1,win)
  Draw()
\end{verbatim}

\item[See Also]   
\verb+BlockE()+, \verb+Bezier()+, \verb+MakeStarPoints()+,
\verb+MakeNGonPoints()+
\end{desc}

\rl


\begin{desc}{Name/Symbol}
\item[Name/Symbol]	\verb+Pow()+ 

\item[Description]	Raises or lowers \verb+<num>+ to the power of \verb+<pow>+.

\item[Usage]		
\begin{verbatim}
Pow(<num>, <pow>)
\end{verbatim}

\item[Example]	
\begin{verbatim}
Pow(2,6)	# == 64
Pow(5,0)	# == 1
\end{verbatim}

\item[See Also]     
\end{desc}

\rl


\begin{desc}{Name/Symbol}
\item[Name/Symbol]	\verb+Print()+

\item[Description]	Prints \verb+<value>+ to stdout (the console [Linux] or the file \texttt{stdout.txt} [Windows]), and then appends a newline afterwards.

\item[Usage]		
\begin{verbatim}
Print(<value>)
\end{verbatim}

\item[Example]	
\begin{verbatim}
  Print("hello world")
  Print(33 + 43)
  x <-Print("Once")

\end{verbatim}
\item[See Also]	\verb+Print_()+, \verb+FilePrint()+
\end{desc}

\rl

\begin{desc}{Name/Symbol}
\item[Name/Symbol]	\verb+PrintProperties()+

\item[Description]	Prints .properties/values for any complex object.
  These include textboxes, fonts, colors, images, shapes, etc. Mostly
  useful as a debugging tool.

\item[Usage]		
\begin{verbatim}
PrintProperties(<object>)
\end{verbatim}

\item[Example]	
\begin{verbatim}


   win <- MakeWindow()
   tb <- EasyTextbox("one",20,20,win,22,400,80)
   PrintProperties(tb)

##Output:
----------
[CURSORPOS]: 0
[EDITABLE]: 0
[HEIGHT]: 80
[ROTATION]: 0
[TEXT]: one
[VISIBLE]: 1
[WIDTH]: 400
[X]: 20
[Y]: 20
[ZOOMX]: 1
[ZOOMY]: 1
----------

\end{verbatim}
\item[See Also]	\verb+Print()+
\end{desc}

\rl



\begin{desc}{Name/Symbol}
\item[Name/Symbol]	\verb+Print_()+

\item[Description]	Prints \verb+<value>+ to stdout; doesn't append a newline afterwards.

\item[Usage]		
\begin{verbatim}
Print_(<value>)
\end{verbatim}

\item[Example]	
\begin{verbatim}
Print_("This line")
Print_(" ")
Print_("and")
Print_(" ")
Print("Another line")
# prints out: 'This line and Another line'
\end{verbatim}

\item[See Also]	\verb+Print()+, \verb+FilePrint()+
\end{desc}

\rl

\begin{desc}{Name/Symbol}
\item[Name/Symbol]	\verb+PrintList()+

\item[Description]	Prints a list, without the ','s or []
  characters. Puts a carriage return at the end.  Returns a string
  that was printed.  If a list contains other lists, the printing will
  wrap multiple lines and the internal lists will be printed as
  normal.  To avoid this, try PrintList(file,Flatten(list)).

\item[Usage]
\begin{verbatim}
PrintList(<filestream>, <list>)
\end{verbatim}

\item[Example]
\begin{verbatim}
PrintList( [1,2,3,4,5,5,5])
##
##  Produces:
##1 2 3 4 5 5 5
PrintList([[1,2],[3,4],[5,6]])
#Produces:
# [1,2]
#,[3,4]
#,[5,6]

PrintList(Flatten([[1,2],[3,4],[5,6]]))
#Produces:
# 1 2 3 4 5 6

\end{verbatim}

\item[See Also]	\verb+Print()+, \verb+Print_()+, \verb+FilePrint()+, \verb+FilePrint_()+, \verb+FilePrintList()+,
\end{desc}

\rl




\section{Q}
\rl

\begin{desc}{Name/Symbol}
\item[Name/Symbol]	\verb+Quantile()+

\item[Description]	Returns the \verb+<num>+ quantile of
		the numbers in \verb+<list>+. \verb+<num>+ should be  between
        0 and 100

\item[Usage]		
\begin{verbatim}
Quantile(<list>, <num>)
\end{verbatim}

\item[Example]	
 \begin{verbatim}
      ##Find 75th percentile to use as a threshold.
      thresh <- Quantile(rts,75) 
 \end{verbatim}
\item[See Also]	\verb+StDev()+, \verb+Median()+, \verb+Mean()+, \verb+Max()+, \verb+Min()+
\end{desc}

\rl


\section{R}
\rl


\begin{desc}{Name/Symbol}
\item[Name/Symbol] 	\verb+RadToDeg()+ 

\item[Description] 	Converts \verb+<rad>+ radians to degrees.

\item[Usage]		
\begin{verbatim}
RadToDeg( <rad>)			 
\end{verbatim}

\item[Example]	

\item[See Also]     	\verb+DegToRad()+, \verb+Tan()+, \verb+Cos()+, \verb+Sin()+, \verb+ATan()+, \verb+ASin()+, \verb+ACos()+
\end{desc}

\rl



\begin{desc}{Name/Symbol}
\item[Name/Symbol]	\verb+Random()+

\item[Description]	Returns a random number between 0 and 1.

\item[Usage]
\begin{verbatim}
Random()
\end{verbatim}

\item[Example]
\begin{verbatim}
a <- Random()
\end{verbatim}

\item[See Also]		\verb+Random()+, \verb+RandomBernoulli()+, \verb+RandomBinomial()+, \verb+RandomDiscrete()+, \verb+RandomExponential()+, \verb+RandomLogistic()+, \verb+RandomLogNormal()+, \verb+RandomNormal()+, \verb+RandomUniform()+, \verb+RandomizeTimer()+, \verb+SeedRNG()+
\end{desc}

\rl


\begin{desc}{Name/Symbol}
\item[Name/Symbol]	\verb+RandomBernoulli()+

\item[Description]	Returns 0 with probability \verb+(1-<p>)+ and 1 with probability \verb+<p>+.

\item[Usage]		
\begin{verbatim}
RandomBernoulli(<p>)
\end{verbatim}

\item[Example]	
\begin{verbatim}
RandomBernoulli(.3)
\end{verbatim}

\item[See Also] \verb+Random()+, \verb+RandomBernoulli()+,
  \verb+RandomBinomial+, \verb+RandomDiscrete()+,
  \verb+RandomExponential()+, \verb+RandomLogistic()+,
  \verb+RandomLogNormal()+, \verb+RandomNormal()+,
  \verb+RandomUniform()+, \verb+RandomizeTimer()+, \verb+SeedRNG()+
\end{desc}

\rl


\begin{desc}{Name/Symbol}
\item[Name/Symbol]	\verb+RandomBinomial+

\item[Description] Returns a random number according to the Binomial
  distribution with probability \verb+<p>+ and repetitions \verb+<n>+,
  i.e., the number of \verb+<p>+ Bernoulli trials that succeed out of
  \verb+<n>+ attempts.

\item[Usage]		
\begin{verbatim}
RandomBinomial(<p> <n>)  
\end{verbatim}

\item[Example]	
\begin{verbatim}
RandomBinomial(.3, 10)		# returns a number from 0 to 10
\end{verbatim}

\item[See Also]	\verb+Random()+, \verb+RandomBernoulli()+, \verb+RandomBinomial+,
		\verb+RandomDiscrete()+, \verb+RandomExponential()+, \verb+RandomLogistic()+,
		\verb+RandomLogNormal()+, \verb+RandomNormal()+, \verb+RandomUniform()+,    
		\verb+RandomizeTimer()+, \verb+SeedRNG()+    
\end{desc}

\rl


\begin{desc}{Name/Symbol}
\item[Name/Symbol]	\verb+RandomDiscrete()+

\item[Description]	Returns a random integer between 1 and the argument 
		(inclusive), each with equal probability.  If the argument is 
		a floating-point value, it will be truncated down; if it is 
		less than 1, it will return 1, and possibly a warning message. 

\item[Usage]		
\begin{verbatim}
RandomDiscrete(<num>)
\end{verbatim}
         
\item[Example]	
\begin{verbatim}
RandomDiscrete(30) # Returns a random integer between 1 and 30
\end{verbatim}

\item[See Also]	\verb+Random()+, \verb+RandomBernoulli()+, \verb+RandomBinomial+, 
		\verb+RandomDiscrete()+, \verb+RandomExponential()+, \verb+RandomLogistic()+,
		\verb+RandomLogNormal()+, \verb+RandomNormal()+, \verb+RandomUniform()+,
		\verb+RandomizeTimer()+, \verb+SeedRNG()+    
\end{desc}

\rl


\begin{desc}{Name/Symbol}
\item[Name/Symbol]	\verb+RandomExponential()+

\item[Description]	Returns a random number according to exponential 
		distribution with mean \verb+<mean>+ (or decay 1/mean).

\item[Usage]		
\begin{verbatim}
RandomExponential(<mean>)
\end{verbatim}

\item[Example]	
\begin{verbatim}
RandomExponential(100)
\end{verbatim}

\item[See Also]	\verb+Random()+, \verb+RandomBernoulli()+, \verb+RandomBinomial+,
		\verb+RandomDiscrete()+, \verb+RandomLogistic()+, \verb+RandomLogNormal()+, 
		\verb+RandomNormal()+, \verb+RandomUniform()+, \verb+RandomizeTimer+, \verb+SeedRNG()+
\end{desc}

\rl                            


\begin{desc}{Name/Symbol}
\item[Name/Symbol]	\verb+RandomizeTimer()+

\item[Description]	Seeds the RNG with the current time.

\item[Usage]
\begin{verbatim}
RandomizeTimer()
\end{verbatim}

\item[Example]	
\begin{verbatim}
RandomizeTimer()
x <- Random()
\end{verbatim}
	     
\item[See Also]	\verb+Random()+, \verb+RandomBernoulli()+, \verb+RandomBinomial+,
		\verb+RandomDiscrete()+, \verb+RandomExponential()+, \verb+RandomLogistic()+,
		\verb+RandomLogNormal()+, \verb+RandomNormal()+, \verb+RandomUniform()+, \verb+SeedRNG()+
\end{desc}

\rl


\begin{desc}{Name/Symbol}
\item[Name/Symbol]	\verb+RandomLogistic()+  

\item[Description]	Returns a random number according to the logistic distribution 
		with parameter \verb+<p>+: f(x) = exp(x)/(1+exp(x))

\item[Usage]		
\begin{verbatim}
RandomLogistic(<p>)
\end{verbatim}

\item[Example]	RandomLogistic(.3)

\item[See Also]	\verb+Random()+, \verb+RandomBernoulli()+, \verb+RandomBinomial+, 
		\verb+RandomDiscrete()+, \verb+RandomExponential()+, \verb+RandomLogNormal()+, 
		\verb+RandomNormal()+, \verb+RandomUniform()+, \verb+RandomizeTimer+, \verb+SeedRNG()+
\end{desc}

\rl


\begin{desc}{Name/Symbol}
\item[Name/Symbol] 	\verb+RandomLogNormal()+

\item[Description]  	Returns a random number according to the log-normal 
		distribution with parameters \verb+<median>+ and \verb+<spread>+. Generated 
		by calculating median \verb!*! exp(spread \verb!*! RandomNormal(0,1)). 
		\verb+<spread>+ is a shape parameter, and only affects the variance 
		as a function of the median; similar to the coefficient of 
		variation.  A value near 0 is a sharp distribution (.1-.3), 
		larger values are more spread out; values greater than 2 make 
		little difference in the shape.

\item[Usage]
\begin{verbatim}
RandomLogNormal(<median>, <spread>)
\end{verbatim}

\item[Example]      	
\begin{verbatim}
RandomLogNormal(5000, .1)
\end{verbatim}

\item[See Also]	\verb+Random()+, \verb+RandomBernoulli()+, \verb+RandomBinomial+, 
		\verb+RandomDiscrete()+, \verb+RandomExponential()+, \verb+RandomLogistic()+,
		\verb+RandomNormal()+, \verb+RandomUniform()+, \verb+RandomizeTimer+, \verb+SeedRNG()+
\end{desc}

\rl


\begin{desc}{Name/Symbol}
\item[Name/Symbol] 	\verb+RandomNormal()+

\item[Description] 	Returns a random number according to the standard
             	normal distribution with \verb+<mean>+ and \verb+<stdev>+.

\item[Usage]       	
\begin{verbatim}
RandomNormal(<mean>, <stdev>)
\end{verbatim}

\item[Example]	

\item[See Also]	\verb+Random()+, \verb+RandomBernoulli()+, \verb+RandomBinomial+,
		\verb+RandomDiscrete()+, \verb+RandomExponential()+, \verb+RandomLogistic()+, 
		\verb+RandomLogNormal()+, \verb+RandomUniform()+, \verb+RandomizeTimer+, \verb+SeedRNG()+
\end{desc}

\rl


\begin{desc}{Name/Symbol}
\item[Name/Symbol]	\verb+RandomUniform()+

\item[Description]	Returns a random floating-point number between 0 and \verb+<num>+.

\item[Usage]		
\begin{verbatim}
RandomUniform(<num>)
\end{verbatim}

\item[Example]	

\item[See Also] \verb+Random()+, \verb+RandomBernoulli()+,
  \verb+RandomBinomial+, \verb+RandomDiscrete()+,
  \verb+RandomExponential()+, \verb+RandomLogistic()+,
  \verb+RandomLogNormal()+, \verb+RandomNormal()+, \verb+RandomizeTimer()+,
  \verb+SeedRNG()+
\end{desc}

\rl



\begin{desc}{Name/Symbol}
\item[Name/Symbol]	\verb+Rank()+

\item[Description]	Returns a list of numbers describing the rank of
  each position, from min to max.  The same as calling \verb+Order(Order(x))+.

\item[Usage]
\begin{verbatim}
		Rank(<list-of-numbers>)
\end{verbatim}

\item[Example]	
\begin{verbatim}
	n <- [33,12,1,5,9]
  	o <- Rank(n)
    Print(o) #should print [5,4,1,2,3]
\end{verbatim}

\item[See Also]	\verb+Order()+
\end{desc}

\rl



\begin{desc}{Name/Symbol}
\item[Name/Symbol]	\verb+Rectangle()+
  
\item[Description]	Creates a rectangle for graphing at x,y with size
  dx and dy. Rectangles are only currently definable oriented in
  horizontal/vertical directions.  A rectangle  must be added
  to a parent widget before it can be drawn; it may be added to
  widgets other than a base window.  The properties of rectangles may be
  changed by accessing their properties directly, including the FILLED
  property which makes the object an outline versus a filled shape.

\item[Usage]
\begin{verbatim}
Rectangle(<x>, <y>, <dx>, <dy>, <color>)
\end{verbatim}

\item[Example]	
\begin{verbatim}
  
  r <- Rectangle(30,30,20,10, MakeColor(green))
  AddObject(r, win)
  Draw()

\end{verbatim}
\item[See Also]	 \verb+Circle()+, \verb+Ellipse()+, \verb+Square()+, \verb+ Line()+
\end{desc}

\rl


\begin{desc}{Name/Symbol}
\item[Name/Symbol]	\verb+ReflectPoints+

\item[Description]  Takes a set of points (defined in a joined list 
[[x1,x2,x3,...],[y1,y2,y3,...]] and reflects them around the vertical
axis x=0, returning a similar [[x],[y]] list.  Identical to
\verb+ZoomPoints(pts,-1,1)+

\item[Usage]
\begin{verbatim}
  ReflectPoints(<points>)
\end{verbatim}

\item[Example] 
\begin{verbatim}
  points <- [[1,2,3,4],[20,21,22,23]]
  newpoints <- ReflectPoints(points)
\end{verbatim}

\item[See Also] \verb+ZoomPoints()+, \verb+RotatePoints+
\end{desc}

\rl



\begin{desc}{Name/Symbol}
\item[Name/Symbol]	\verb+RemoveObject()+

\item[Description] Removes a child widget from a parent.  Useful if
  you are adding a local widget to a global window inside a loop.  If
  you do not remove the object and only \verb+Hide()+ it, drawing will
  be sluggish.  Objects that are local to a function are removed
  automatically when the function terminates, so you do not need to
  call \verb+RemoveObject()+ on them at the end of a function.

\item[Usage]
\begin{verbatim}
RemoveObject( <object>, <parent>)
\end{verbatim}

\item[Example]	

\item[See Also]	
\end{desc}

\rl


\begin{desc}{Name/Symbol}
\item[Name/Symbol] 	\verb+Repeat()+

\item[Description] 	Makes and returns a list by repeating \verb+<object>+ \verb+<n>+ times. 
		Has no effect on the object. Repeat will not make new copies 
		of the object. If you later change the object, 
		you will change every object in the list.

\item[Usage]       	
\begin{verbatim}
Repeat(<object>, <n>)
\end{verbatim}
	    	
\item[Example]     	
\begin{verbatim}
x <- "potato"
y <- repeat(x, 10)
Print(y)
# produces ["potato","potato","potato","potato","potato", 
#           "potato","potato","potato","potato","potato"]
\end{verbatim}
	     	     
\item[See Also]    	\verb+RepeatList()+
\end{desc}

\rl


\begin{desc}{Name/Symbol}
\item[Name/Symbol] 	\verb+RepeatList()+

\item[Description]  	Makes a longer list by repeating a shorter list \verb+<n>+ times. 
	Has no effect on the list itself, but changes made to objects 
	in the new list will also affect the old list.

\item[Usage]       	
\begin{verbatim}
RepeatList(<list>, <n>)
\end{verbatim}

\item[Example]     	
\begin{verbatim}
RepeatList([1,2],3) # == [1,2,1,2,1,2]
\end{verbatim}

\item[See Also]    	\verb+Repeat()+, \verb+Merge()+, \verb+[ ]+
\end{desc}

\rl



\begin{desc}{Name/Symbol}
\item[Name/Symbol] \verb+Replace()+

\item[Description]  	Creates a copy of a (possibly nested) list in which
		items matching some list are replaced for other items.  
		\verb+<template>+ can be any data structure, and can be nested.  
		\verb+<replacementList>+ is a list containing two-item list pairs:
		the to-be-replaced item and to what it should be transformed.\\
		Note: replacement searches the entire \verb+<replacementList>+ for 
		matches.  If multiple keys are identical, the item will be 
		replaced with the last item that matches.

\item[Usage]        	
\begin{verbatim}
Replace(<template>,<replacementList>)
\end{verbatim}
			  
\item[Example]     	
\begin{verbatim}

x <- ["a","b","c","x"]
rep <- [["a","A"],["b","B"],["x","D"]]
Print(Replace(x,rep))
# Result:  [A, B, c, D] 
\end{verbatim}

\item[See Also]	
\verb+ReplaceChar()+
\end{desc}

\rl


\begin{desc}{Name/Symbol}
\item[Name/Symbol]  \verb+ReplaceChar()+

\item[Description]  	Substitutes  \verb+<char2>+ for \verb+<char>+
  in \verb+<string>+. Useful for saving subject entry data in a file;
  replacing spaces with some other character.

\item[Usage]        	
\begin{verbatim}
ReplaceChar(<string>,<char>,<char2>)
\end{verbatim}
			  
\item[Example]     	
\begin{verbatim}

x <- ["Sing a song of sixpence"]
rep <- ReplaceChar(x," ", "_")
Print(rep)
# Result:  Sing_a_song_of_sixpence
\end{verbatim}

\item[See Also]	
 for list items: \verb+Replace()+ 
\end{desc}

\rl


\begin{desc}{Name/Symbol}
\item[Name/Symbol] 	\verb+return+

\item[Description]  	Enables a function to return a value.

\item[Usage]
\begin{verbatim}
define funcname()
{
 return 0
}
\end{verbatim}

\item[Example]	

\item[See Also]	
\end{desc}

\rl


\begin{desc}{Name/Symbol}
\item[Name/Symbol]	\verb+Rotate()+

\item[Description] 	Returns a list created by rotating a list by \verb!<n>! items.  
		The new list will begin with the \verb!<n+1>!th item of the old 
		list (modulo its length), and contain all of its items in 
		order, jumping back to the beginning and ending with the \verb!<n>!th
		item. Rotate(\verb!<list>!,0) has no effect.  Rotate does not modify 
		the original list.

\item[Usage]
\begin{verbatim}
Rotate(<list-of-items>, <n>)
\end{verbatim}

\item[Example]     	
\begin{verbatim}
Rotate([1,11,111],1)  # == [11,111,1]
\end{verbatim}

\item[See Also]    	\verb+Transpose()+
\end{desc}

\rl



\begin{desc}{Name/Symbol}
\item[Name/Symbol]	\verb+RotatePoints+

\item[Description]  Takes a set of points (defined in a joined list 
[[x1,x2,x3,...],[y1,y2,y3,...]] and rotates them verb+<angle>+ degrees
around the point [0,0],  returning a similar [[x],[y]] list.

\item[Usage]
\begin{verbatim}
  ZoomPoints(<points>,<angle>)
\end{verbatim}

\item[Example] 
\begin{verbatim}
  points <- [[1,2,3,4],[20,21,22,23]]
  newpoints <- RotatePoints(points,10)
\end{verbatim}

\item[See Also] \verb+ZoomPoints()+, \verb+ReflectPoints+
\end{desc}

\rl



\begin{desc}{Name/Symbol}
\item[Name/Symbol] 	\verb+Round()+

\item[Description] 	Rounds \verb+<num>+ to nearest integer.

\item[Usage]        	
\begin{verbatim}
Round(<num>)
\end{verbatim}

\item[Example]
\begin{verbatim}
Round(33.23)  # == 33
Round(56.65)  # == 57
\end{verbatim}

\item[See Also]     	\verb+Ceiling()+, \verb+Floor()+, \verb+AbsFloor()+, \verb+ToInt()+
\end{desc}

\rl

%%% Local Variables: 
%%% mode: latex
%%% TeX-master: "main"
%%% End: 
\section{S}
\rl

\begin{desc}{Name/Symbol}
\item[Name/Symbol]  	\verb+SampleN()+

\item[Description] Samples \verb+<number>+ items from list, returning
  a randomly- ordered list. Items are sampled without replacement, so
  once an item is chosen it will not be chosen again. If
  \verb+<number>+ is larger than the length of the list, the entire
  list is returned shuffled.  It differs from \verb+ChooseN+ in that
  \verb+ChooseN+ returns items in the order they appeared in the
  originial list.  It is implemented as \verb+Shuffle(ChooseN())+.

\item[Usage]       	
\begin{verbatim}
SampleN(<list>, <n>)
\end{verbatim}

\item[Example]   	
\begin{verbatim}
SampleN([1,1,1,2,2], 5)     # Returns 5 numbers
SampleN([1,2,3,4,5,6,7], 3) # Returns 3 numbers from 1 and 7
\end{verbatim}

\item[See Also]    	\verb+ChooseN()+, \verb+SampleNWithReplacement()+, \verb+Subset()+
\end{desc}

\rl


\begin{desc}{Name/Symbol}
\item[Name/Symbol] 	\verb+SampleNWithReplacement()+

\item[Description] \verb+SampleNWithReplacement+ samples
  \verb+<number>+ items from \verb+<list>+, replacing after each draw
  so that items can be sampled again.  \verb+<number>+ can be larger
  than the length of the list. It has no side effects on its
  arguments.  
\item[Usage]        	
\begin{verbatim}
SampleNWithReplacement(<list>, <number>)
\end{verbatim}

\item[Example] 	
\begin{verbatim}
x <- Sequence(1:100,1)
SampleNWithReplacement(x, 10)
# Produces 10 numbers between 1 and 100, possibly 
# repeating some.
\end{verbatim}

\item[See Also]     	\verb+SampleN()+, \verb+ChooseN()+, \verb+Subset()+
\end{desc}

\rl


\begin{desc}{Name/Symbol}
\item[Name/Symbol] 	\verb+SDTBeta()+

\item[Description] \verb+SDTBeta+ computes beta, as defined by signal detection theory.  

\item[Usage]        	
\begin{verbatim}
SDTBeta(<hr>, <far>)
\end{verbatim}

\item[Example] 	
\begin{verbatim}

  Print(SDTBeta(.1,.9))  #.67032
  Print(SDTBeta(.1,.5))  #.88692
  Print(SDTBeta(.5,.5))  #1
  Print(SDTBeta(.8,.9))  #0.918612
  Print(SDTBeta(.9,.95)) #0.954803
\end{verbatim}

\item[See Also]\verb+SDTDPrime()+,
\end{desc}

\rl

\begin{desc}{Name/Symbol}
\item[Name/Symbol] 	\verb+SDTDPrime()+

\item[Description] \verb+SDTDPrime+ computes d-prime, as defined by
signal detection theory.  This is a measure of sensitivy based jointly
on hit rate and false alarm rate.

\item[Usage]        	
\begin{verbatim}
SDTDPrime(<hr>, <far>)
\end{verbatim}

\item[Example] 	
\begin{verbatim}

  Print(SDTDPrime(.1,.9))  #2.56431
  Print(SDTDPrime(.1,.5))  #1.28155
  Print(SDTDPrime(.5,.5))  #0
  Print(SDTDPrime(.8,.9))  #.43993
  Print(SDTDPrime(.9,.95)) #.363302

\end{verbatim}

\item[See Also]\verb+SDTBeta()+,
\end{desc}

\rl




\begin{desc}{Name/Symbol}
\item[Name/Symbol]   \verb+SeedRNG()+

\item[Description] Seeds the random number generator with \verb+<num>+
  to reproduce a random sequence.  This function can be used cleverly
  to create a multi-session experiment: Start by seeding the RNG with
  a single number for each subject; generate the stimulus sequence,
  then extract the appropriate stimuli for the current block. Remember
  to \verb+RandomizeTimer()+ afterward if necessary.

\item[Usage] 
\begin{verbatim}
SeedRNG(<num>) 
\end{verbatim}

\item[Example]	

\begin{verbatim}
    ##This makes sure you get the same random order across sessions
    ## for individual subjects.
     SeedRNG(gSubNum)
     stimTmp <- Sequence(1:100,1)
     stim <- Shuffle(stimTmp)
     RandomizeTimer()
\end{verbatim}

\item[See Also]	
     \verb+RandomizeTimer()+
\end{desc}

\rl

\begin{desc}{Name/Symbol}
\item[Name/Symbol]	\verb+SendData()+

\item[Description]	Sends data on network connection.  Example of
  usage in demo/nim.pbl. You can only send text data.

\item[Usage]
\begin{verbatim}
 SendData(<network>,<data_as_string>)
\end{verbatim}

\item[Example]	

On 'server':
\begin{verbatim}
  net <- WaitForNetworkConnection("localhost",1234)
  SendData(net,"Watson, come here. I need you.")
  CloseNetworkConnection(net)
\end{verbatim}
On Client:
\begin{verbatim}
  net <- ConnectToHost("localhost",1234)
  value <-  GetData(net,20)
  Print(value)
  CloseNetworkConnection(net)
##should print out "Watson, come here. I need you."
\end{verbatim}
\item[See Also]
  \verb+ConnectToIP+, \verb+ConnectToHost+, \verb+WaitForNetworkConnection+, \verb+GetData+, \verb+ConvertIPString+, \verb+CloseNetworkConnection+
\end{desc}

\rl


\begin{desc}{Name/Symbol}
\item[Name/Symbol]   	\verb+Sequence()+

\item[Description] Makes a sequence of numbers from \verb+<start>+ to
  \verb+<end>+ at \verb+<step>+-sized increments. If \verb+<step>+ is
  positive, \verb+<end>+ must be larger than \verb+<start>+, and if
  \verb+<step>+ is negative, \verb+<end>+ must be smaller than
  \verb+<start>+. If \verb!<start> + n*<step>! does not exactly equal
  \verb+<end>+, the last item in the sequence will be the number
  closest number to \verb+<end>+ in the direction of \verb+<start>+
  (and thus \verb+<step>+).

\item[Usage] 
\begin{verbatim}
Sequence(<start>, <end>, <step>)
\end{verbatim}

\item[Example]
\begin{verbatim}
Sequence(0,10,3)    # == [0,3,6,9]
Sequence(0,10,1.5)  # == [0,1.5,3,4.5, 6, 7.5, 9]
Sequence(10,1,3)    # error
Sequence(10,0,-1)   # == [10,9,8,7,6,5,4,3,2,1]
\end{verbatim}

\item[See Also]    	\verb+Repeat()+, \verb+RepeatList()+
\end{desc}

\rl


\begin{desc}{Name/Symbol}
\item[Name/Symbol] 	\verb+SetCursorPosition()+

\item[Description] 	Moves the editing cursor to a specified character
		position in a textbox.

\item[Usage]
\begin{verbatim}
SetCursorPosition(<textbox>, <integer>)
\end{verbatim}

\item[Example]
\begin{verbatim}
SetCursorPosition(tb, 23)
\end{verbatim}

\item[See Also]   	\verb+SetEditable()+, \verb+GetCursorPosition()+, \verb+SetText()+, \verb+GetText()+
\end{desc}

\rl


\begin{desc}{Name/Symbol}
\item[Name/Symbol] 	\verb+SetEditable()+

\item[Description] Sets the ``editable'' status of the textbox.  All
  this really does is turns on or off the cursor; editing must be done
  with the (currently unsupported) device function \verb+GetInput()+.

\item[Usage] 
\begin{verbatim}
SetEditable()
\end{verbatim}

\item[Example]
\begin{verbatim}

SetEditable(tb, 0)
SetEditable(tb, 1)
\end{verbatim}

\item[See Also]    	\verb+GetEditable()+
\end{desc}

\rl


\begin{desc}{Name/Symbol}
\item[Name/Symbol] 	\verb+SetFont()+

\item[Description] Resets the font of a textbox or label.  Change will
  not appear until the next \verb+Draw()+ function is called.  Can be
  used, for example, to change the color of a label to give richer
  feedback about correctness on a trial (see example below).  Font can
alse be set by assigning to the object.font property of an object.

\item[Usage]
\begin{verbatim}
SetFont(<text-widget>, <font>)
\end{verbatim}

\item[Example]   	
\begin{verbatim}
fontGreen <- MakeFont("vera.ttf",1,22,MakeColor("green"),
MakeColor("black"), 1)
fontRed   <- MakeFont("vera.ttf",1,22,MakeColor("red"),
MakeColor("black"), 1)
label <- MakeLabel(fontGreen, "Correct")

#Do trial here.       	

if(response == 1)
{
SetText(label, "CORRECT")
SetFont(label, fontGreen)
} else {
SetText(label, "INCORRECT")
SetFont(label, "fontRed)
}
Draw()
\end{verbatim}

\item[See Also]    	\verb+SetText()+
\end{desc}

\rl


\begin{desc}{Name/Symbol}
\item[Name/Symbol]	\verb+SetMouseCursorPosition()+

\item[Description] Sets the current x,y coordinates of the mouse
  pointer, 'warping' the mouse to that location immediately

\item[Usage]
\begin{verbatim}
   SetMouseCursorPosition(<x>,<y>)
\end{verbatim}

\item[Example]	
\begin{verbatim}

  ##Set mouse to center of screen:
  SetMouseCursorPosition(gVideoWidth/2,gVideoHeight/2)
\end{verbatim}


\item[See Also]
  \verb+ShowCursor+, \verb+WaitForMouseButton+,
  \verb+SetMouseCursorPosition+, \verb+GetMouseCursorPosition+
\end{desc}

\rl

\begin{desc}{Name/Symbol}

\item[Name/Symbol] 	\verb+SetText()+

\item[Description] 	Resets the text of a textbox or label.  Change will not
		appear until the next \verb+Draw()+ function is called.  The
object.text property can also be used to change text of an object, by
doing: \verb+object.text <- "new text"+

\item[Usage]
\begin{verbatim}
SetText(<text-widget>, <text>)
\end{verbatim}

\item[Example]
\begin{verbatim}
# Fixation Cross:
label <- MakeLabel(font, "+")
Draw()

SetText(label, "X")
Wait(100)
Draw()
\end{verbatim}

\item[See Also]    	\verb+GetText()+, \verb+SetFont()+
\end{desc}

\rl


\begin{desc}{Name/Symbol}
\item[Name/Symbol]  	\verb+Show()+

\item[Description] Sets a widget to visible, once it has been added to
  a parent widget.  This just changes the visibility property, it does
  not make the widget appear.  The widget will not be displayed until
  the \verb+Draw()+ function is called.  The .visible property of
objects can also be used to hide or show the object.

\item[Usage]
\begin{verbatim}
Show(<object>)
\end{verbatim}

\item[Example]
\begin{verbatim}
window <- MakeWindow()
image1  <- MakeImage("pebl.bmp")
image2  <- MakeImage("pebl.bmp")
AddObject(image1, window)
AddObject(image2, window)
Hide(image2)
Draw()
Wait(300)
Show(image2)
Draw()
\end{verbatim}

\item[See Also]     	\verb+Hide()+
\end{desc}

\rl




\begin{desc}{Name/Symbol}
\item[Name/Symbol]  	\verb+ShowCursor()+

\item[Description] Hides or shows the mouse cursor.  Currently, the
  mouse is not used, but on some systems in some configurations, the
  mouse cursor shows up.  Calling \verb+ShowCursor(0)+ will turn off the
  cursor, and \verb+ShowCursor(1)+ will turn it back on.  Be sure to turn it
  on at the end of the experiment, or you may actually lose the cursor
  for good.

\item[Usage]
\begin{verbatim}
ShowCursor(<value>)
\end{verbatim}

\item[Example]
\begin{verbatim}
window <- MakeWindow()
ShowCursor(0)
## Do experiment here
##

## Turn mouse back on.
ShowCursor(1)
\end{verbatim}
\item[See Also] 
\end{desc}

\rl




\begin{desc}{Name/Symbol}
\item[Name/Symbol] 	\verb+Shuffle()+

\item[Description] 	Randomly shuffles a list.

\item[Usage]    
\begin{verbatim}
Shuffle(list)
\end{verbatim}

\item[Example]
\begin{verbatim}
Print(Shuffle([1,2,3,4,5]))
# Results might be anything, like [5,3,2,1,4]
\end{verbatim}

\item[See Also]    	\verb+Sort()+, \verb+SortBy()+, \verb+ShuffleRepeat()+,
                    \verb+ShuffleWithoutAdjacents()+
\end{desc}

\rl




\begin{desc}{Name/Symbol}
\item[Name/Symbol] 	\verb+ShuffleRepeat()+

\item[Description] 	Randomly shuffles  \verb+<list>+, repeating \verb+<n>+ times.  Shuffles
  each iteration of the list separately, so you are guaranteed to go
  through all elements of the list before you get another.

\item[Usage]    
\begin{verbatim}
ShuffleRepeat(<list>, <n>)
\end{verbatim}

\item[Example]
\begin{verbatim}
Print(ShuffleRepeat([1,2,3,4,5]),3)
##  Results might be anything, like:
## [5,3,2,1,4, 3,2,5,1,4, 1,4,5,3,2]
\end{verbatim}

\item[See Also]    	\verb+Sort()+, \verb+SortBy()+, \verb+ShuffleRepeat()+,
                    \verb+ShuffleWithoutAdjacents()+
\end{desc}

\rl



\begin{desc}{Name/Symbol}
\item[Name/Symbol] 	\verb+ShuffleWithoutAdjacents()+

\item[Description] 	Randomly shuffles  <nested-list>, attempting to
  create a list where the nested elements do not appear adjacently in
  the new list. Returns a list that is flattened one level. It will
  always return a shuffled list, but it is not guaranteed to return
  one that has the non-adjecent structure specified, because this is
  sometimes impossible or very difficult to do randomly.  Given small
  enough non-adjacent constraints with enough fillers, it should be
  able to find something satisfactory.

\item[Usage]    
\begin{verbatim}
ShuffleWithoutAdjacents(<nested-list>)
\end{verbatim}

\item[Example]

\begin{verbatim}
Print(ShuffleWithoutAdjacents([[1,2,3], [4,5,6], [7,8,9]])
## Example Output: 
## [8, 5, 2, 7, 4, 1, 6, 9, 3]
## [7, 4, 8, 1, 9, 2, 5, 3, 6]

## Non-nested items are shuffled without constraint
Print(ShuffleWithoutAdjacents([[1,2,3], 11,12,13,14,15,16]))
## output: [13, 11, 2, 14, 3, 15, 1, 16, 12]
##         [13, 12, 2, 16, 15, 11, 1, 14, 3]
##         [11, 1, 15, 2, 12, 16, 14, 13, 3]

## Sometimes the constraints cannot be satisfied.  9 will always
## appear in position 2
Print(ShuffleWithoutAdjacents([[1,2,3], 9])
## output: [3, 9, 1, 2]
##         [2, 9, 3, 1]
##         [3, 9, 2, 1]
\end{verbatim}

\item[See Also]    	\verb+Shuffle()+, \verb+Sort()+, \verb+SortBy()+,
        \verb+ShuffleRepeat()+, \verb+ShuffleWithoutAdjacents()+
\end{desc}

\rl


\begin{desc}{Name/Symbol}
\item[Name/Symbol] 	\verb+Sign()+

\item[Description] 	Returns +1 or -1, depending on sign of argument.

\item[Usage]
\begin{verbatim}
Sign(<num>)
\end{verbatim}

\item[Example]
\begin{verbatim}
Sign(-332.1)  # == -1
Sign(65)      # == 1

\end{verbatim}

\item[See Also]     	\verb+Abs()+
\end{desc}

\rl


\begin{desc}{Name/Symbol}
\item[Name/Symbol] 	\verb+SignalFatalError()+

\item[Description] Stops PEBL and prints \verb+<message>+ to stderr.
  Useful for type-checking in user-defined functions.

\item[Usage]
\begin{verbatim}
SignalFatalError(<message>)
\end{verbatim}
\item[Example]
\begin{verbatim}

If(not IsList(x))
{
 SignalFatalError("Tried to frobnicate a List.")
}
##Prints out error message and line/filename of function
\end{verbatim}

\item[See Also]     	\verb+Print()+
\end{desc}

\rl


\begin{desc}{Name/Symbol}
\item[Name/Symbol]  	\verb+Sin()+

\item[Description]  	Sine of \verb+<deg>+ degrees.

\item[Usage]        	
\begin{verbatim}
Sin(<deg>)
\end{verbatim}
\item[Example]
\begin{verbatim}
 Sin(180)
 Sin(0)
\end{verbatim}
\item[See Also]    	\verb+Cos()+, \verb+Tan()+, \verb+ATan()+, \verb+ACos()+, \verb+ATan()+ 
\end{desc}

\rl


\begin{desc}{Name/Symbol}
\item[Name/Symbol] 	\verb+Sort()+

\item[Description] 	Sorts a list by its values from smallest to largest.

\item[Usage]       	
\begin{verbatim}
Sort(<list>)
\end{verbatim}

\item[Example]
\begin{verbatim}
Sort([3,4,2,1,5]) # == [1,2,3,4,5]
\end{verbatim}

\item[See Also]    	\verb+SortBy()+, \verb+Shuffle()+
\end{desc}

\rl


\begin{desc}{Name/Symbol}
\item[Name/Symbol] 	\verb+SortBy()+

\item[Description] 	Sorts a list by the values in another list, in ascending
		order.

\item[Usage]
\begin{verbatim}
SortBy(<value-list>, <key-list>)
\end{verbatim}

\item[Example]
\begin{verbatim}
SortBy(["Bobby","Greg","Peter"], [3,1,2]) 
# == ["Greg","Peter","Bobby"]
\end{verbatim}

\item[See Also]    	\verb+Shuffle()+, \verb+Sort()+
\end{desc}

\rl


\begin{desc}{Name/Symbol}
\item[Name/Symbol]  	\verb+SplitString()+

\item[Description]	Splits a string into tokens. \verb+<split>+ must be a string. If 
		\verb+<split>+ is not found in \verb+<string>+, a list containing the entire 
		string is returned; if split is equal to \verb+""+, the each letter 
		in the string is placed into a different item in the list.  
		Multiple delimiters, as well as delimiters at the beginning 
		and end of a list, will produce empty list items. 


\item[Usage]
\begin{verbatim}
SplitString(<string>, <split>)
\end{verbatim}

\item[Example]      	
\begin{verbatim}
SplitString("Everybody Loves a Clown", " ") 
# Produces ["Everybody", "Loves", "a", "Clown"]
\end{verbatim}

\item[See Also]     	\verb+FindInString()+
\end{desc}

\rl




\begin{desc}{Name/Symbol}
\item[Name/Symbol]	\verb+Square()+
  
\item[Description]	Creates a square for graphing at x,y with size
  \verb+<size>+. Squares are only currently definable oriented in
  horizontal/vertical directions.  A square  must be added
  to a parent widget before it can be drawn; it may be added to
  widgets other than a base window.  The properties of squares may be
  changed by accessing their properties directly, including the FILLED
  property which makes the object an outline versus a filled shape.

\item[Usage]
\begin{verbatim}
Ellipse(<x>, <y>, <size>, <color>)
\end{verbatim}

\item[Example]	
\begin{verbatim}
  
  s <- Square(30,30,20, MakeColor(green))
  AddObject(s, win)
  Draw()

\end{verbatim}
\item[See Also]	 \verb+Circle()+, \verb+Ellipse()+, \verb+Rectangle()+, \verb+Line()+
\end{desc}

\rl



\begin{desc}{Name/Symbol}
\item[Name/Symbol]  	\verb+Sqrt()+ 

\item[Description]  	Square root of \verb+<num>+.

\item[Usage]        	
\begin{verbatim}
Sqrt(<num>)
\end{verbatim}

\item[Example]
\begin{verbatim}
Sqrt(100)  # == 10
\end{verbatim}

\item[See Also]	
\end{desc}

\rl


\begin{desc}{Name/Symbol}
\item[Name/Symbol]  	\verb+StDev()+ 

\item[Description]  Returns the standard deviation of \verb+<list>+.

\item[Usage]       	
\begin{verbatim}
StDev(<list>)        
\end{verbatim}

\item[Example]	
\begin{verbatim}
   sd <- StDev([3,5,99,12,1.3,15])        
\end{verbatim}

\item[See Also]     	\verb+Min()+, \verb+Max()+, \verb+Mean()+, \verb+Median()+, \verb+Quantile()+, \verb+Sum()+
\end{desc}

\rl




\begin{desc}{Name/Symbol}
\item[Name/Symbol]  	\verb+Stop()+	

\item[Description] Stops a sound playing in the background from
  playing.  Calling \verb+Stop()+ on a sound object that is not
  playing should have no effect, but if an object is aliased,
  \verb+Stop()+ will stop the file.  Note that sounds play in a
  separate thread, so interrupting the thread has a granularity up to
  the duration of the thread-switching quantum on your computer; this
  may be tens of milliseconds.

\item[Usage]
\begin{verbatim}
Stop(<sound-object>)
\end{verbatim}

\item[Example]     	
\begin{verbatim}
buzz <- LoadSound("buzz.wav")
PlayBackground(buzz)
Wait(50)
Stop(buzz)
\end{verbatim}

\item[See Also]    	\verb+PlayForeground()+, \verb+PlayBackGround()+
\end{desc}

\rl


\begin{desc}{Name/Symbol}
\item[Name/Symbol]  	\verb+StringLength()+

\item[Description] 	Determines the length of a string, in characters.

\item[Usage]
\begin{verbatim}
StringLength(<string>)
\end{verbatim}

\item[Example]     	
\begin{verbatim}
StringLength("absolute")     # == 8
StringLength("   spaces   ") # == 12
StringLength("")             # == 0
\end{verbatim}

\item[See Also]    	\verb+Length()+, \verb+SubString()+
\end{desc}

\rl


\begin{desc}{Name/Symbol}
\item[Name/Symbol]  	\verb+SubList()+

\item[Description] 	Extracts a list from another list, by specifying 
	     	beginning and end points of new sublist.

\item[Usage]
\begin{verbatim}
SubList(<list>, <begin>, <end>)
\end{verbatim}

\item[Example]     	
\begin{verbatim}
SubList([1,2,3,4,5,6],3,5)	# == [3,4,5]
\end{verbatim}

\item[See Also]    	\verb+SubSet()+, \verb+ExtractListItems()+
\end{desc}

\rl


\begin{desc}{Name/Symbol}
\item[Name/Symbol]  	\verb+Subset()+

\item[Description] Extracts a subset of items from another list,
  returning a new list that includes items from the original list only
  once and in their original orders.  Item indices in the second
  argument that do not exist in the first argument are ignored.  It
  has no side effects on its arguments.  

\item[Usage]       	
\begin{verbatim}
Subset(<list>, <list-of-indices>)
\end{verbatim}

\item[Example]     	
\begin{verbatim}
Subset([1,2,3,4,5,6],[5,3,1,1])	# == [1,3,5]
Subset([1,2,3,4,5], [23,4,2])		# == [2,4]
\end{verbatim}

\item[See Also]   	\verb+SubList()+, \verb+ExtractItems()+, \verb+SampleN()+
\end{desc}

\rl


\begin{desc}{Name/Symbol}
\item[Name/Symbol]  	\verb+SubString()+

\item[Description]  	Extracts a substring from a longer string.

\item[Usage]
\begin{verbatim}
SubString(<string>,<position>,<length>)
\end{verbatim}
  If position is larger than the length of the string, an empty string
  is returned.  If position + length exceeds the length of the string,
  a string from \verb+<position>+ to the last character of the string
  is returned.

\item[Example]
\begin{verbatim}
SubString("abcdefghijklmnop",3,5)	# == "cdefg"
\end{verbatim}

\item[See Also]	
\end{desc}

\rl





\begin{desc}{Name/Symbol}
\item[Name/Symbol]  	\verb+Sum()+ 

\item[Description]  Returns the sum  of \verb+<list>+.

\item[Usage]       	
\begin{verbatim}
Sum(<list>)        
\end{verbatim}

\item[Example]	
\begin{verbatim}
   sum <- StDev([3,5,99,12,1.3,15])      # == 135.3
\end{verbatim}

\item[See Also]     	\verb+Min()+, \verb+Max()+, \verb+Mean()+, \verb+Median()+, \verb+Quantile()+, \verb+StDev()+
\end{desc}

\rl



\begin{desc}{Name/Symbol}
\item[Name/Symbol]  	\verb+SummaryStats()+

\item[Description] Computes summary statistics for a data list,
  aggregated by labels in a condition list.
For each condition (distinct label in the \verb+<cond>+ list), it will 
return a list with the following entries:
\verb+<cond>+ \verb+<N>+ \verb+<median>+ \verb+<mean>+ \verb+<sd>+

\item[Usage]		
\begin{verbatim}
SummaryStats(<data>,<cond>)        
\end{verbatim}

\item[Example]	
\begin{verbatim}
  dat <- [1.1,1.2,1.3,2.1,2.2,2.3]
  cond <- [1,1,1,2,2,2]
  Print(SummaryStats(dat,cond))
\end{verbatim}
Result:
[[1, 3, 1.1, 1.2, 0.0816497]
, [2, 3, 2.1, 2.2, 0.0816497]
]
\item[See Also]	
  	\verb+StDev()+, \verb+Min()+, \verb+Max()+, \verb+Mean()+, \verb+Median()+, \verb+Quantile()+, \verb+Sum()+
\end{desc}
\rl




\section{T}
\rl


\begin{desc}{Name/Symbol}

\item[Name/Symbol] \verb+Tab()+

\item[Description]  Produces a tab character which can be added to a
  string. If displayed in a text box, it will use a 4-item tab stop.

\item[Usage]        \verb!Tab(3)!


\item[Example]     
\begin{verbatim}
         Print("Number: "  Tab(1) + number )
         Print("Value: "  Tab(1) + value )
         Print("Size: "  Tab(1) + size )
\end{verbatim}
\item[See Also]
\verb+Format()+, \verb+CR()+
\end{desc}

\rl     

\begin{desc}{Name/Symbol}
\item[Name/Symbol]  	\verb+Tan()+	

\item[Description] 	Tangent of \verb+<deg>+ degrees.

\item[Usage]       	
\begin{verbatim}
Tan(<deg>)
\end{verbatim}

\item[Example]
\begin{verbatim}
Tan(180)
\end{verbatim}

\item[See Also]    	\verb+Cos()+, \verb+Sin()+, \verb+ATan()+, \verb+ACos()+, \verb+ATan()+ 
\end{desc}

\rl


\begin{desc}{Name/Symbol}
\item[Name/Symbol]  	\verb+TimeStamp()+

\item[Description] Returns a string containing the date-and-time,
  formatted according to local conventions. Should be used for
  documenting the time-of-day and date an experiment was run, but not
  for keeping track of timing accuracy.  For that, use
  \verb+GetTime()+.
	     
\item[Usage]
\begin{verbatim}
TimeStamp()
\end{verbatim}

\item[Example]
\begin{verbatim}
a <- TimeStamp()
Print(a)
\end{verbatim}

\item[See Also]     	\verb+GetTime()+
\end{desc}

\rl


\begin{desc}{Name/Symbol}
\item[Name/Symbol]  	\verb+ToInteger()+
              
\item[Description]  	Rounds a number to an integer, changing internal 
		representation.

\item[Usage]
\begin{verbatim}
ToInteger(<number>)
ToInteger(<floating-point>)
ToInteger(<string-as-number>)
\end{verbatim}

\item[Example]
\begin{verbatim}
ToInteger(33.332)  # == 33
ToInteger("3213")  # == 3213
\end{verbatim}

\item[See Also]    	\verb+Round()+, \verb+Ceiling()+, \verb+AbsCeiling()+, \verb+Floor()+, \verb+AbsFloor()+
\end{desc}

\rl


\begin{desc}{Name/Symbol}
\item[Name/Symbol]  	\verb+ToFloat()+

\item[Description] 	Converts number to internal floating-point representation.

\item[Usage]
\begin{verbatim}
ToFloat(<number>)
\end{verbatim}

\item[Example]	

\item[See Also]	
\end{desc}

\rl


\begin{desc}{Name/Symbol}
\item[Name/Symbol]  	\verb+ToNumber()+

\item[Description] Converts a variant to a number. Most useful for
  character strings that are interpretable as a number, but may also
  work for other subtypes.

\item[Usage]     
\begin{verbatim}
ToNumber(<string)
ToNumber(<number>)
\end{verbatim}

\item[Example]
\begin{verbatim}
a <- ToNumber("3232")
Print(a + 1)		# produces the output 3233. 
\end{verbatim}

\item[See Also]     	\verb+ToString()+, \verb+ToFloat()+, \verb+Round()+
\end{desc}

\rl


\begin{desc}{Name/Symbol}
\item[Name/Symbol]  	\verb+ToString()+

\item[Description] Converts value to a string representation. Most
  useful for numerical values.  This conversion is done automatically
  when strings are combined with numbers.

\item[Usage]     
\begin{verbatim}
ToString(<number>)
ToString(<string>)
\end{verbatim}

\item[Example]
\begin{verbatim}
a <- ToString(333.232)
Print(a + "111")
# produces the output '333.232111'.
\end{verbatim}
		

\item[See Also] \verb+ToString()+, \verb|+|.
\end{desc}

\rl


\begin{desc}{Name/Symbol}
\item[Name/Symbol]  	\verb+TranslateKeyCode()+

\item[Description] Translates a code corresponding to a keyboard key
  into a keyboard value.  This code is returned by some event/device
  polling functions.

\item[Usage]		

\item[Example]	

\item[See Also]	
\end{desc}

\rl


\begin{desc}{Name/Symbol}
\item[Name/Symbol]  	\verb+Transpose()+

\item[Description] Transposes or ``rotates'' a list of lists.  Each
  sublist must be of the same length.

\item[Usage]       	
\begin{verbatim}
Transpose(<list-of-lists>)
\end{verbatim}

\item[Example]     	
\begin{verbatim}
Transpose([[1,11,111],[2,22,222],[3,33,333], [4,44,444]])
# == [[1,2,3,4],[11,22,33,44],[111,222,333,444]]
\end{verbatim}

\item[See Also]    	\verb+Rotate()+
\end{desc}

\rl
\section{U}
\rl


\begin{desc}{Name/Symbol}
\item[Name/Symbol]  	\verb+Uppercase()+

\item[Description]  	Changes a string to uppercase.  Useful for testing user
	      	input against a stored value, to ensure case differences
	      	are not detected.

\item[Usage]
\begin{verbatim}
Uppercase(<string>)
\end{verbatim}

\item[Example]     
\begin{verbatim}
Uppercase("POtaTo")  # == "POTATO"
\end{verbatim}

\item[See Also]     	\verb+Lowercase()+
\end{desc}

\rl
\section{W}
\rl


\begin{desc}{Name/Symbol}
\item[Name/Symbol]  	\verb+Wait()+ 

\item[Description] 	Waits the specified number of milliseconds, then returns. 

\item[Usage]
\begin{verbatim}
Wait(<time>)
\end{verbatim}

\item[Example]
\begin{verbatim}
Wait(100)
Wait(15)
\end{verbatim}

\item[See Also]	
\end{desc}

\rl


\begin{desc}{Name/Symbol}
\item[Name/Symbol]  	\verb+WaitForAllKeysUp()+

\item[Description]	
               Wait until all keyboard keys are in the up
               position. This includes numlock, capslock, etc.
\item[Usage]		

\item[Example]	

\item[See Also]	

\end{desc}

\rl


\begin{desc}{Name/Symbol}
\item[Name/Symbol]  	\verb+WaitForAnyKeyDown()+

\item[Description]	
             Waits for any key to be detected in the down position.
             This includes numlock, capslock, etc, which can be locked
             in the down position even if they are not being held
             down.  Will return immediately if a key is being held
             down before the function is called. 

\item[Usage]		

\item[Example]	

\item[See Also]	
            \verb+WaitForAnyKeyPress()+
\end{desc}

\rl


\begin{desc}{Name/Symbol}
\item[Name/Symbol]  	\verb+WaitForAnyKeyDownWithTimeout()+

\item[Description] Waits until any key is detected in the down position, but will return
  after a specified number of milliseconds.

\item[Usage]
\begin{verbatim}
WaitForAnyKeyDownWithTimeout(<time>)
\end{verbatim}

\item[Example]	

\item[See Also]	
\end{desc}

\rl


\begin{desc}{Name/Symbol}
\item[Name/Symbol]  	\verb+WaitForKeyDown()+

\item[Description]	

\item[Usage]		

\item[Example]	

\item[See Also]	
\end{desc}

\rl


\begin{desc}{Name/Symbol}
\item[Name/Symbol]  	\verb+WaitForKeyListDown()+

\item[Description] Returns when any one of the keys specified in the
  argument is down. If a key is down when called, it will return immediately.

\item[Usage]
\begin{verbatim}
WaitForKeyListDown(<list-of-keys>)
\end{verbatim}

\item[Example]     	
\begin{verbatim}
WaitForKeyListDown(["a","z"])
\end{verbatim}

\item[See Also]	
 \end{desc}

\rl





\begin{desc}{Name/Symbol}
\item[Name/Symbol]  	\verb+WaitForListKeyPressWithTimeout()+

\item[Description] Returns when any one of the keys specified in the
  argument is pressed, or when the timeout has elapsed; whichever
  comes first. Will only return on a new keyboard/timeout events, and
  so a previously pressed key will not trip this function, unlike
  \verb+WaitForKeyListDown()+.  The \verb+<style>+ parameter is currently
  unused, but may be deployed in the future for differences in how
  or when things should be returned.  Returns the value of the pressed
  key.  If the function terminates by exceeding the \verb+<timeout>+,
  it will return the string \verb+"<unknown>"+.

\item[Usage]
\begin{verbatim}
 WaitForListKeyPressWithTimeout(<list-of-keys>,
                                <timeout>,<style>)
\end{verbatim}

\verb+<list-of-keys>+ can include text versions of many keys.  See Chapter 4,
section ``Keyboard Entry'' for complete list of keynames.

\item[Example]     	
\begin{verbatim}
  x <- WaitForListKeyPressWithTimeout(["a","z"],2000,1)
  if(x == "<unknown>") 
  {
     Print("Did Not Respond.")
  }
\end{verbatim}

\item[See Also]	
   \verb+WaitForKeyListDown+, \verb+WaitForListKeyPress+
 \end{desc}

\rl





\begin{desc}{Name/Symbol}
\item[Name/Symbol]  	\verb+WaitForListKeyPress()+

\item[Description] Returns when any one of the keys specified in the
  argument is pressed. Will only return on a new keyboard event, and
  so a previously pressed key will not trip this function, unlike
  \verb+WaitForKeyListDown()+  Returns a string indicating the value
  of the keypress.

\item[Usage]
\begin{verbatim}
WaitForListKeyPress(<list-of-keys>)
\end{verbatim}

\item[Example]     	
\begin{verbatim}
WaitForListKeyPress(["a","z"])
\end{verbatim}

\item[See Also]	
   \verb+WaitForKeyListDown+, \verb+WaitForListKeyPressWithTimeout+
 \end{desc}

\rl


\begin{desc}{Name/Symbol}
\item[Name/Symbol]  	\verb+WaitForKeyPress()+

\item[Description] Waits for a keypress event that matches the
  specified key.  Usage of this function is preferred over
  \verb+WaitForKeyDown()+, which tests the state of the key. Returns the
  value of the key pressed.

\item[Usage]
\begin{verbatim}
WaitForKeyPress(<key>)
\end{verbatim}

\item[Example]	

\item[See Also]     	\verb+WaitForAnyKeyPress()+, \verb+WaitForKeyRelease()+, \verb+WaitForListKeyPress()+
\end{desc}

\rl


\begin{desc}{Name/Symbol}
\item[Name/Symbol] 	\verb+WaitForKeyUp()+

\item[Description]	

\item[Usage]		

\item[Example]	

\item[See Also]	
\end{desc}


\rl
\begin{desc}{Name/Symbol}
\item[Name/Symbol]	\verb+WaitForMouseButton()+

\item[Description] Waits for a mouse click event to occur.
  This takes no arguments, and returns a 4-tuple list, indicating:

  \verb+[xpos, ypos, button id [1-3], "<pressed>" or "<released>"]+

\item[Usage]
\begin{verbatim}
   WaitForMouseButton()
\end{verbatim}

\item[Example]	
\begin{verbatim}
## Here is how to wait for a mouse down-click

 continue <- 1
 while(continue)
 {
     x <- WaitForMouseButton()
     if(Nth(x,4)=="<pressed>")
      {
          continue <- 0
      }
 }
 Print("Clicked")
\end{verbatim}


\item[See Also]
  \verb+ShowCursor+,
  \verb+SetMouseCursorPosition+, \verb+GetMouseCursorPosition+
\end{desc}

\rl





\begin{desc}{Name/Symbol}
\item[Name/Symbol]	\verb+WaitForNetworkConnection()+

\item[Description]	Listens on a port, waiting until another computer or process
  connects. Return a network object that can be used for communication.

\item[Usage]
\begin{verbatim}
   WaitForNetworkConnection(<port>)
\end{verbatim}

\item[Example]	

  See nim.pbl for example of two-way network connection.
\begin{verbatim}

  net <- WaitForNetworkConnection(1234)
  dat <- GetData(net,20)
  Print(dat)
  CloseNetworkConnection(net)
\end{verbatim}

\item[See Also]
  \verb+ConnectToHost+, \verb+ConnectToIP+, \verb+GetData+, \verb+WaitForNetworkConnection+,
   \verb+SendData+, \verb+ConvertIPString+, \verb+CloseNetworkConnection+
\end{desc}

\rl


\begin{desc}{Name/Symbol}
\item[Name/Symbol]	\verb+while+

\item[Description] `while' is a keyword, and so is part of the syntax,
  not a function per se.  It executes the code inside the \verb+{}+
  brackets until the test inside the \verb+()+ executes as false.
  This can easily lead to an infinite loop if conditions are not met.
  Also, there is currently no break statement to allow execution to
  halt early.  Unlike some other languages, PEBL requires that the
  \verb+{}+ be present.

\item[Usage]
\begin{verbatim}

while(<test expression>)
{
 code line 1
 code line 2
}
\end{verbatim}

\item[Example] 
\begin{verbatim}
i <- 1
while(i <= 10)
{
 Print(i)
 i <- i + 1
}		# prints out the numbers 1 through 10
\end{verbatim}

\item[See Also] 	\verb+loop()+, \verb+{ }+


\end{desc}

\rl
\section{Z}
\rl


\begin{desc}{Name/Symbol}
\item[Name/Symbol]	\verb+ZoomPoints+

\item[Description]  Takes a set of points (defined in a joined list 
[[x1,x2,x3,...],[y1,y2,y3,...]] and adjusts them in the x and y
direction independently, returning a similar [[x],[y]] list.

Note: The original points should be centered at zero, because the get
adjusted relative to zero, not relative to their center.

\item[Usage]
\begin{verbatim}
  ZoomPoints(points,<xzoom>,<yzoom>)
\end{verbatim}

\item[Example] 
\begin{verbatim}
  points <- [[1,2,3,4],[20,21,22,23]]
  newpoints <- ZoomPoints(points,2,.5)
  ##Produces [[2,4,6,8],[10,11.5,11,11.5]]
\end{verbatim}

\item[See Also] \verb+RotatePoints()+, \verb+ReflectPoints+
\end{desc}

\rl

%%% Local Variables: 
%%% mode: latex
%%% TeX-master: "main"
%%% End: 
