\section{L}
\rl

\begin{desc}{Name/Symbol}
\item[Name/Symbol]	\verb+Last()+

\item[Description]	Returns the last item in a list. Provides faster 
		access to the last item of a list than does Nth().

\item[Usage]
\begin{verbatim}
Last(<list>)
\end{verbatim}

\item[Example]
\begin{verbatim}
Last([1,2,3,444])	# == 444
\end{verbatim}

\item[See Also]	\verb+Nth()+, \verb+First()+
\end{desc}

\rl




\begin{desc}{Name/Symbol}
\item[Name/Symbol]	\verb+Line()+

\item[Description] Creates a line for graphing at x,y ending at x+dx,
  y+dy.  dx and dy describe the size of the line.  Lines must be added
  to a parent widget before it can be drawn; it may be added to
  widgets other than a base window. Properties of lines may be
  accessed and set later.

\item[Usage]
\begin{verbatim}
Line(<x>, <y>, <dx>, <dy>, <color>)
\end{verbatim}

\item[Example]	
\begin{verbatim}
  l <- Line(30,30,20,20, MakeColor("green")
  AddObject(l, win)
  Draw()

\end{verbatim}
\item[See Also]	\verb+Square()+, \verb+Ellipse()+, \verb+Rectangle()+, \verb+Circle()+
\end{desc}

\rl

\begin{desc}{Name/Symbol}
\item[Name/Symbol]	\verb+List()+

\item[Description]	Creates a list of items. Functional version of \verb+[]+.

\item[Usage]
\begin{verbatim}
List(<item1>, <item2>, ....)
\end{verbatim}

\item[Example]
\begin{verbatim}
List(1,2,3,444)		# == [1,2,3,444]
\end{verbatim}

\item[See Also]	\verb+[ ]+, \verb+Merge()+, \verb+Append()+
\end{desc}

\rl




\begin{desc}{Name/Symbol}
\item[Name/Symbol]	\verb+Length()+

\item[Description]	Returns the number of items in a list.

\item[Usage]
\begin{verbatim}
Length(<list>)
\end{verbatim}

\item[Example]
\begin{verbatim}
Length([1,3,55,1515])	# == 4
\end{verbatim}

\item[See Also]	\verb+StringLength()+
\end{desc}

\rl



\begin{desc}{Name/Symbol}
\item[Name/Symbol]	\verb+LoadSound()+

\item[Description]	Loads a soundfile from \verb+<filename>+, 
		returning a variable that can be played.

\item[Usage]
\begin{verbatim}
LoadSound(<filename>)
\end{verbatim}

\item[Example]	

\item[See Also]	
\end{desc}

\rl



\begin{desc}{Name/Symbol}
\item[Name/Symbol]	\verb+Log10()+

\item[Description]	Log base 10 of \verb+<num>+.

\item[Usage]
\begin{verbatim}
Log10(<num>)
\end{verbatim}

\item[Example]	

\item[See Also]	\verb+Log2()+, \verb+LogN()+, \verb+Ln()+, \verb+Exp()+
\end{desc}

\rl


\begin{desc}{Name/Symbol}
\item[Name/Symbol]	\verb+Log2()+

\item[Description]	Log base 2 of \verb+<num>+.

\item[Usage]
\begin{verbatim}
Log2(<num>)
\end{verbatim}

\item[Example]	

\item[See Also]	\verb+Log()+, \verb+LogN()+, \verb+Ln()+, \verb+Exp()+
\end{desc}

\rl


\begin{desc}{Name/Symbol}
\item[Name/Symbol]	\verb+LogN()+

\item[Description]	Log base \verb+<base>+ of \verb+<num>+.

\item[Usage]
\begin{verbatim}
LogN(<num>, <base>)
\end{verbatim}

\item[Example]
\begin{verbatim}
LogN(100,10)	# == 2
LogN(256,2)	# == 8
\end{verbatim}

\item[See Also]	\verb+Log()+, \verb+Log2()+, \verb+Ln()+, \verb+Exp()+
\end{desc}

\rl


\begin{desc}{Name/Symbol}
\item[Name/Symbol]	\verb+Lowercase()+

\item[Description]	Changes a string to lowercase.  Useful for testing user
		input against a stored value, to ensure case differences
		are not detected.

\item[Usage]
\begin{verbatim}
Lowercase(<string>)
\end{verbatim}

\item[Example]
\begin{verbatim}
Lowercase("POtaTo")	# == "potato"
\end{verbatim}

\item[See Also]	\verb+Uppercase()+
\end{desc}

\rl


\begin{desc}{Name/Symbol}
\item[Name/Symbol]	\verb+Ln()+

\item[Description]	Natural log of \verb+<num>+.

\item[Usage]		
\begin{verbatim}
Ln(<num>)
\end{verbatim}

\item[Example]	

\item[See Also]	\verb+Log()+, \verb+Log2()+, \verb+LogN()+, \verb+Exp()+     
\end{desc}

\rl


\begin{desc}{Name/Symbol}
\item[Name/Symbol]	\verb+loop()+

\item[Description]	Loops over elements in a list.  During each iteration, <counter> is bound to each consecutive member of \verb+<list>+.

\item[Usage]		
\begin{verbatim}
loop(<counter>, <list>)
{
 statements
 to
 be	   
 executed
}
\end{verbatim}

\item[Example]	

\item[See Also]	\verb+while()+, \verb+{ }+
\end{desc}

\rl

\section{M}
\rl


\begin{desc}{Name/Symbol}
\item[Name/Symbol]	\verb+MakeChirp()+  

\item[Description]	NOT IMPLEMENTED.

\item[Usage]		

\item[Example]	

\item[See Also]	\verb+MakeSawtoothWave()+, \verb+MakeSineWave()+, \verb+MakeSquareWave()+
\end{desc}

\rl




\begin{desc}{Name/Symbol}
\item[Name/Symbol]	\verb+MakeColor()+

\item[Description] Makes a color from \verb+<colorname>+ such as
  ``red'', ``green'', and nearly 800 others.  Color names and
  corresponding RGB values can be found in \verb+doc/colors.txt+.

\item[Usage]
\begin{verbatim}
MakeColor(<colorname>)
\end{verbatim}

\item[Example]	

\item[See Also]	\verb+MakeColorRGB()+
\end{desc}

\rl


\begin{desc}{Name/Symbol}
\item[Name/Symbol]	\verb+MakeColorRGB()+ 

\item[Description] Makes an RGB color by specifying \verb+<red>+,
  \verb+<green>+, and \verb+<blue>+ values (between 0 and 255).

\item[Usage]		
\begin{verbatim}
MakeColorRGB(<red>, <green>, <blue>)
\end{verbatim}

\item[Example]	

\item[See Also]	\verb+MakeColor()+
\end{desc}

\rl





\begin{desc}{Name/Symbol}
\item[Name/Symbol]	\verb+MakeFont()+

\item[Description]	Makes a font.

\item[Usage]
\begin{verbatim}
MakeFont(<ttf_filename>, <style>, <size>, 
<fgcolor>, <bgcolor>, <anti-aliased>)
\end{verbatim}

\item[Example]	

\item[See Also]	
\end{desc}

\rl



\begin{desc}{Name/Symbol}
\item[Name/Symbol]	\verb+MakeImage()+

\item[Description]	Makes an image widget from an image file.
		\texttt{.bmp} formats should be supported; others may be as well.

\item[Usage]		
\begin{verbatim}

MakeImage(<filename>)
\end{verbatim}

\item[Example]	

\item[See Also]	
\end{desc}

\rl




\begin{desc}{Name/Symbol}
\item[Name/Symbol]	\verb+MakeLabel()+

\item[Description] Makes a text label for display on-screen. Text will
  be on a single line, and the \verb+Move()+ command centers
  \verb+<text>+ on the specified point.

\item[Usage]
\begin{verbatim}
MakeLabel(<text>, <font>)
\end{verbatim}

\item[Example]	

\item[See Also]	
\end{desc}

\rl




\begin{desc}{Name/Symbol}
\item[Name/Symbol]	\verb+MakeMap()+

\item[Description]	NOT IMPLEMENTED.

\item[Usage]		

\item[Example]	

\item[See Also]	
\end{desc}

\rl


\begin{desc}{Name/Symbol}
\item[Name/Symbol]	\verb+MakeSawtoothWave()+   

\item[Description]	NOT IMPLEMENTED.

\item[Usage]		

\item[Example]	

\item[See Also]	\verb+MakeSquareWave()+, \verb+MakeSineWave()+, \verb+MakeChirp()+
\end{desc}

\rl


\begin{desc}{Name/Symbol}
\item[Name/Symbol]	\verb+MakeSineWave()+    

\item[Description]	NOT IMPLEMENTED.

\item[Usage]		

\item[Example]	

\item[See Also]	\verb+MakeSquareWave()+, \verb+MakeSawtoothWave()+, \verb+MakeChirp()+
\end{desc}

\rl


\begin{desc}{Name/Symbol}
\item[Name/Symbol]	\verb+MakeSquareWave()+     

\item[Description]	NOT IMPLEMENTED.

\item[Usage]		

\item[Example]	

\item[See Also]	\verb+MakeSineWave()+, \verb+MakeSawtoothWave()+, \verb+MakeChirp()+
\end{desc}

\rl


\begin{desc}{Name/Symbol}
\item[Name/Symbol]	M\verb+akeTextBox()+

\item[Description]	Creates a textbox in which to display text. 
		Textboxes allow multiple lines of text to be rendered;
		automatically breaking the text into lines. 

\item[Usage]
\begin{verbatim}
MakeWindow(<text>,<font>,<width>,<height>)
\end{verbatim}

\item[Example]	
\begin{verbatim}
font <-MakeFont("Vera.ttf", 1, 12, MakeColor("red"), 
MakeColor("green"), 1)
tb <- MakeTextBox("This is the text in the textbox", 
font, 100, 250)
\end{verbatim}

\item[See Also]	\verb+MakeLabel()+, \verb+GetText()+, \verb+SetText()+, \verb+SetCursorPosition()+,
		\verb+GetCursorPosition()+, \verb+SetEditable()+
\end{desc}

\rl


\begin{desc}{Name/Symbol}
\item[Name/Symbol]	\verb+MakeWindow()+ 

\item[Description]	Creates a window to display things in.
		Background is specified by \verb+<color>+.

\item[Usage]		
\begin{verbatim}
MakeWindow(<color>)
\end{verbatim}

\item[Example]	

\item[See Also]	
\end{desc}

\rl


\begin{desc}{Name/Symbol}
\item[Name/Symbol]	\verb+Max()+            

\item[Description] Returns the largest of \verb+<list>+.

\item[Usage]		
\begin{verbatim}
Max(<list>)
\end{verbatim}

\item[Example]	
\begin{verbatim} 
  c <- [3,4,5,6]
  m <- Max(c) # m == 6
\end{verbatim}

\item[See Also]	\verb+Min()+, \verb+Mean()+, \verb+StDev()+
\end{desc}

\rl





\begin{desc}{Name/Symbol}
\item[Name/Symbol]	\verb+Mean()+

\item[Description] 	Returns the mean of the numbers in \verb+<list>+.

\item[Usage]	Mean(\verb+<list-of-numbers>+)	

\item[Example]	
\begin{verbatim} 
  c <- [3,4,5,6]
  m <- Mean(c) # m == 4.5
\end{verbatim}

\item[See Also]	\verb+Median()+, \verb+Quantile()+, \verb+StDev()+, \verb+Min()+, \verb+Max()+
\end{desc}

\rl



\begin{desc}{Name/Symbol}
\item[Name/Symbol]	\verb+Median()+

\item[Description]	Returns the median of the numbers in
  \verb+<list>+.  Implemented as a PEBL function.

\item[Usage]	Median(<list-of-numbers>)

\item[Example]	
  \begin{verbatim} 
  c <- [3,4,5,6,7]
  m <- Median(c) # m == 5
\end{verbatim}
\item[See Also]	\verb+Mean()+, \verb+Quantile()+, \verb+StDev()+, \verb+Min()+, \verb+Max()+
\end{desc}

\rl


\begin{desc}{Name/Symbol}
\item[Name/Symbol]	\verb+Merge()+

\item[Description]	Combines two lists, \verb+<lista>+ and \verb+<listb>+, into a single list.

\item[Usage]		
\begin{verbatim}
Merge(<lista>,<listb>)
\end{verbatim}

\item[Example]	
\begin{verbatim}
Merge([1,2,3],[8,9]) 	# == [1,2,3,8,9]
\end{verbatim}

\item[See Also]	\verb+[ ]+, \verb+Append()+, \verb+List()+
\end{desc}

\rl


\begin{desc}{Name/Symbol}
\item[Name/Symbol]	\verb+Min()+

\item[Description]	Returns the `smallest' element of a list.

\item[Usage]	
\begin{verbatim}
Min(<list>)
\end{verbatim}

\item[Example]	
\begin{verbatim}
  c <- [3,4,5,6]
  m <-  Min(c) # == 3
\end{verbatim}

\item[See Also]	\verb+Max()+
\end{desc}

\rl


\begin{desc}{Name/Symbol}
\item[Name/Symbol]	\verb+Mod()+

\item[Description]	Returns \verb+<num>+, \verb+<mod>+, or remainder of \verb+<num>/<mod>+

\item[Usage]		
\begin{verbatim}
Mod( <num> <mod>)
\end{verbatim}

\item[Example]	
\begin{verbatim}
Mod(34, 10)	# == 4
Mod(3, 10)	# == 3
\end{verbatim}

\item[See Also]	\verb+Div()+
\end{desc}

\rl


\begin{desc}{Name/Symbol}
\item[Name/Symbol]	\verb+Move()+

\item[Description]	Moves an object to a specified location.  
		Images and Labels are moved according to their center; 
		TextBoxes are moved according to their upper left corner.

\item[Usage]
\begin{verbatim}
Move(<object>, <x>, <y>)
\end{verbatim}

\item[Example]	
\begin{verbatim}
Move(label, 33, 100)
\end{verbatim}

\item[See Also]	\verb+MoveCorner()+, \verb+MoveCenter()+, \verb+.X+ and \verb+.Y+ properties.
\end{desc}

\rl


\begin{desc}{Name/Symbol}
\item[Name/Symbol]	\verb+MoveCorner()+

\item[Description]	Moves a label or image to a specified location
		according to its upper left corner, instead of its center. 

\item[Usage]
\begin{verbatim}
MoveCorner(<object>, <x>, <y>)
\end{verbatim}

\item[Example]	
\begin{verbatim}
MoveCorner(label, 33, 100)
\end{verbatim}

\item[See Also]	\verb+Move()+, \verb+MoveCenter()+, \verb+.X+ and \verb+.Y+ properties
\end{desc}

\rl




\begin{desc}{Name/Symbol}
\item[Name/Symbol]	\verb+MoveCenter()+

\item[Description]	Moves a TextBox to a specified location
		according to its center, instead of its upper left corner.

\item[Usage]
\begin{verbatim}
MoveCenter(<object>, <x>, <y>)
\end{verbatim}

\item[Example]	
\begin{verbatim}
MoveCenter(TextBox, 33, 100)
\end{verbatim}

\item[See Also]	\verb+Move()+, \verb+MoveCorner()+, \verb+.X+ and \verb+.Y+ properties
\end{desc}

\rl


\begin{desc}{Name/Symbol}
\item[Name/Symbol]	\verb+not+

\item[Description]	Logical not

\item[Usage]		

\item[Example]	

\item[See Also]	\verb+and+, \verb+or+
\end{desc}

\rl
\section{N}
\rl



\begin{desc}{Name/Symbol}
\item[Name/Symbol]	\verb+Nth()+

\item[Description]	Extracts the Nth item from a list.  Indexes from 1 upwards.
		\verb+Last()+ provides faster access than \verb+Nth()+ to the end of a list, 
		which must walk along the list to the desired position.

\item[Usage]
\begin{verbatim}
Nth(<list>, <index>)
\end{verbatim}

\item[Example]	
\begin{verbatim}
a <- ["a","b","c","d"]
Print(Nth(a,3)) 		# == 'c'
\end{verbatim}

\item[See Also]	\verb+First()+, \verb+Last()+ 
\end{desc}

\rl


\begin{desc}{Name/Symbol}
\item[Name/Symbol]	\verb+NthRoot()+

\item[Description]	\verb+<num>+ to the power of  1/\verb+<root>+.

\item[Usage]		
\begin{verbatim}
NthRoot(<num>, <root>)
\end{verbatim}

\item[Example]	

\item[See Also]	
\end{desc}

\rl
\section{O}
\rl


\begin{desc}{Name/Symbol}
\item[Name/Symbol]	\verb+or+                   

\item[Description]	Logical or

\item[Usage]		

\item[Example]	

\item[See Also]	\verb+and+, \verb+not+
\end{desc}

\rl






\begin{desc}{Name/Symbol}
\item[Name/Symbol]	\verb+Order()+

\item[Description]	Returns a list of indices describing the order of values by position, from min to max. 

\item[Usage]
\begin{verbatim}
		Order(<list-of-numbers>)
\end{verbatim}

\item[Example]	
\begin{verbatim}
	n <- [33,12,1,5,9]
  	o <- Order(n)
    Print(o) #should print [3,4,5,2,1]
\end{verbatim}

\item[See Also]	\verb+Rank()+
\end{desc}

\rl
\section{P}
\rl


\begin{desc}{Name/Symbol}
\item[Name/Symbol]	\verb+PlayForeground()+  

\item[Description]	Plays the sound `in the foreground'; 
		does not return until the sound is complete.

\item[Usage]		
\begin{verbatim}
PlayForeground(<sound>)
\end{verbatim}

\item[Example]	

\item[See Also]	\verb+PlayBackground()+, \verb+Stop()+
\end{desc}

\rl


\begin{desc}{Name/Symbol}
\item[Name/Symbol]	\verb+PlayBackground()+
 
\item[Description]	Plays the sound `in the background', returning immediately.

\item[Usage]		
\begin{verbatim}
PlayBackground(<sound>)
\end{verbatim}

\item[Example]	

\item[See Also]	\verb+PlayForeground()+, \verb+Stop()+
\end{desc}

\rl


\begin{desc}{Name/Symbol}
\item[Name/Symbol]	\verb+Pow()+ 

\item[Description]	Raises or lowers \verb+<num>+ to the power of \verb+<pow>+.

\item[Usage]		
\begin{verbatim}
Pow(<num>, <pow>)
\end{verbatim}

\item[Example]	
\begin{verbatim}
Pow(2,6)	# == 64
Pow(5,0)	# == 1
\end{verbatim}

\item[See Also]     
\end{desc}

\rl


\begin{desc}{Name/Symbol}
\item[Name/Symbol]	\verb+Print()+

\item[Description]	Prints \verb+<value>+ to stdout (the console [Linux] or the file \texttt{stdout.txt} [Windows]), and then appends a newline afterwards.

\item[Usage]		
\begin{verbatim}
Print(<value>)
\end{verbatim}

\item[Example]	

\item[See Also]	\verb+Print_()+, \verb+FilePrint()+
\end{desc}

\rl


\begin{desc}{Name/Symbol}
\item[Name/Symbol]	\verb+Print_()+

\item[Description]	Prints \verb+<value>+ to stdout; doesn't append a newline afterwards.

\item[Usage]		
\begin{verbatim}
Print_(<value>)
\end{verbatim}

\item[Example]	
\begin{verbatim}
Print_("This line")
Print_(" ")
Print_("and")
Print_(" ")
Print("Another line")
# prints out: 'This line and Another line'
\end{verbatim}

\item[See Also]	\verb+Print()+, \verb+FilePrint()+
\end{desc}

\rl

\section{Q}
\rl

\begin{desc}{Name/Symbol}
\item[Name/Symbol]	\verb+Quantile()+

\item[Description]	NOT IMPLEMENTED. Returns the \verb+<num>+ quantile of
		the numbers in \verb+<list>+.

\item[Usage]		
\begin{verbatim}
Quantile(<list>, <num>)
\end{verbatim}

\item[Example]	

\item[See Also]	\verb+StDev()+, \verb+Median()+, \verb+Mean()+, \verb+Max()+, \verb+Min()+
\end{desc}

\rl


\section{R}
\rl


\begin{desc}{Name/Symbol}
\item[Name/Symbol] 	\verb+RadToDeg()+ 

\item[Description] 	Converts \verb+<rad>+ radians to degrees.

\item[Usage]		
\begin{verbatim}
RadToDeg( <rad>)			 
\end{verbatim}

\item[Example]	

\item[See Also]     	\verb+DegToRad()+, \verb+Tan()+, \verb+Cos()+, \verb+Sin()+, \verb+ATan()+, \verb+ASin()+, \verb+ACos()+
\end{desc}

\rl



\begin{desc}{Name/Symbol}
\item[Name/Symbol]	\verb+Random()+

\item[Description]	Returns a random number between 0 and 1.

\item[Usage]
\begin{verbatim}
Random()
\end{verbatim}

\item[Example]
\begin{verbatim}
a <- Random()
\end{verbatim}

\item[See Also]		\verb+Random()+, \verb+RandomBernoulli()+, \verb+RandomBinomial()+, \verb+RandomDiscrete()+, \verb+RandomExponential()+, \verb+RandomLogistic()+, \verb+RandomLogNormal()+, \verb+RandomNormal()+, \verb+RandomUniform()+, \verb+RandomizeTimer()+, \verb+SeedRNG()+
\end{desc}

\rl


\begin{desc}{Name/Symbol}
\item[Name/Symbol]	\verb+RandomBernoulli()+

\item[Description]	Returns 0 with probability \verb+(1-<p>)+ and 1 with probability \verb+<p>+.

\item[Usage]		
\begin{verbatim}
RandomBernoulli(<p>)
\end{verbatim}

\item[Example]	
\begin{verbatim}
RandomBernoulli(.3)
\end{verbatim}

\item[See Also] \verb+Random()+, \verb+RandomBernoulli()+,
  \verb+RandomBinomial+, \verb+RandomDiscrete()+,
  \verb+RandomExponential()+, \verb+RandomLogistic()+,
  \verb+RandomLogNormal()+, \verb+RandomNormal()+,
  \verb+RandomUniform()+, \verb+RandomizeTimer()+, \verb+SeedRNG()+
\end{desc}

\rl


\begin{desc}{Name/Symbol}
\item[Name/Symbol]	\verb+RandomBinomial+

\item[Description] Returns a random number according to the Binomial
  distribution with probability \verb+<p>+ and repetitions \verb+<n>+,
  i.e., the number of \verb+<p>+ Bernoulli trials that succeed out of
  \verb+<n>+ attempts.

\item[Usage]		
\begin{verbatim}
RandomBinomial(<p> <n>)  
\end{verbatim}

\item[Example]	
\begin{verbatim}
RandomBinomial(.3, 10)		# returns a number from 0 to 10
\end{verbatim}

\item[See Also]	\verb+Random()+, \verb+RandomBernoulli()+, \verb+RandomBinomial+,
		\verb+RandomDiscrete()+, \verb+RandomExponential()+, \verb+RandomLogistic()+,
		\verb+RandomLogNormal()+, \verb+RandomNormal()+, \verb+RandomUniform()+,    
		\verb+RandomizeTimer()+, \verb+SeedRNG()+    
\end{desc}

\rl


\begin{desc}{Name/Symbol}
\item[Name/Symbol]	\verb+RandomDiscrete()+

\item[Description]	Returns a random integer between 1 and the argument 
		(inclusive), each with equal probability.  If the argument is 
		a floating-point value, it will be truncated down; if it is 
		less than 1, it will return 1, and possibly a warning message. 

\item[Usage]		
\begin{verbatim}
RandomDiscrete(<num>)
\end{verbatim}
         
\item[Example]	
\begin{verbatim}
RandomDiscrete(30) # Returns a random integer between 1 and 30
\end{verbatim}

\item[See Also]	\verb+Random()+, \verb+RandomBernoulli()+, \verb+RandomBinomial+, 
		\verb+RandomDiscrete()+, \verb+RandomExponential()+, \verb+RandomLogistic()+,
		\verb+RandomLogNormal()+, \verb+RandomNormal()+, \verb+RandomUniform()+,
		\verb+RandomizeTimer()+, \verb+SeedRNG()+    
\end{desc}

\rl


\begin{desc}{Name/Symbol}
\item[Name/Symbol]	\verb+RandomExponential()+

\item[Description]	Returns a random number according to exponential 
		distribution with mean \verb+<mean>+ (or decay 1/mean).

\item[Usage]		
\begin{verbatim}
RandomExponential(<mean>)
\end{verbatim}

\item[Example]	
\begin{verbatim}
RandomExponential(100)
\end{verbatim}

\item[See Also]	\verb+Random()+, \verb+RandomBernoulli()+, \verb+RandomBinomial+,
		\verb+RandomDiscrete()+, \verb+RandomLogistic()+,\verb+RandomLogNormal()+, 
		\verb+RandomNormal()+, \verb+RandomUniform()+, \verb+RandomizeTimer+, \verb+SeedRNG()+
\end{desc}

\rl                            


\begin{desc}{Name/Symbol}
\item[Name/Symbol]	\verb+RandomizeTimer()+

\item[Description]	Seeds the RNG with the current time.

\item[Usage]
\begin{verbatim}
RandomizeTimer()
\end{verbatim}

\item[Example]	
\begin{verbatim}
RandomizeTimer()
x <- Random()
\end{verbatim}
	     
\item[See Also]	\verb+Random()+, \verb+RandomBernoulli()+, \verb+RandomBinomial+,
		\verb+RandomDiscrete()+, \verb+RandomExponential()+, \verb+RandomLogistic()+,
		\verb+RandomLogNormal()+, \verb+RandomNormal()+, \verb+RandomUniform()+, \verb+SeedRNG()+
\end{desc}

\rl


\begin{desc}{Name/Symbol}
\item[Name/Symbol]	\verb+RandomLogistic()+  

\item[Description]	Returns a random number according to the logistic distribution 
		with parameter \verb+<p>+: f(x) = exp(x)/(1+exp(x))

\item[Usage]		
\begin{verbatim}
RandomLogistic(<p>)
\end{verbatim}

\item[Example]	RandomLogistic(.3)

\item[See Also]	\verb+Random()+, \verb+RandomBernoulli()+, \verb+RandomBinomial+, 
		\verb+RandomDiscrete()+, \verb+RandomExponential()+, \verb+RandomLogNormal()+, 
		\verb+RandomNormal()+, \verb+RandomUniform()+, \verb+RandomizeTimer+, \verb+SeedRNG()+
\end{desc}

\rl


\begin{desc}{Name/Symbol}
\item[Name/Symbol] 	\verb+RandomLogNormal()+

\item[Description]  	Returns a random number according to the log-normal 
		distribution with parameters \verb+<median>+ and \verb+<spread>+. Generated 
		by calculating median \verb!*! exp(spread \verb!*! RandomNormal(0,1)). 
		\verb+<spread>+ is a shape parameter, and only affects the variance 
		as a function of the median; similar to the coefficient of 
		variation.  A value near 0 is a sharp distribution (.1-.3), 
		larger values are more spread out; values greater than 2 make 
		little difference in the shape.

\item[Usage]
\begin{verbatim}
RandomLogNormal(<median>, <spread>)
\end{verbatim}

\item[Example]      	
\begin{verbatim}
RandomLogNormal(5000, .1)
\end{verbatim}

\item[See Also]	\verb+Random()+, \verb+RandomBernoulli()+, \verb+RandomBinomial+, 
		\verb+RandomDiscrete()+, \verb+RandomExponential()+, \verb+RandomLogistic()+,
		\verb+RandomNormal()+, \verb+RandomUniform()+, \verb+RandomizeTimer+, \verb+SeedRNG()+
\end{desc}

\rl


\begin{desc}{Name/Symbol}
\item[Name/Symbol] 	\verb+RandomNormal()+

\item[Description] 	Returns a random number according to the standard
             	normal distribution with \verb+<mean>+ and \verb+<stdev>+.

\item[Usage]       	
\begin{verbatim}
RandomNormal(<mean>, <stdev>)
\end{verbatim}

\item[Example]	

\item[See Also]	\verb+Random()+, \verb+RandomBernoulli()+, \verb+RandomBinomial+,
		\verb+RandomDiscrete()+, \verb+RandomExponential()+, \verb+RandomLogistic()+, 
		\verb+RandomLogNormal()+, \verb+RandomUniform()+, \verb+RandomizeTimer+, \verb+SeedRNG()+
\end{desc}

\rl


\begin{desc}{Name/Symbol}
\item[Name/Symbol]	\verb+RandomUniform()+

\item[Description]	Returns a random floating-point number between 0 and \verb+<num>+.

\item[Usage]		
\begin{verbatim}
RandomUniform(<num>)
\end{verbatim}

\item[Example]	

\item[See Also] \verb+Random()+, \verb+RandomBernoulli()+,
  \verb+RandomBinomial+, \verb+RandomDiscrete()+,
  \verb+RandomExponential()+, \verb+RandomLogistic()+,
  \verb+RandomLogNormal()+, \verb+RandomNormal()+, \verb+RandomizeTimer()+,
  \verb+SeedRNG()+
\end{desc}

\rl



\begin{desc}{Name/Symbol}
\item[Name/Symbol]	\verb+Rank()+

\item[Description]	Returns a list of numbers describing the rank of
  each position, from min to max.  The same as calling Order(Order(x))

\item[Usage]
\begin{verbatim}
		Rank(<list-of-numbers>)
\end{verbatim}

\item[Example]	
\begin{verbatim}
	n <- [33,12,1,5,9]
  	o <- Rank(n)
    Print(o) #should print [5,4,1,2,3]
\end{verbatim}

\item[See Also]	\verb+Order()+
\end{desc}

\rl



\begin{desc}{Name/Symbol}
\item[Name/Symbol]	\verb+Rectangle()+
  
\item[Description]	Creates a rectangle for graphing at x,y with size
  dx and dy. Rectangles are only currently definable oriented in
  horizontal/vertical directions.  A rectangle  must be added
  to a parent widget before it can be drawn; it may be added to
  widgets other than a base window.  The properties of rectangles may be
  changed by accessing their properties directly, including the FILLED
  property which makes the object an outline versus a filled shape.

\item[Usage]
\begin{verbatim}
Rectangle(<x>, <y>, <dx>, <dy>, <color>)
\end{verbatim}

\item[Example]	
\begin{verbatim}
  
  r <- Rectangle(30,30,20,10, MakeColor(green))
  AddObject(r, win)
  Draw()

\end{verbatim}
\item[See Also]	 \verb+Circle()+, \verb+Ellipse()+, \verb+Square()+,\verb+ Line()+
\end{desc}

\rl




\begin{desc}{Name/Symbol}
\item[Name/Symbol]  	\verb+RegisterEvent()+

\item[Description]	NOT IMPLEMENTED.  Advanced event loop management.

\item[Usage]		

\item[Example]	

\item[See Also]	
\end{desc}

\rl






\begin{desc}{Name/Symbol}
\item[Name/Symbol]  	\verb+Remove()+

\item[Description]  	NOT IMPLEMENTED.  Removes an item from a list

\item[Usage]		

\item[Example]	

\item[See Also]	
\end{desc}

\rl


\begin{desc}{Name/Symbol}
\item[Name/Symbol]	\verb+RemoveDuplicates()+

\item[Description]	NOT IMPLEMENTED.

\item[Usage]		

\item[Example]	

\item[See Also]	 
\end{desc}

\rl


\begin{desc}{Name/Symbol}
\item[Name/Symbol]	\verb+RemoveObject()+

\item[Description] Removes a child widget from a parent.  Useful if
  you are adding a local widget to a global window inside a loop.  If
  you do not remove the object and only \verb+Hide()+ it, drawing will
  be sluggish.  Objects that are local to a function are removed
  automatically when the function terminates, so you do not need to
  call \verb+RemoveObject()+ on them at the end of a function.

\item[Usage]
\begin{verbatim}
RemoveObject( <object>, <parent>)
\end{verbatim}

\item[Example]	

\item[See Also]	
\end{desc}

\rl


\begin{desc}{Name/Symbol}
\item[Name/Symbol] 	\verb+Repeat()+

\item[Description] 	Makes and returns a list by repeating \verb+<object>+ \verb+<n>+ times. 
		Has no effect on the object. Repeat will not make new copies 
		of the object. If you later change the object, 
		you will change every object in the list.

\item[Usage]       	
\begin{verbatim}
Repeat(<object>, <n>)
\end{verbatim}
	    	
\item[Example]     	
\begin{verbatim}
x <- "potato"
y <- repeat(x, 10)
Print(y)
# produces ["potato","potato","potato","potato","potato", 
#           "potato","potato","potato","potato","potato"]
\end{verbatim}
	     	     
\item[See Also]    	\verb+RepeatList()+
\end{desc}

\rl


\begin{desc}{Name/Symbol}
\item[Name/Symbol] 	\verb+RepeatList()+

\item[Description]  	Makes a longer list by repeating a shorter list \verb+<n>+ times. 
	Has no effect on the list itself, but changes made to objects 
	in the new list will also affect the old list.

\item[Usage]       	
\begin{verbatim}
RepeatList(<list>, <n>)
\end{verbatim}

\item[Example]     	
\begin{verbatim}
RepeatList([1,2],3) # == [1,2,1,2,1,2]
\end{verbatim}

\item[See Also]    	\verb+Repeat()+, \verb+Merge()+, \verb+[ ]+
\end{desc}

\rl


\begin{desc}{Name/Symbol}
\item[Name/Symbol]  	Replace()

\item[Description]  	Creates a copy of a (possibly nested) list in which
		items matching some list are replaced for other items.  
		\verb+<template>+ can be any data structure, and can be nested.  
		\verb+<replacementList>+ is a list containing two-item list pairs:
		the to-be-replaced item and to what it should be transformed.\\
		Note: replacement searches the entire \verb+<replacementList>+ for 
		matches.  If multiple keys are identical, the item will be 
		replaced with the last item that matches.

\item[Usage]        	
\begin{verbatim}
Replace(<template>,<replacementList>)
\end{verbatim}
			  
\item[Example]     	
\begin{verbatim}

x <- ["a","b","c","x"]
rep <- [["a","A"],["b","B"],["x","D"]]
Print(Replace(x,rep))
# Result:  [A, B, c, D] 
\end{verbatim}

\item[See Also]	
\end{desc}

\rl


\begin{desc}{Name/Symbol}
\item[Name/Symbol] 	\verb+return+

\item[Description]  	Enables a function to return a value.

\item[Usage]
\begin{verbatim}
define funcname()
{
 return 0
}
\end{verbatim}

\item[Example]	

\item[See Also]	
\end{desc}

\rl


\begin{desc}{Name/Symbol}
\item[Name/Symbol]	\verb+Rotate()+

\item[Description] 	Returns a list created by rotating a list by \verb!<n>! items.  
		The new list will begin with the \verb!<n+1>!th item of the old 
		list (modulo its length), and contain all of its items in 
		order, jumping back to the beginning and ending with the \verb!<n>!th
		item. Rotate(\verb!<list>!,0) has no effect.  Rotate does not modify 
		the original list.

\item[Usage]
\begin{verbatim}
Rotate(<list-of-items>, <n>)
\end{verbatim}

\item[Example]     	
\begin{verbatim}
Rotate([1,11,111],1)  # == [11,111,1]
\end{verbatim}

\item[See Also]    	\verb+Transpose()+
\end{desc}

\rl


\begin{desc}{Name/Symbol}
\item[Name/Symbol] 	\verb+Round()+

\item[Description] 	Rounds \verb+<num>+ to nearest integer.

\item[Usage]        	
\begin{verbatim}
Round(<num>)
\end{verbatim}

\item[Example]
\begin{verbatim}
Round(33.23)  # == 33
Round(56.65)  # == 57
\end{verbatim}

\item[See Also]     	\verb+Ceiling()+, \verb+Floor()+, \verb+AbsFloor()+, \verb+ToInt()+
\end{desc}

\rl

%%% Local Variables: 
%%% mode: latex
%%% TeX-master: "main"
%%% End: 
